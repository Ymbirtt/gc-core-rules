\part{Crafting, Research and Rituals}

\chapter{Engineering}

The Engineering skill can be used to craft various items of tech that can be used by characters. It can also be used to hack and analyse tech, however this requires a multispectrum system analyser. This is also called a ``multispec'', and will be issued as standard if you are of the Engineer, mascen Little'un Tinkerer, or vrede classes; if you are not then you must use your engineering skill to craft one. The multispec takes the form of a small portable computer which assists with a variety of engineering tasks. Please physrep it appropriately.

The skill can also be used to research new tech and teach others how to craft it. However, these abilities are restricted to certain classes (Engineer, mascen Little'un Tinkerer, and vrede) and require a workbench physrep and lammie for research, and a set of blueprints for teaching. These blueprints should be physrepped by the player, and a lammie will be provided by the crafting ref which should be attached to this physrep.

All Engineering abilities require appropriate physreps, which do not need to be fully working items; for example a simple black plastic box with wires attached could act as a diagnostic tool by plugging it into whatever you are scanning.

A crafting referee is required for crafting, teaching and researching. These will be available at certaintimes each day listed in GOD. If for any reason they are not listed, please go to GOD and ask. To use the

Engineering skill for hacking and analysing only a normal referee is required, and it can be done at any time during the game.

Please note that although Engineers can research and craft explosives, they cannot use them without the Explosives skill (see \textit{pp. 71}).

While it may seem strange to need to research technology that we have in real life today for use in a game set in the future, this reflects how much knowledge was lost to humanity in the millennium of desolation, as well as the limited resources and poor conditions in which research, crafting, and use of this tech is taking place.

Engineering Work Points (EWP)

Engineering uses work points (EWP). These points represent the skill and efficiency of the character undertaking the project, and the quality of the tools they are using. The amount of EWP a character has per day is primarily granted by their tier of the Engineering skill.

The total amount of EWP for each tier of the Engineering skill is as follows:

\begin{table}
\begin{tabular}{|l|l|l|l|} \hline 
 & Tier 1 & Tier 2 & Tier 3 \\
 \hline EWP & 4 & 6 & 8 \\
 \hline \end{tabular}

\end{table}

Some items (such as a higher quality workbench) can also add additional work points. A workbench can be used for two research projects a day, but only one character can receive the EWP bonus (assuming the workbench grants one). Workbenches can be crafted by anybody with the Engineering skill. See \textit{Appendix}

\textit{E. Crafting and Components Lists: Engineering} for the components and EWP required for this.

Additional EWP granted by workbenches is as follows:

\begin{table}
\begin{tabular}{|l|l|l|l|} \hline 
\multirow{1}{*}{}& \multicolumn{3}{|l|}{Quality of workbench} \\
\cline{2-2}\cline{3-3}\cline{4-4} & Standard & Advanced & Expert \\
 \hline EWP & 0 & 2 & 4 \\
 \hline \end{tabular}

\end{table}

\subsection{Tech Tiers}

Three tiers of tech exist to denote the varying complexity of tech items. These tiers are \textbf{standard}, \textbf{advanced} and \textbf{expert}. Items in each tier are harder to analyse, craft, repair, teach, research and hack than items in the lower tier(s), and require the appropriate Engineering skill tier.

It is possible to analyse, craft, repair, teach, research and hack tech that is higher than your Engineering skill tier, however it is extremely difficult and will take a significant amount of time and resources.

The following Engineering skill tiers are required for the tech tiers:

\begin{table}
\begin{tabular}{|l|l|l|l|} \hline 
\multirow{1}{*}{}& \multicolumn{3}{|l|}{Tech tier} \\
\cline{2-2}\cline{3-3}\cline{4-4} & Standard & Advanced & Expert \\
 \hline Skill tier required & 1 & 2 & 3 \\
 \hline \end{tabular}

\end{table}

\textit{Vrede-tech}

A subset of tech exists called vrede-tech.

This tech still falls under the three tech tiers, however is has been designed to work only for vrede and in many cases for only one vrede in particular. It is usually considerably more complex than normal tech and can only be properly researched and crafted by vrede. Vrede-tech often has unique upgrades and progression different to that of tech on the crafting lists.

It is possible to attempt to reverse engineer (see \textit{pp. 92}) this tech as a non-vrede. However it is difficult and will take a significant amount of time and resources.

\subsection{Engineering Projects}

Crafting, research, reverse engineering, teaching and maintenance are referred to as ``projects''. These are Engineering tasks that use EWP. Complex projects can take more than one day to complete and the total EWP cost can be spread over multiple days or events.

In order to use your EWP you must go to the crafting referee at a time and place indicated in GOD. You can then inform them of the projects you wish to undertake that day. They will update you on any progress of projects that you have started on previous days and assist you with any new projects.

When you go to the crafting referee please have an idea of the projects on which you wish to spend your EWP for the day. Unused EWP does not carry over to the next day and will be lost. The crafting referee will then give you an amount of time which you are required to roleplay performing your task that day. It will be between 1 and 3 hours, depending on the amount of EWP being spent and the complexity of the project.

This roleplay can be interrupted without penalty, but you must return to it as soon as possible.

Please inform a referee in your regiment when you are starting and when you have finished roleplaying your project.

A crafting referee may visit you while you are roleplaying. Use of good roleplay and props may result in a bonus to completing your project.

\subsection{Crafting}

Crafting requires the engineering skill, an appropriate set of physreps and whatever components are needed. Every item requires a \textbf{set amount of EWP to craft.} see \textit{Appendix E. Crafting and Components} \textit{Lists: Engineering} for the components and EWP cost of various items.

Anybody with the Engineering skill can craft tech from the crafting lists (see \textit{pp. 110-115}), provided they have the required tier of the skill. No blueprints are required for crafting tech in these basic lists. If an

Engineer wishes to craft tech that is not in these basic lists, they must possess the knowledge of how to craft the tech and a set of blueprints for that item of tech.

If they posses the knowledge, but do not have a set of blueprints (for instance, they have already been taught how to craft an item but have lost the blueprints), they must spend \textbf{50\% (rounded up)} of the item's EWP crafting cost to make a new set of blueprints in order to be able to craft that item again.

\textit{Ad-hoc Crafting}

\textbf{Ad-hoc crafting}: Kalelix, a vrede, wishes to produce a very powerful grenade that will repel everyone in the area in which it is thrown. He has not research how to do this and wishes to use ad-hoc crafting to produce the desired item. He spends all his EWP attempting to create this grenade and succeeds. The grenade will do as he wishes (Mass Repel), however due to the nature of the crafting it will no longer function after a day and will also cause Blast 4 on the person who throws it.

Ad-hoc crafting is a sub-set of crafting, and the way in which it is undertaken is the same as normal crafting. Any character with the Engineering skill may attempt to craft an item that performs a very particular purpose without

researching how to do so. This type of crafting still requires EWP. However, given the character's limited knowledge about what they are doing, these items cost more resources to make, are extremely unstable, most likely will not function exactly how the crafter wishes them to, and always have an extremely short period of use.

Please note that due to the nature of this type of crafting it is extremely unlikely it will produce the same results twice, and items created this way cannot be used for research.

\subsection{Research}

Research can only be attempted by the Engineer, mascen Little'un Tinkerer and vrede classes. You can attempt to research new tech or improve existing tech, however the tech tier of the item will increase with the complexity and power of the tech. In order to undertake research, a workbench physrep and lammie are required. A workbench can be used for two research projects a day, but only one character can receive the EWP bonus (assuming the workbench grants one). Workbenches may be crafted by anybody with the Engineering skill. See \textit{Appendix E. Crafting and Components Lists: Engineering} for the components and EWP required for this.

It costs \textbf{1 EWP to perform the initial research} to assess the approximate difficulty of a project and what resources are required for it. The day after this is done (or on the first day of the next event if the initial research is performed on the last day of an event), the crafting referee will inform you of the approximate difficulty of the research you are attempting. You can put more EWP into the project after the initial research, but not before you know how difficult it will be. However, you do not get this EWP back should you decide to discontinue the project after learning its difficulty. The amount of EWP required to complete a research project is not known by the player, and the referee will inform you of approximately how far off completing your research you are.

After you have completed research it is recorded in a database at GOD and you will receive \textbf{a set of blueprints} (represented by a lammie; the player should physrep these blueprints appropriately). You can then craft that item. All members of the Engineering team (see \textit{pp. 93}) who assisted you with the research receive a 50\% EWP bonus to learning how to craft the new tech item, provided they assisted with at least 25\% of the total EWP project costs. No more than two characters may receive this bonus.

New research projects can be started at some sanctioned events, however you will not know the difficulty of the project until the crafting referees have met again at a main event. This can potentially result in characters overspending EWP on a project, and any EWP that has been overspent is lost.

Research is dangerous: every time you attempt it (beyond the initial research) there is a chance something may go wrong. The crafting referee will ask you to roll a dice to determine if an issue has occured. If there is an issue, they will arrange an appropriate encounter, which will take place during your roleplay of the research.

Reverse engineering

It is possible to reverse engineer tech from either a set of blueprints or an item of tech in your possession as a research project, and this is easier than researching an item from scratch.

Reverse engineering from \textbf{blueprints} is quicker than doing so from an item of tech, however it will not produce a set another set of blueprints: it will only give you the knowledge of how to craft the item. Reverse engineering from \textbf{a tech item} will produce a set of blueprints, however it will destroy the item.

Both methods give you the \textbf{knowledge} of how to craft the item and the \textbf{ability to produce more blueprints}. In order to make a new set of blueprints, you must spend \textbf{50\% (rounded up) of its EWP crafting cost}.

    \subsection{Teaching}

Characters may be taught how to craft tech by anybody of the Engineer, mascen Little'un Tinkerer or vrede classes, who already knows how to craft the tech and possesses a set of blueprints for that tech. While

a vrede may teach in this way, it should be noted that they will not willingly teach non-vrede how to craft vrede-tech.

Only known items can be taught, and you cannot teach incomplete research. You may only teach one character how to craft one item at a time. You may not teach the crafting of an item to more than one character in a teaching session.

Teaching how to craft an item requires \textbf{100\% of its EWP crafting cost and part of the component cost}. This cost is divided equally between the student and teacher, rounded up if it is an odd number. Once a character with the Engineering skill has learnt how to craft an item of tech, it is recorded in a database at GOD and they receive a set of blueprints. As with other Engineering projects, the crafting referee will advise if the teaching will be complex enough to require being done over a number of days.

    \subsection{Maintenance}

Tech requires maintenance in order to keep it functional. Every item of tech will feature a maintenance date on its lammie, and if the item is not maintained before this date it can no longer be used. The maintenance date will generally be 1 year after the creation of the tech. \textit{Appendix E. Crafting and Components Lists:} \textit{Engineering} lists the maintenance times for many items.

You can only perform maintenance on an item if you possess the knowledge of how to craft it or a set of blueprints for the item.

Maintenance of an item costs \textbf{40\% of its EWP crafting cost and a small number of components}, depending on the tech being maintained. If you have an item that has passed its maintenance date you can attempt to maintain it, however it will cost \textbf{100\% of its EWP crafting cost and a larger number of components}.

    \subsection{Engineering Teams}

The Engineer, mascen Little'un Tinkerer and vrede classes are able to lead up to two other characters who have the Engineering skill in research and crafting projects (even though characters not of these three classes are unable to research by themselves).

Characters with the Engineering skill who assist are able to grant \textbf{up to 50\% of their unmodified EWP} to a project. Their remaining EWP may be spent on their own projects. They can only assist with one project per day.

In cases of team research, only the leading Engineer gains the knowledge and blueprints at the end of the project. However, team members (up to a maximum of two) receive a 50\% bonus to learning how to craft the new tech item, provided they assisted with at least 25\% of the total EWP costs.

\textbf{Engineering teams}: Arthur is a vrede researching how to create a device that will allow a person to use Awareness 3 without any skills for a short period of time. This is quite a hard project and would take him a significant amount of time on his own. However, he has two friends: a Medic called Lucrecia and a Sniper called Fabian, who both have the Engineering skill. They agree to assist him in his research and form a team with him as the lead.

Lucrecia is able to assist him for the whole project, however Fabian decides not to help after a while. When the project is completed Lucrecia has contributed 30\% of the total PWP and Fabian only 10\%. Therefore Arthur is able to teach Lucrecia his new tech for only 50\% of the normal teaching costs. Fabian will have to learn it for the full costs.

\subsection{Other Engineering Tasks}

Engineering tasks are distinct from Engineering projects in that they do not require EWP to perform, and do not require a crafting referee. However, they do require a multispectrum systems analyser (also called a ``multispec''), and consultation with a referee to find out the results of the task. The exception to this is repairing, which will be complete after a set amount of time roleplaying the task, unless otherwise indicated at the time by a referee. These tasks can be performed at any time during the game.

\subsection{Hacking}

Characters with the Engineering skill may attempt to hack items of tech using a multispec. Not all items can

be hacked, and the method and difficulty of the task depends on the situation.

The tier of the tech being hacked and your own Engineering skill tier affect how long or difficult this task may be. The result of the hacking depends on context. It can be anything from opening a door to reprogramming an android.

If you wish to attempt any hacking, please get a referee.

\subsection{Analysis}

Characters with the Engineering skill may attempt to analyse an item of tech in order to understand what its function may be.

The tier of the tech being analysed and your own Engineering skill tier affect how long or difficult this task may be. A multispec can be used to aid analysis, and it will reduce the time and difficulty of the task.

If you wish to attempt any analysis, please get a referee.

\subsection{Repairing}

Characters with the Engineering skill can use the Repair call, which restores armour hits to a character, in less time than somebody without the skill. If it is the repairer's own armour they wish to repair, they must take it off to do so; if worn by another character then it can remain worn during repair. It takes some time to repair armour using the Engineering skill, and the armour hits are not restored until the repairer has completed their skill count and made the Repair call (see pp. 30). Characters with the Engineering skill may be led in a team of up to three by a character with the class attribute ``Chief Engineer'' to reduce the time required. The minimum amount of time it may be reduced to is 30 seconds, taking into account all modifiers.

The length of roleplay, in seconds, required to repair 1 armour hit is given below.

\begin{table}
\begin{tabular}{|l|l|l|l|} \hline 
\multirow{2}{*}{}& \multicolumn{3}{|l|}{Armour repair times using the engineering skill (in seconds)} \\
\cline{2-2}\cline{3-3}\cline{4-4} & \multicolumn{3}{|l|}{Without engineering skill: 180} \\
\cline{2-2}\cline{3-3}\cline{4-4} & Tier 1 & Tier 2 & Tier 3 \\
 \hline Alone & 100 & 80 & 60 \\
 \hline 1 assistant & 80 & 60 & 40 \\
 \hline 2 assistants & 60 & 40 & 30 \\
 \hline \end{tabular}

\end{table}

In order to repair using the Engineering skill, you must have a physrep of items that would be used for this task. For repair performed without the Engineering skill, a character does not need physreps of engineering tools, as they will make do with what they have to hand. This is reflected by the increased time it takes forthem to repair.

\subsection{Components}

All tech requires components to craft it. Different tech requires different combinations of components. These are represented by specific printed slips provided by the system. Components are commonly found when using the Self-sufficient skill or elsewhere in the game world. They can also be purchased/traded from other players and NPC traders.

Relevant components can sometimes be used in research to reduce the overall EWP required. It is possible to research how to craft the rarer and more advanced components.

The current known components are listed below, and as the \textit{Green Cloaks} system advances it is likely that new components may be discovered.

\begin{table}
\begin{tabular}{|l|l|l|} \hline 
\multicolumn{3}{|l|}{Known components and their rarity} \\
 \hline Common & Uncommon & Rare \\
 \hline Circuit board & Battery & Emitter unit \\
 \hline Fabric & Gunparts & Powerpack \\
 \hline Gears & Liquid gas & Scanner \\
 \hline Glass & Motor & Sensor \\
 \hline Gunpowder & Organic circuits & Thermoplasma \\
 \hline Metal & Radio transmitter &  \\
 \hline Oil & Refined oil &  \\
 \hline Tubing &  &  \\
 \hline Wiring &  &  \\
 \hline Wood &  &  \\
 \hline \end{tabular}

\end{table}

\chapter{Pharmacology}

The Pharmacology skill can be used to craft and analyse various pharmaceuticals that can be used by characters.

The skill can also be used to research new pharmaceuticals and teach others how to make them. However, these abilities are restricted to certain classes (Medic and tae'go) and require a laboratory physrep and lammie for research, and a formula for teaching. These formulae should be physrepped by the player, and a lammie will be provided by the crafting ref which should be attached to this physrep.

All Pharmacology abilities require appropriate physreps. This may be as simple as coloured powder that is dipped into small bottles to test substances; please do not use any sharp objects as physreps for safety reasons.

A crafting referee is required for crafting, teaching and researching. These will be available at certain times each day listed in GOD. If for any reason they are not listed, please go to GOD and ask.

Pharmacology Work Points (PWP)

Pharmacology uses work points (PWP). These points represent the skill and efficiency of the character undertaking the project, and the quality of the tools they are using. The amount of PWP your character has per day is primarily indicated by their tier of the Pharmacology skill.

The total amount of PWP for each tier is as follows:

\begin{table}
\begin{tabular}{|l|l|l|l|} \hline 
 & Tier 1 & Tier 2 & Tier 3 \\
 \hline PWP & 4 & 6 & 8 \\
 \hline \end{tabular}

\end{table}

Some items (such as higher quality laboratories) can also add additional work points. Laboratories can be used for two research projects a day, but only one character can receive the PWP bonus (assuming the laboratory grants one). Laboratories may be crafted by a character with the Engineering or Pharmacology skill. See \textit{Appendix F. Crafting and Ingredients Lists: Pharmacology} for the PWP and components required.

Additional PWP granted by laboratories are as follows:

\begin{table}
\begin{tabular}{|l|l|l|l|} \hline 
\multirow{1}{*}{}& \multicolumn{3}{|l|}{Quality of laboratory} \\
\cline{2-2}\cline{3-3}\cline{4-4} & Standard & Advanced & Expert \\
 \hline PWP & 0 & 2 & 4 \\
 \hline \end{tabular}

\end{table}

\subsection{Pharmaceutical Tiers}

Three tiers of pharmaceuticals exist to denote their varying complexity. These tiers are \textbf{standard}, \textbf{advanced} and \textbf{expert}. Items in each tier are harder to analyse, craft, and research than items in the lower tier(s), and require the appropriate Pharmacology skill tier. It is possible to analyse, craft, and research pharmaceuticals higher than your Pharmacology skill tier, however it is extremely difficult and will take a significant amount of time and resources.

The following Pharmacology skill tiers are required for the pharmaceutical tiers:

\begin{table}
\begin{tabular}{|l|l|l|l|} \hline 
\multirow{1}{*}{}& \multicolumn{3}{|l|}{Pharmaceutical tier} \\
\cline{2-2}\cline{3-3}\cline{4-4} & Standard & Advanced & Expert \\
 \hline Skill tier required & 1 & 2 & 3 \\
 \hline \end{tabular}

\end{table}

\subsection{Pharmacology Projects}

Crafting, research and teaching are referred to as ``projects''. These are Pharmacology tasks that use PWP. Complex projects can take more than one day to complete and the total PWP cost can be spread over multiple days or events.

In order to use your PWP you must go to the crafting referee at a time and place indicated in GOD. You can then inform them of the projects you wish to undertake that day. They will update you on any progress of projects that you have started on previous days and assist you with any new projects.

When you go to the crafting referee please have an idea of the projects on which you wish to spend your PWP for the day. Unused PWP does not carry over to the next day and will be lost. The crafting referee will then give you an amount of time which you are required to roleplay performing your task that day. It will be between 1 and 3 hours, depending on the amount of PWP being spent and the complexity of the project.

This roleplay can be interrupted without penalty, but you must return to it as soon as possible.

Please inform a referee in your regiment when you are starting and when you have finished roleplaying

your project.

A crafting referee may visit you while you are roleplaying. Use of good roleplay and props may result in a bonus to completing your project.

\subsection{Crafting}

Crafting requires the Pharmacology skill, an appropriate set of physreps and whatever ingredients are needed. Every item requires a \textbf{set number of PWP} to craft. See \textit{Appendix F. Crafting and Ingredient Lists:} \textit{Pharmacology} for the PWP and ingredients required to craft various pharmaceuticals.

Anybody with the Pharmacology skill can craft pharmaceuticals from the crafting list (see \textit{pp. 116-118}), provided they have the required tier of the Pharmacology skill. No formulae are required for crafting pharmaceuticals in these lists.

If somebody with the Pharmacology skill wishes to craft pharmaceuticals that are not in these lists, they must possess the knowledge of how to craft the pharmaceuticals and a formula. If they posses the knowledge, but do not have the formula (for example, they have been taught how to craft the

pharmaceutical, but have lost the formula), they must spend \textbf{50\% (rounded up) of its PWP crafting cost}

to make a new formula in order to be able to craft that pharmaceutical again.

All pharmaceuticals will feature an \textbf{expiry date} on their card, and after this date it can no longer be used. The expiry date will generally be \textbf{1 year} after the creation of the pharmaceutical. These dates are listed in \textit{Appendix F}. \textit{Crafting and Ingredient Lists: Pharmacology}.

\textbf{Ad-hoc crafting}: Elena is a Rossii Medic with the Pharmacology skill. She wishes to make a pharmaceutical that will grant the user Tier

1 Extra Hits. Since she has not previously researched this and wishes to use it soon, she opts to attempt ad-hoc crafting. She spends all her PWP attempting to create this pharmaceutical and succeeds. However it is not exactly what she wanted, and only lasts 30 minutes as well as giving the user mood swings for the next six hours.

\textit{Ad-hoc Crafting}

Ad-hoc crafting is a sub-set of crafting, and the way in which it is undertaken is the same as normal crafting. Any character with the Pharmacology skill may attempt to craft a pharmaceutical that performs a very particular purpose without researching how to do so. This type of crafting

still requires PWP. However, given the character's limited knowledge about what they are doing, these items cost more resources to make, are extremely unstable, most likely will not function exactly how they wishes them to, and always have an extremely short period of use.

Please note that due to the nature of this type of crafting it is extremely unlikely it will produce the same results twice, and items created this way cannot be used for research.

\subsection{Research}

Research can only be attempted by the Medic and tae'go classes, providing they have the Pharmacology skill. You can attempt to research new pharmaceuticals or improve existing pharmaceuticals, however the tier of the pharmaceutical will increase with the complexity and power of the pharmaceutical. In order to research, a laboratory physrep and lammie are required. \textbf{Laboratories can be crafted by engineers or pharmacologists}. Laboratories can be used for two research projects a day, but only one can receive the PWP bonus (assuming the laboratory grants one). See \textit{Appendix F. Crafting and Ingredients Lists:} \textit{Pharmacology} for the PWP and components required to craft a laboratory.

It costs \textbf{1 PWP to perform the initial research} to assess the approximate difficulty of a project and what resources are required for it. The day after this is done (or on the first day of the next event if the initial research is performed on the last day of an event), the crafting referee will inform you of the approximate difficulty of the research you are attempting.

You can put more PWP into the project after the initial research, but not before you know how difficult it will be. However, you do not get this PWP back should you decide to discontinue the project after learning its difficulty. The amount of PWP required to complete a research project is not known by the player, and the referee will inform you of approximately how far off completing your research you are.

When you start your research you choose \textbf{three ingredients} which will be the main ingredients of the pharmaceutical. As the research progresses you may switch these as you experiment with different properties to find the best combination for your research. After you have completed research it is recorded in a database at GOD and you will receive a formula. You can then craft that item. All members of the Pharmacology team (see \textit{pp. 98}) who assisted you with the research receive a 50\% PWP bonus to learning how to craft the new pharmaceutical item, provided they assisted with at least 25\% of the total PWP project costs. No more than two characters may receive this bonus.

New research projects can be started at some sanctioned events, however you will not know the difficulty of the project until the crafting referees have met again at a main event. This can potentially result in characters overspending PWP on a project, and any PWP that has been overspent is lost.

Research is dangerous: every time you attempt it (beyond the initial research) there is a chance something may go wrong. The crafting referee will ask you to roll a dice to determine if an issue has occured. If there is an issue, they will arrange an appropriate encounter, which will take place during your roleplay of the research.

\textit{Ingredient research}

\textbf{Ingredient research}: Ines is a Medic with the pharmacology skill. She wishes to know more about the properties of an ingredient. She knows already that it has necrotic properties. She performs ingredient research and discovers that it has a secondary toughness property. This turns out to be useful for her as she can now choose to use the toughness property (rather than the unhelpful necrotic property) in her research to produce a pharmaceutical that grants additional body hits.

All the rules regarding pharmaceutical research also apply to this subset of research. When undertaking this research project you may attempt to find out exactly what

Pharmacological properties an ingredient has (see \textit{pp. 116}).

Each ingredient has three pharmacological properties that can be researched, each assigned to a different tier. You can only research the next tier of an ingredient's properties once you know the properties of all the previous tiers. Each tier is rated in ascending difficulty as per the pharmaceutical tiers. Everybody with the Pharmacology skill is granted knowledge of the first-tier properties of \textbf{six basic ingredients of their choice} when they first register with the crafting referees.

Once you have completed research into a tier of an ingredient's properties it is noted down by the referees in a database and you will receive a bonus when using that ingredient in research. With this knowledge you will be able to make informed decisions as to which ingredients to use in further pharmaceutical research, and if you know more than one property for an ingredient you will be able to choose which one you wish

to take effect in a pharmaceutical you are researching or crafting. You can teach this knowledge as per teaching any pharmaceuticals, however no formula is required or produced.

\subsection{Deformulation}

It is possible to deformulate pharmaceuticals (like reverse engineering tech - see \textit{pp. 92}) from either a formula or pharmaceuticals in your possession as a research project, and this is easier than researching an item from scratch.

Deformulating from a \textbf{formula} is quicker than doing so from a pharmaceutical, however it will not produce another formula: it will only give you the knowledge of how to craft the pharmaceutical. Deformulating from a \textbf{pharmaceutical} will produce a formula, however it will destroy the pharmaceutical.

Both methods give you the \textbf{knowledge} of how to craft the pharmaceutical and the \textbf{ability to produce more formulae}. In order to make a new formula, you must spend \textbf{50\% (rounded up) of its PWP crafting cost}.

    \subsection{Teaching}

Characters may be taught how to craft pharmaceuticals by somebody of the Medic or tae'go classes who already knows how to craft the pharmaceutical and possesses a formula for that pharmaceutical.

Only known items can be taught, and you cannot teach incomplete research. You may only teach onecharacter how to craft one pharmaceutical at a time. You may not teach the crafting of an item to more than one character in a teaching session.

Teaching how to craft a pharmaceutical requires \textbf{100\% of its PWP crafting cost and part of the ingredient cost}. This cost is divided equally between the student and teacher, rounded up if it is an odd number. Once a character with the Pharmacology skill has learnt how to craft pharmaceuticals, it is

recorded in a database at GOD and they will receive a formula. As with other Pharmacology projects, the crafting referee will advise if the teaching will be complex enough to require being done over a number of days.

\subsection{Pharmacology Teams}

The Medic and tae'go classes (providing they have the Pharmacology skill) are able to lead up to two other characters who have the Pharmacology skill in research and crafting projects (even though characters not of these two classes are unable to research by themselves).

Characters with the Pharmacology skill who assist are able to grant \textbf{up to 50\% of their unmodified PWP} to a project. Their remaining PWP may be spent on their own projects. They can only assist with one project per day.

In cases of team research, only the leading pharmacologist gains the knowledge and formula at the end of the project. However, team members (up to a maximum of two) receive a 50\% bonus to learning how to craft the new pharmaceutical, provided they assisted with at least 25\% of the total PWP costs.

\textbf{Pharmacology teams}: Isklan is a tae'go researching how to create a pharmaceutical that will allow a person to enter stealth as per Tier 3 of the Stealth skill, without having the skill, for a short period of time. This is quite a hard project and would take him a significant amount of time on his own. However he has two friends, a heavy called Anton and a sniper called Kieran, who both have the Pharmacology skill. They agree to assist him in his research and form a team with him as the lead.

Isklan picks three ingredients he knows to have chameleonic properties and starts the research with his team. Anton is able to assist him for the whole project, however Keiran decides not to help after a while. When the project is completed Anton has contributed 30\% of the total PWP and Kieran only 10\%. Therefore Isklan is able to teach Anton his new pharmaceutical for only 50\% of the normal teaching costs. Kieran will have to learn it for the full costs.

\subsection{Other Pharmacology Tasks}

Pharmacology tasks are distinct from Pharmacology projects in that they do not require PWP to perform, and do not require a crafting referee. However, they do require a referee present to witness the roleplay of the task and to inform you of its outcome.

\subsection{Analysis}

Characters with the Pharmacology skill may attempt to analyse pharmaceuticals in order to understand what their function may be.

The tier of the pharmaceuticals being analysed and your own Pharmacology skill tier affect how long or difficult this task may be.

If you wish to attempt any analysis, please get a referee.

\subsection{Ingredients}

All pharmaceuticals require ingredients to craft. Different pharmaceuticals require different combinations of ingredients. These are represented by specific printed slips provided by the system. Ingredients are commonly found when using the Self-sufficient skill or elsewhere in the game world. They can also be purchased/traded from other players and NPC traders.

Relevant ingredients can sometimes be used in research to reduce the overall PWP required.

The current known ingredients are listed below, and as the \textit{Green Cloaks} system advances it is likely than new ingredients may be discovered. More details on their attributes can be found in \textit{Appendix F. Crafting} \textit{and Ingredients Lists: Pharmacology}\textit{.}

\begin{table}
\begin{tabular}{|l|l|l|} \hline 
\multicolumn{3}{|l|}{Known ingredients and their rarity} \\
 \hline Common & Uncommon & Rare \\
 \hline Curdleclove & Dragon Hazel & Clawhorn \\
 \hline Dawn Seed & Forager's Folly & Morning Leaf \\
 \hline Greenweald & Moonbark & Root of Viskeri \\
 \hline Pink Damsel & Nightweed &  \\
 \hline Sunbright & Sprig-fist &  \\
 \hline Ubria Grass &  &  \\
 \hline \end{tabular}

\end{table}

\chapter{Thaumaturgy}

Thaumaturgy is the study of the manipulation of the power of Omega and will allow Omega power users to unlock the true potential of the Omega within themselves.

Thaumaturgy research can be performed by the Adept, mascen Little'un Shaman and myr'na Healer classes. \textbf{It requires a focus}: an object (normally of some personal significance) imbued with a sense of purpose and self by the Omega user within an Omega sphere.

This research allows you to expand and manipulate your Omega abilities in almost any way imaginable, combining them to produce new and interesting powers. For example you could refine your existing powers so that they flow more freely at your will (ie, they cost fewer FP to cast), or you could empower your touch abilities so they can be performed at range. You can even create totally new powers from being able to catch a glimpse of the future and sense ripples in the fabric of the Omega itself.

Thaumaturgy is an intensely personal experience, and as such you \textbf{may not teach others} the abilities you have learnt through it. You may attempt to guide them in their research, but you cannot contribute focus points, also called FP (see \textit{pp. 103}).

Thaumaturgy projects

In order to attempt Thaumaturgy research you must go to the crafting referee at a time and place indicated in GOD. You can then inform them of the research you wish to undertake that day. They will update you on any progress of projects that you have started on previous days and assist you with any new projects. The crafting referee will then give you an amount of time which you are required to roleplay performing your task that day. It will be between 1 and 3 hours, depending on the amount of FP being spent and the complexity of the project. This roleplay can be interrupted without penalty, but you must return to it as soon as possible.

Please inform a referee in your regiment when you are starting and when you have finished roleplaying

your project.

A crafting referee may visit you while you are roleplaying. Use of good roleplay and props may result in a bonus to completing your project.

\subsection{Research}

Thaumaturgy research requires a focus.

It costs \textbf{3 FP to perform the initial research} to assess the approximate difficulty of the project and what resources are required for it. The day after this is done (or on the first day of the next event if the initial research is performed on the last day of an event), the crafting referee will inform you of the approximate difficulty of the research you are attempting.

You can put more FP into the project after the initial research. However, you do not get this FP back should

you decide to discontinue project after learning its difficulty.

The amount of FP required to complete a research project is not known by the player, and the referee will inform you of approximately how far off completing your research you are.

After you have completed research it is recorded in a database at GOD and you will receive a lammie detailing the ability. You can then use your new ability.

New research projects can be started at some sanctioned events, however you will not know the difficulty of the project until the crafting referees have met again at a main event. This can potentially result in characters overspending FP on a project, and any FP that has been overspent is lost.

Your Omega Attunement skills are important for research. If you wish to do research into a particular area of the Omega, you must have the relevant Attunement skill. The higher the tier of that Attunement skill, the easier the research will be.

Research is dangerous: every time you attempt it (beyond the initial research) there is a chance something may go wrong. The crafting referee will ask you to roll a dice to determine if an issue has occured. If there is an issue, they will arrange an appropriate encounter, which will take place during your roleplay of the research.

\textbf{Roleplaying Thaumaturgy research}

Da Vinci is an adept who has just imbued his focus in the Omega sphere. The ritual went well and he only lost an eye. Now he has his focus he is able to start his Thaumaturgy research. He decides that he wishes to enable his melee weapons to channel the power of the Omega for a limited amount of time. With this decided he visits the crafting referee and registers his research.

To roleplay researching this he first seeks out other Adepts and Omega-sensitive beings to get their opinion on how best this could be done. After much discussion he decides that a form of the Omega Energy Attunement skill would be closest to the eventual goal he has in mind. He sets to work on the practical part of his research and roleplays attempting to channel his energy into his blade, testing the results by striking people and seeking out Omega creatures to attack. At first there is little to no effect. However, as he keeps the crafting referee updated on his research, the referee informs him that he can now start to show results, making the call once every five strikes or so. This trend continues with possible other additions until the research is complete.

Cassandra is an Adept with the Omega Spirit Attunement skill, and has recently become fascinated with the Spirit Sense: Psychometry ability. She wants to learn how to use the ability to spy on a person who picks up an item that she charges with this ability, linking its ``sight'' to hers. She begins her research by speaking with Engineers and Pharmacologists about the make-up of different materials, and what makes some materials more porous than others, or more resistant to water, etc. She wants to discover the best item to use to most effectively soak up an Omega power and retain it. Having discovered an answer, she then undertakes various experiments: she paints the item with blue chem based on the theory that this will help it take on the Omega power more readily; she asks for volunteers to pick up an item that she tries to charge with Spirit Sense, and gradually moves them further away. After many such experiments, and many days of research, she finally has a breakthrough: it's not much, but for a brief moment she glimpses the Pharmacology station of her latest experimental volunteer, stationed across the other side of the site. Now she just needs to practice.

\chapter{Ritual and the Omega}

\subsection{The Omega Plane}

Rumours and tales from the Terran past frequently told of monsters in the dark, magic, and all manner of supernatural beings. As humankind advanced scientifically it relied less and less on these stories, and little attention was paid to fanciful notions of an ``other'' world.

However, this all changed with the discovery of the Segovax cluster. When the first explorers entered

the cluster they found focal points that acted as gateways for creatures and power that seemed to have stepped right out of myth and nightmare. Daemons, spirits and all manner of monsters existed in this area of space having somehow crossed over from the Omega plane into the Segovax cluster via these focal points. The thin veil between the scientific world of fact and the supernatural world of the Omega that existed here allowed an interaction between the two planes that humans could only marvel at and grow to fear.

For reasons not fully understood the Segovax cluster seems to be a nexus between the Alpha - material - plane and the Omega, where focal points that don't exist elsewhere let the Omega in. It is a place of great danger, but also great potential, and the Terran Sovereignty was quick to begin colonisation.

Not much is known about the plane of Omega itself. Few have returned from visiting the plane with their sanity intact, and it is unwise to trust the word of the daemons there who bargain for souls. These daemons and madmen speak of a plane of creation, destruction and chaos. The laws of reality are subject to the whims of the denizens, who constantly battle for power and the chance to create their own realm to rule in the maelstrom.

Some races seem to have a natural link to the Omega and can draw forth power from the otherworldly plane to manipulate it for various uses. Humans have recently managed to emulate these abilities and make a connection of their own to the Omega through the use of technology.

\subsection{Adepts}

Recently a human was discovered augmented by arcane technology to the point where he could harness the power of the Omega. No one knows exactly where he came from or who modified him to be able to do this, but it led to the biggest leap in the human understanding of the Omega for decades. The technology was studied and reverse engineered, resulting in the Adept Program. In this program volunteers are implanted with strange devices, which themselves seem to contain some power of the Omega. Once implanted, the technology physically mutates and changes beyond the understanding of those who created it. This results in almost every Adept's implant being different, a reflection of their personality and the Omega combined. Some may be a simple display on the wrist, while others are whole limbs sheathed in living metal or billions of nanites coursing through their blood streams.

The Adept Program has allowed humans to access the power of the Omega for the first time. It comes with a cost, however. Most Adepts are mentally unstable and either need careful management or medication to remain useful to the Sovereignty. However, despite the risks they bring they are considered a resource that cannot afford to be overlooked.

\subsection{Focus Points}

The method of how beings from the Alpha plane access the power of the Omega is not clearly understood, however it is known that a high level of focus and concentration is required to do so.

Activating Omega abilities costs focus points (FP). If you are a class with FP, before time-in each day you must go to GOD and collect a set of cards that represent your FP for that day. FP do not carry over to the next day, if you do not use them they are wasted. The amount of FP you have per day depends on your class and your tier of the Focus skill (if you have it). When an Omega ability is activated the user must rip the correct amount of FP cards for that ability.

To use an Omega ability you must spend at least \textbf{3 seconds} activating the ability verbally before indicating a target and announcing the effect. You must mention the power of the Omega and the relevant Attunement in your activation; for example, ``by the power of the Omega I strike at your spirit.'' When using an ability activated by FP you must state ``Omega'', followed by the focus point cost and then the call effect eg, ``Omega 1 Knockdown''

\subsection{Omega Spheres}

The small focal points in the Segovax cluster where the power of the Omega seems to break directly through to the Alpha Plane are known as Omega spheres.

Omega spheres come in different forms, and some are easier to identify than others. In a number of places the Sovereignty has already located them and has constructed tech around them to expand their power.

Others are surrounded by alien artefacts or monoliths. Some can be as simple as a clearing in a forest. It is wise to seek out their exact location with the aid of those who are sensitive to the power of the Omega.

It is not known how these hotspots are formed, but the Sovereignty has made efforts to take full advantage of them where they can. Using technology similar to that used in the Adept Program, they have attempted to manipulate and force the spheres open, drawing more of the raw power of the Omega into the Alpha.

As a result of this, beings that can harness the Omega are able to attempt great feats of power, changing the very nature of reality to suit their whims. However, all power comes at a cost. All the spheres have a consciousness and spirit of their own, with its own drives and desires. These spirits of the spheres must besuitably impressed or entertained to gain their favour and power, and to do so they summon a being linked to the Omega to observe your effort through their eyes. Although no study has yet been undertaken into the nature of these beings, it has been noted that their species range from shapeshifters and unknown aliens to daemons and creatures of beautiful form. They are known as Witnesses.

\subsection{Ritual}

In reference to age-old Terran ideas fitting to this arcane art, the act of attempting to use an Omega sphere is known as ritual.

If you wish to perform a ritual, please inform a referee or preferably GOD \textbf{at least one hour} before you wish to do so. Once you have done this you are free to continue playing the game elsewhere and you will be informed when the sphere is ready for the ritual. Please make it clear exactly what you wish to achieve in your ritual.

In addition to the above you must also make an in-character effort to inform the Omega sphere of your intention shortly before or after you have informed GOD. The waiting period before your ritual represents by the sphere summoning a Witness to observe the ritual.

Omega spheres are generally barred to those without the power of the Omega, and it costs a varying amount of FP to unlock the sphere and allow entry. This amount increases for each individual with each use of the sphere they make each day. The amount of FP required by an individual is reset each day to the

sphere's base amount.

The ability to perform rituals is an extremely new concept to the Terran Sovereignty and even those races with natural Omega ability seldom perform them. In order to find out about exactly what you must do in these arcane rites you must seek out the knowledge in-game.

Powers and items granted by ritual are seldom permanent, unless the sacrifice is great. As such they will expire in time, the exact date being indicated on the lammie that will be given. If the lammie relates to an item you must attach it to that item; if it relates to a power or ability your character has gained, then you must wear it in the same place and way that you wear your character card. You can use ritual to renew the power or item before it expires however.

Once inside the sphere one person must lead the ritual, however any number of people may contribute to it. Rituals must last \textbf{at least 3 minutes}.

In order to gain what you wish from the sphere you must sacrifice things in return. What you choose to sacrifice can be almost anything, such as FP, money, your memory, or even your soul. What is your wish worth?

Remember to be very clear when explaining to the circle what you want and what you wish to sacrifice. One wrong word can change the meaning of a ritual significantly.
