\part{Playing the Game}

\chapter{Glossary of Game Terms}

\textbf{Calls} - There are two different types of calls: OC calls and IC calls. OC calls (see \textit{Chapter 2: Out-of-} \textit{character Game Calls}) are calls used by referees to indicate things that apply to the whole game at a given time and do not have an IC (in-character) effect. IC calls (see \textit{Chapter 8: Calls}) can be called by referees and players, and represent IC effects upon your character.

\textbf{Character} - Created by a player at the Game Organisation Desk, a character is made up of its class, skills, hits and background. It is portrayed by a player during time in.

\textbf{EWP} - Engineering work points. These represent the amount of work a character with the engineering skill can perform in a day. See \textit{Chapter 15: Engineering}.

\textbf{FP} - Focus points. These represent the amount of power an Adept, Little'un Shaman, or myr'na Healer can use each day. They are represented by cards that must be ripped when focus points are used. See \textit{Chapter} \textit{18: Ritual and the Omega}.

\textbf{Game Team} - The senior referees who oversee the running of the game.

\textbf{GOD} - The Game Organisation Desk. This is where the game is organised, where you sign in at the start of the game, create your character, and where you go during the game to perform certain actions, such as Self-sufficiency (see \textit{pp. 80}) and getting ready for monstering (see \textit{pp. 20}). There will always be a staff member in GOD during time in. If you have any questions or a problem and cannot find a ref, go to GOD where someone will be able to help you.

\textbf{IC} - In character. This refers to anything that occurs in the context of the game.

\textbf{Lammie} - Laminated card. This is a card that is attached to physreps of items that have been specially crafted, or which have special attributes. The lammie will include details of the item's name, what is does, and the registration number of that item, which matches a database kept by GOD listing the owner of the item.

\textbf{Monster} - Anyone who is not playing their own character, and is currently monstering (see \textit{pp. 20}).

\textbf{NPC} - Non-player character. An NPC is usually played by a member of the Game Team, and is directed by the plot writers. An NPC will interact with players in ways appropriate to their given skills, personality and goals. They may be an ally, enemy or something in between. The difference between a monster and an NPC is that NPCs are often recurring characters with names and goals, whereas monsters are not recurring characters.

\textbf{OC/OOC} - Out of character. This refers to anything that relates to the real world. If you see someone with their fist in the air, they are not there and are considered OC. Players must generally have a good reason for going OC. Do not put your fist in the air to stop your character from dying or being injured: this is cheating. If you are IC and see someone with their fist raised you must ignore them. (Please note: this

is not to be confused with somebody holding 1, 2 or 3 fingers in the air: they are using the Stealth skill and therefore are IC but hidden. See \textit{pp. 80-81 for further details}.)

\textbf{Physrep} - Physical representation. This refers to the real world representation of an object in the game world. This is any prop you would need including weapons, armour and any equipment. If there is any doubt, ask a referee. The physreps for guns that are used in \textit{Green Cloaks} are Nerf guns and other foam dart guns. Airsoft weapons are \textbf{not} allowed in \textit{Green Cloaks}. External modifications, such as painting weapons, are allowed within reason. All functional modifications must be declared and will be thoroughly checked before being allowed in the game. If a weapons checker deems your weapon unsafe, it is up

to you to supply a new physrep. Please see \textit{Appendix J. Blastersmiths UK Blaster Safety Guidelines} for advice on blaster safety.

All melee weapons and shields must be LARP-safe, in that they must be designed for use in close combat

without causing harm to anyone when used correctly (see \textit{Chapter 3: Safety}).

\textbf{Player} - Any person who is currently playing their own character.

\textbf{PWP} - Pharmacology work points. These represent the amount of work a character with the Pharmacology skill can perform in a day. See \textit{Chapter 16: Pharmacology}.

\textbf{Referee/Ref} - Identified by their high-visibility vests. They help run the game and make sure everything from combat and roleplay to crafting and out-of-character issues are all resolved smoothly. They are the first point of contact for players who have questions regarding rules or who need help and direction. If you have any issues about the game, consult with a referee. If a referee makes a decision, please abide by it.

\textbf{Sanctioned event} - an event that is not one of the four main events run by the Game Team each year. It is usually an event designed for a specific regiment, but also open to players from other regiments. Attending a sanctioned event will not grant you the skill point reward of a main event, but Engineering, Pharmacology and an Omega sphere will sometimes be available.

\textbf{Tech} - A term used to refer to various technologies that can be used, researched and crafted in the \textit{Green Cloaks} universe.

\textbf{Time in} - The time of day during which the game is played. It is used to indicate the start of each game day. During time in everything in an IC area should be done IC, and OC interactions should only be for necessary clarifications of in-game events or safety concerns. During this time everything that takes place is canonical (ie, has happened in the game world), and players are free to move around and interact with each other in character.

\textbf{Time out} - During this time no IC interactions may take place. It is used to indicate the end of each game day and the end of the event. During this time all IC areas are treated as OC.

\textbf{Weapons checking} - All weapons and shields must be weapons checked at the start of every event and before every large battle. This process involves a referee or designated weapons checker checking the weapon or shield visually and through handling to ensure that it is safe for use. All blasters must be

weapons checked by \textit{Blastersmiths UK}, who will tag them if they pass the check. A weapons checker may perform random weapons checks if they notice a weapon or shield that they have concerns about, or if somebody brings a particular item to their attention. All weapons checkers undertake training, which is recorded by the system, to ensure the highest level of weapon safety possible.

\chapter{Out-of-character Game Calls}

If you hear a referee or a player call these words, act accordingly. Generally, these will be called by a referee with the exception of ``Safety'', which can be called at any time by anybody.

\textbf{Time in} - Everyone is now in-character and the game is in progress.

\textbf{Time out} - The game is over for the day. Everyone is now out of character and all IC interaction should cease.

\textbf{Safety} - If you hear this, stop what you're doing immediately and kneel where you are. If you are the person who has made the call, you must remain standing (if able). This is to allow referees or safety personnel to quickly see and get to OC injuries during play.

\textbf{Time freeze} - This call means that you have to close your eyes and hum loudly until someone calls ``Time In''. This usually represents something happening instantaneously in-game, such as teleportation. Your character is unaware of its occurrence until ``Time In'' is called again.

\chapter{Safety}

\section{Combat Safety}

Safety should be a primary concern for everyone. Please follow these rules to ensure the safety of yourself and other players:

\begin{enumerate}[1]

\item Be sensible.

\item Do not hit anybody with anything that is not LARP-safe.

\item Do not stab with a melee weapon, unless it is stab-safe. A stab-safe weapon has a compressible striking area. You should be aware if your weapon is stab-safe, but please consult with a referee.

\item If you wish to use a stab-safe weapon, you must register it at GOD and pass a stab-safe competency test before time in.

\item Do not block or parry with a gun.

\item Do not attempt to strike or fire around corners or objects where you cannot clearly see who you

are attacking: somebody may be closer than you think.

\item Pull your blows with melee weapons. To pull your blow, as your weapon hits the target pull it back so that it does not make full-force contact. Combatants are aiming to tap their opponent, not hurt them. Never hit somebody with full force.

\item Wrestling and physical grappling are not allowed. There are in-game mechanics that allow for the restraint and movement of resisting and unresisting characters: please follow these instead of acting them out with full force.

\item Do not aim any attacks at the groin area or eyes. Accidents happen, but please be as diligent in avoiding them as possible. If you are concerned about your eye safety please wear goggles.

\item You must have your character card visible on your person at all times. Please note that this card is not visible IC, but may be used OC in the event of an emergency.

\item Players are not permitted to use any form of pyrotechnics during an event, or while on a site being used by \textit{Green Cloaks}. Limited and controlled pyrotechnics are sometimes used by trainedpersonnel. If you have issues with smoke machines, loud noises, or flashes, speak to GOD when you register at the beginning of an event.

\item There are trained first aiders on site at all times. If you or somebody else is injured, please call a referee and they will bring a first aider to the injured person. If it is during time-out, please go to GOD to seek assistance.

\item Any illegal activities or actions taking place at a \textit{Green Cloaks} event will result in the incident being immediately reported to the police, and the player being banned from the system.

\end{enumerate}

A safety briefing and demonstration on how to fight safely will be given at the start of each event, and any referee will be happy to assist you further with this. If you have any concerns during time-in, or witness unsafe combat, please speak with a referee when it is appropriate to do so, and they will address the problem.

Non-combatants

If you are unable to take part in combat for a medical reason, you are still able to take part in the game as a non-combatant character. These characters can still play vital roles in the Green Cloaks regiments, and can take on roles such as (but not limited to) diplomats, engineers, pharmacologists, pacifist healers, quartermasters, or war correspondants. As a non-combatant you are unable to take part in any battles that take place, and should you find yourself in a combat situation (eg, during a camp attack) you must get to a place that is away from the combat. It is up to you to avoid being hit. If you are unable to move away safely, or are threatened by an attacker with any weapon or attack, or struck, you must immediately lay down on the ground and begin your 120-second death count (see \textit{pp. 24}). This is not because your

character is necessarily weaker than others, but so that you are out of harm's way and will not be attacked further.

Please note that a medically non-combatant character is still affected normally by all other game effects and damage. The only difference is when they are the target of direct physical combat. For example, a non-

combatant who is the subjet of the Grapple or Cripple call should react as per the normal rules.

OC non-combatants will wear a luminous wristband supplied by the Game Team to indicate this, and the monster team will be made aware.

You may choose to play a pacifist or non-physical character even if you are physically able to take part in battles, and in these cases you will not wear a wristband, and any attacks or damage you sustain will be taken in the same way as any combat character.

\subsection{Alcohol}

If you are aged 18 or over you may consume alcohol that you have brought with you to an event. There may be a fully-licensed IC bar at \textit{Green Cloaks} events. The bar will not be able to serve you alcohol without valid ID, so please bring some with you if you wish to purchase alcohol. If you are under the age of 18

you may not consume alcohol at any \textit{Trinity Games} events, and if found doing so you will be subject to all relevant UK law, as well as being ejected from the event immediately. Anybody found supplying alcohol to those under the age of 18 will face the same penalty.

\subsection{Young Players}

\textit{Green Cloaks} games, unless otherwise indicated for a specific event, are open to anybody aged 16 or above. For anybody below the age of 18, a consent form (see \textit{Appendix L. Parent / Guardian Consent} \textit{Form}) must be completed and signed by the player's parent/guardian after the parent/guardian and player have read the Disclaimer (see \textit{Appendix K. Disclaimer}), and handed it to the Game Organisation Desk at registration. The player will also need to sign the same Disclaimer form as other players. Players under the age of 18 may not consume alcohol at an event or be supplied with it. For the purposes of the game they are otherwise treated the same as an adult player.

\chapter{Acceptable Behaviour}

\subsection{Respect for Other Players and Honour}

Every LARP game is based on trust. Everyone playing the game is responsible for playing it safely and sensibly. Referees will be monitoring the game to make sure that people are playing by the rules. Players must also be aware of other players' comfort zones and must respect each other's property.

Playing by the rules is important as it makes the game more enjoyable and interesting for everyone. If something bad happens to your character, be honest. Anything else is cheating and no-one likes a cheater! Play the game as you would want others to play it.

\subsection{Character Death}

Character death in the \textit{Green Cloaks} system is a common occurrence. Characters are routinely put into life- threatening situations, as befits soldiers on the front lines of battle. Losing a character can be sad for theplayer, as a lot of effort and time has been put into kit and costume, developing a personality and in-game contacts, and gathering renown and achievements over time. However, we ask that you also see character death as an opportunity to start afresh, create something new, and try out an aspect of the game you may not yet have experienced. Please remember that although your character is dead, their friends and fellow soldiers will mourn them and remember them.

If you are stuck for a new character concept following character death, or need some time to think about it, why not visit GOD and have a chat? They will be more than happy to assist you with ideas! They can also get in touch with the referee for the regiment your new character wishes to join to help create a memorable introduction for them.

In-game Theft

Stealing in an out-of-character capacity, eg from a player's tent or another OC area, is not allowed under any circumstances. This includes stealing a physrep and not returning it to GOD. Theft of a player's property is treated seriously in \textit{Green Cloaks} and will be reported to the police. However, there is a mechanic to allow in-game theft of items with a laminated card (lammie) attached to them, or resources such as herbs or dermograft patches that are represented by printed pieces of paper supplied by the system. This allows the in-character roleplay of one character stealing an in-game item from another character. All personal physreps of in-game items (guns, weapons, lock boxes, etc) must be returned to GOD. System physreps (lammies or paper slips) are kept by the character that has stolen them.

It is advisable to ask a referee to be present if you plan to steal an item in-game. In all instances of in-game theft you must immediately take the item to GOD so that the previous owner may be made aware, and the personal physrep returned to them safely. GOD will also arrange for the database of items to be changed to reflect the new ownership. Each lammie has a registration number on it which matches a database of ownership kept by GOD. If you are found in possession of an item with a registration number that does not

have your name attached to it on the database without the permission of the owner and you are not on your way to GOD to report it stolen, it may be treated as cheating.

It should be noted that if an item is stolen in-character and the physrep is returned, your character no longer has the item. You may use the item that you used to physrep the stolen item, but it must be treated as a different item. However if you are happy for the thief to use the physrep, this can be declared to the referee when it is returned to you (note that this may make it easier to recover the item in-game). All physreps loaned in this way must be returned to their correct owner or GOD at the end of the event, and the player who stole it is responsible for this.

In-character and Out-of-character Areas

In-character (IC) areas are where the events of the game take place. Most of the site being used will be IC. This generally includes all the regimental camps, the paths between, and the entire area surrounding them. Anyone you see in these areas should be treated as IC unless they are clearly not (eg, wearing a hi-vis vest or are a member of the public). In those situations you should ignore them, and if they are not part of the game please be courteous. The sites used by \textit{Green Cloaks} are large and sometimes shared by other groups. Member of the public may ask questions about what we are doing. Feel free to explain! (If you see a player doing this, please treat them as OC.) Anything else that takes place in an IC area during time in should be seen as IC unless there is a very good reason not to do so. During time in, OC chatter should bekept to a minimum and any OC interactions should take place in an OC area.

Out-of-character (OC) areas, are any places not being used as IC areas. This will include GOD, the nearby area (which will be clearly marked, OC camping and toilet facilities. The specific OC areas will likely be different between events and individual sites, so they will be clearly defined at the start of each event. Unless otherwise specified, please treat all areas as IC. In OC areas everything should be taken as OC, and it is here that OC situations can be dealt with. Please note that the eating area may be IC or OC depending on who is providing catering, so unless it has been defined as an IC area please keep IC interactions to a minimum. If it is not clear, please assume the eating area is OC. Note that this does not apply to any eating areas in regimental camps, as these are always IC.

\subsection{Monstering}

When you attend LARP event you can generally expect to ``monster'' at some point during the event. Thisis where you get to be the bad guy, characters that create plot for the players, or threats posed by the environment. Monstering roles are varied, so expect to play many different characters during a monstering slot. Non-combatants are able to take part in monstering, and will be given non-combat roles suitable for them, eg roleplay encounters, and accompanied by a referee.

At every mainline event each regiment will be assigned a ``monster slot'': usually a period of two hours during which the regiment's camp is considered ``time out'' and all players in that regiment are asked to remove any items and costume that distinguishes them as their current character, replacing it with generic dark clothing. All players taking part are then considered ``monsters'', and must report to GOD.

While monstering, be gracious. You are not trying to kill other players by any means necessary, but instead trying to give them a memorable game. As a monster you may be asked to engage in combat encounters or roleplay encounters. Try and make every encounter as engaging for the other players as possible. They'll be giving you the same consideration when they monster.

\subsection{Referees}

Referees can be identified by their high-visibility vests. The referees will guide players through missions and oversee battles, as well as handling IC and OC disputes or rules ambiguities. There are also designated crafting referees that oversee research and crafting (see \textit{Part 3: Crafting, Research and Rituals}). Please bear in mind that referees are not actually there IC. If you see a referee, ignore them IC. There may be instances where a referee informs you of something your character has discovered or can see/hear, and

in these cases please do not react to the referee but instead roleplay a suitable reaction to the information. Referees accompany and manage monster slots, and during these please be attentive to all briefs given to you by the referee, and any further instructions they give you during an encounter.

There are three referees in each regiment. These are generally the Colonel and Captain of the regiment, and one other person that is chosen by Game Team. Between them there will be two radios, and there will be at least one of these radios in camp at all times.

\chapter{Chapter 5: Encounters}

Definition of an Encounter

You will notice as you read this rulebook that there are many references to the term ``encounter''. These normally take the form of ``You can do X things per encounter'' or ``You regenerate X global hits after each encounter''. This mechanic has been introduced to reduce the need to actively count in your head while in combat.

An encounter is defined as the \textbf{time between periods of recovery}, a period of recovery being:

\begin{itemize}
\item at least 120 seconds where the character is out of harm's way and actively roleplaying attempting to recover their energy

or

\item at least 300 seconds where the character is out of harm's way.

\end{itemize}
Note that being in a dying state \textbf{does} count towards the 300 seconds as long as the character is out of harm's way, but it \textbf{does not} count towards the 120 seconds as it is not actively roleplayed recovery. If an item's ability is being recharged you must spend at least some of this time roleplaying recharging it.

\textbf{Active roleplay for 120 seconds}: Miriam's character, Kendra, has been fighting with two swords in skirmish lines. She has been wounded on all locations, and has used her melee encounter call, Through, to take down an opponent. She steps out of the lines and seeks the triage station. As the medics begin to roleplay healing her, she cries out in pain. Breathing deeply she roleplays trying to recover: for 120 seconds she talks herself through the pain and psychs herself up. She flexes the muscles in her right arm, massaging it gently, and when she has full use of her arms she roleplays a few precise test swings with her swords. By the time the medics have finished working on her, she's raring to go again, and has regained use of all calls she is able to use, including ``Through'' with a melee weapon.

\textbf{Out of harm's way for 300 seconds}: Robert's character, Sgt. Michael Fuller, has been sitting around the regiment's campfire enjoying a cup of tea. Suddenly a group of One Bakkar attack the camp, and he begrudgingly places the tea down and picks up his rifle. As the enemy strike at the gates he fires several shots, and uses the Stun call twice during the encounter (as a Trooper with Tier 2 Rifle he is able to make two calls per encounter). When all the creatures are dead, Sgt. Fuller returns to his tea - still hot - and sits back down. Several minutes later a second group of One Bakkar attack - perhaps a backup squad sent to retake any captured comrades - and he is able to use the Stun call twice again.

\chapter{Chapter 6: Hits}

Hits are the way in which the amount of damage a character can take in-game is measured. Each character has \textbf{six locations} that take damage from hits. These locations are: \textbf{left arm}, \textbf{right arm}, \textbf{left leg}, \textbf{right leg}, \textbf{head} and \textbf{torso}.

Every melee attack or foam dart that strikes a character does \textbf{1 point of damage} to the location

that it hits. For melee attacks to count, each blow must be made with a full swing of the weapon: ``drum rolling'' (where the weapon is not drawn back far but instead tapped very quickly on the target like a drummer's drumstick) is not allowed.

Hits are divided into three types that each work differently: \textbf{body hits}, \textbf{armour hits} and \textbf{global hits}.

\subsection{Body Hits}

Body hits represent the character's physical toughness. Most starting characters have by default 2 body hits per location. The exceptions to this are the mascen Little'un Shaman (1 body hit per location), mascen Little'un Tinkerer (1 body hit per location) and mascen Big'un (3 body hits per location). This can be modified using skills.

Body hits have the following attributes:

\begin{itemize}
\item If any location (left arm, right arm, left leg, right leg, head or torso) is reduced to 0 body hits then it is incapacitated and cannot be used until it has been healed by a character with the medicae skill or other means of healing.

\item If the head or torso location is reduced to 0 body hits the character is dying and the player must begin a 120-second death count. The character must carefully fall to the floor, releasing their grip on any items they are carrying once they have done so. Please consider your safety when falling to the floor.

\item If the head is reduced to 0 body hits then the character falls unconscious (and is dying).

\item If the torso is reduced to 0 body hits then the character is still conscious (but dying) and can call out (in agony) for help. You can crawl very slowly in this state (as long as you still have use of at least one arm), but cannot use any items or skills. Any further hits to any location (regardless of any remaining

hits of any type) while in this state will render the character unconscious. Please roleplay this state: yourcharacter will be in a lot of pain.

\item Body hits are \textbf{taken last}, after global hits and armour hits.

\end{itemize}
Please note that if one leg location is reduced to 0 body hits you may not hop, but you may drag yourself using your arms, as long as you have use of at least one arm.

For more information about death and restoring body hits (healing and stabilisation), see \textit{Chapter 7: Death,} \textit{Healing and Stabilisation}.

\subsection{Armour Hits}

Armour hits represent the protection granted by armour that you are wearing. They are granted by wearing a physical representation of armour. In order for a location to count as armoured and to be granted the armour hits, at least 50\% of that location must be covered by the armour.

Armour hits have the following attributes:

\begin{itemize}
\item If a location is reduced to 0 armour hits then any additional hits on that location are taken from the character's body hits. The armour will grant no further protection until repaired.

\item Armour does not stack: you can only gain the benefit of the highest grade of armour you are wearing. Although you may wear additional layers for costume or comfort purposes, you may not count those additional layers as armour once the highest grade is destroyed.

\item Armour damage is always taken before body damage, unless an accompanying call specifically says so

\end{itemize}
(see \textit{Chapter 8: Calls}).

Armour hits are \textbf{taken second}, after global hits and before body hits.

For more information about types of armour, repairing armour hits and physical representations of armour, please see \textit{pp. 82-83}.

\subsection{Global Hits}

Global hits represent the protection granted by means that protect the body as a whole, rather than just a certain location.

Global hits have the following attributes:

\begin{itemize}
\item Global hits are not locational. A hit anywhere on a character will result in the loss of a global hit.

\item Global hits regenerate.

\item Global hits granted from different sources may be reduced in any order when attacked.

\item Global hits granted by items such as energy shields regenerate at the rate specified on the item that

granted them.

\item Global granted by the dodge skill hits fully regenerate to the maximum hits granted by the skill after each encounter.

\item Global hits granted by the dodge skill cannot be taken on any attack from behind or from a hidden ranged attacker (you have to be able to see the attack coming to avoid it).

\item Global hits granted by the Dodge skill may only be applied if you can move your body of your own accord (not when you are being grappled, unconscious, paralysed or controlled).

\item Global hits are \textbf{taken first}, before armour hits and body hits.

\end{itemize}
\chapter{Dying, Healing and Stabilisation}

Death and Dying

If a character is reduced to 0 body hits on the head or torso they are dying. While in the dying state you cannot use any skills or items.

The time before death is \textbf{120 seconds}. This is referred to as your death count. As soon as you reach the dying state you must begin counting from 0 to 120: it is important that you do this and that you remember your count. The death count will always start from 0 each time you enter the dying state: it does not carry over each time you start dying. Do not count out loud as it is a personal game mechanic that other players should not be able to track.

\begin{itemize}
\item If the head is reduced to 0 body hits then the character falls unconscious (and is dying).

\item If the torso is reduced to 0 body hits then the character is still conscious (but dying) and can call out (in agony) for help. You can crawl very slowly in this state (as long as you have use of at least one arm), but cannot use any items or skills. Any further hits to any location (regardless of any remaining hits

of any type) while in this state will render the character unconscious. Please roleplay this state: your character will be in a lot of pain.

\end{itemize}
The death count may be extended by \textbf{stabilisation} (see below) and is cancelled when both your head and torso body hits have been restored to at least 1.

The character will return to consciousness when they have at least 1 body hit on their head (but will still be dying if the body hits on their torso have not been restored to at least 1 as well).

If a character of the Medic class inspects you as you are dying, you may tell them your death count after they have declared they are diagnosing you and roleplayed doing so for 5 seconds. This is an out-of- character mechanic representing their skills at diagnosing your wounds.

If you reach the end of your death count then your character has died. If there are still players in the area please remain where you are, as they may wish to recover your body. If there are no players in the area you may put your fist in the air as per the out-of-character gesture (see \textit{pp. 14}) and seek out a referee for assistance. If no referee is present please return to the GOD desk to generate a new character. Please note all in-character items (dermograft patches, IC money, crafted items or similar) must be handed into GOD upon death if the items were not removed from your corpse by another character.

Please note that character death in the \textit{Green Cloaks} system is common. While it is natural to be sad at the loss of a character, we ask that you also see it as an opportunity to create a new one, with new roleplay opportunities, personality, skills and relationships. If you are affected strongly by the loss of your character, you could visit GOD for a chat, and then start thinking about what you'd like to play next.

Healing and stabilisation

\textbf{Stabilisation} may be performed more than once on a character during the same instance of dying, and each time it is used it \textbf{resets the death count to zero and extends it to 240 seconds}. Note that if one of the below methods of stabilisation is used on a character whose death count is already extended beyond 240 seconds (for instance, by another item), the stabilisation sets the death count to 240 seconds regardless.

There are three main ways to be stabilised: the \textbf{Medicae skill} (see \textit{pp. 74}), a \textbf{dermograft patch} (see \textit{pp.}

or the \textbf{Omega Body Attunement skill} (see \textit{pp. 75}). When a dermograft patch is used stabilisation is instant. The other two methods take some time to be applied, and while they are being applied your

death count continues (you may still die). Your death count does not reset and change until the healer has completed using the skill and made the Stabilise call (see \textit{pp. 31}). Please see \textit{pp. 74} for the times required

for stabilisation with the Medicae skill.

Healing restores body hits to a character. There are three main ways that this can be done: the \textbf{Medicae skill} (see \textit{pp. 74}), a \textbf{pharmaceutical} (see \textit{pp. 116-117}) or the \textbf{Omega Body Attunement skill} (see

\textit{pp. 75}). Each restores a different number of body hits to different locations. The character healing you will tell you exactly what has been restored. Please see \textit{pp. 74} for the times required for healing with the Medicae skill.

Pharmaceuticals generally cause hits to be restored instantly, however some do not. The character who gives you the pharmaceutical will tell you the exact effects. The other two methods take some time to be applied, and while they are being applied your death count continues (you may still die). For more on healing and the Heal call, see \textit{pp. 28}.

The Heal call can also be used to restore consciousness, or limb usage, to a character under the effects of the Subdue call (see \textit{pp. 31-32}). The location and amount of hits restored does not matter, 1 use of the call will restore all usage to all locations/consciousness affected by the Subdue call.

\chapter{Calls}

Calls are the OC mechanic used to indicate certain IC effects. The call cannot be heard in-game by characters, however they may respond to it as if they had heard an appropriate sound, eg, the ``Marksman'' call cannot be heard, but the distinctive shot can be.

If one of these calls is used against you then react accordingly. Calls are indicated by a player saying the name of the call clearly and either striking another character with the relevant weapon or pointing at the character (depending on the call) to indicate the intended target.

Characters may only use calls if they have a relevant skill or item that allows them to do so. The four exceptions to this are the \textbf{Execute}, \textbf{Grapple}, \textbf{Repair} and \textbf{Subdue} call, which any character may use.

The following rules apply to calls:

\begin{itemize}
\item Unless otherwise indicated, all damage is taken in the default hits order: global hits, armour hits, body hits.

\item Unless otherwise indicated, all calls delivered through a physical attack cause 1 point of damage.

\item Some of these calls bypass certain hit types. If a call bypasses a hit type, it means that the character still has the hits granted by the bypassed type.

\textbf{Roleplaying blocking a call}: Jim is playing a heavy called Seamus with the bulging biceps skill and a shield. He charges an enemy who fires his pistol at him, calling ``Knockdown!''. Seamus blocks the shot with his shield, however he roleplays the force of the shot by slowing down and bracing for a second.

\item Calls that are applied via a physical attack can be blocked (by melee weapons or shields) unless indicated otherwise in the description. However if you do block them please react as if you have just blocked a particularly strong shot or challenging blow.

\end{itemize}
Calls that cause Location to Zero may only deal a lot of damage to some tougher monsters, instead of dropping the targetted location to zero hits.

Please note that locations are incapacitated when your \textbf{body hits} are reduced to 0, even if you have other types of hits left.

All calls must be made loudly and clearly. Do not whisper. The only exception to this is when using Omega powers, which you may choose to covertly cast, however this requires the presence of a ref to ensure the target is aware of the effect.

\subsection{Awareness}

When this call is made it will be followed by a number to indicate the skill tier of the Awareness skill that is being used.

If this call is made within 30 feet of you and its tier is high enough to locate you in stealth (see \textit{pp. 80-81}) you must clearly state your Stealth tier so the caller can hear you. If you are hidden, but not stealthed, you must say ``0''.

Please see \textit{pp. 67} for further information on use of the Awareness skill.

\subsection{Befriend}

The target becomes friendly towards the caller for 120 seconds.

This call does not grant any level of control over the target and they do not feel less friendly towards their existing friends.

\subsection{Blast}

The target takes 1 hit to every location and also takes a Knockdown (see \textit{pp. 29}) call effect.

If this call is followed by a number, then you take that many hits to every location. For example, ``Blast 3'' would result in 3 hits to every location and a Knockdown call effect.

If the target has global hits then the first level of Blast will reduce them to 0 and every subsequent level will

cause 1 hit to every location. The target will be knocked down regardless.

\subsection{Blind}

The target must close their eyes for 5 seconds as if temporarily blinded. (Please do not make any attempt to fight/attack while in this state. If you cannot see then it is unsafe to do so.)

\subsection{Blink}

This call represents a short teleport.

The caller makes the call ``Blink'' and holds their hand high in the air, with the palm open. The caller cannot be seen or heard while the open palm is in the air, except by those under the effects of the Spirit Sense call (see \textit{pp. 31}). They may travel at any speed for a maximum of 15 seconds, before dropping their hand and becoming visible. This is indicated by lowering the hand and making the Blink call again.

\subsection{Bolt}

Bypasses global and armour hits.

The target takes 1 hit to a specified location. If no location is specified the hit is to the torso.

If this call is followed by a number, then you take that many hits to the specified location. For example, ``Bolt 3'' would result in 3 hits to the specified location.

\subsection{Cripple}

Bypasses global and armour hits.

Can only be used on limb locations (left leg, right leg, left arm, right arm). The body hits on the location struck or indicated are reduced to 0. If the call accidentally strikes the head or torso, then it counts as a single hit of damage (taken in the default order) to that location.

\subsection{Courage}

This call is generated by the Courage skill (see \textit{pp. 69}) and is used to resist the Fear call (see \textit{pp. 28}) and negate its effects. This call can also be used with the addition of ``Resist terror'' in order to resist a Terror call at Tier 3 of the Courage skill.

\subsection{Crush}

Bypasses global hits.

The armour hits on the location struck or indicated are reduced to 0. If the armour hits on the location are already 0 or the target has no armour on the location, then the body hits on the location are reduced to 0. If delivered with a physical attack the armour, melee weapon or shield that is struck is then damaged and cannot be used again until it has been repaired as per repairing 1 armour hit (see \textit{pp. 82-83}).

\subsection{Disarm}

The target (carefully) drops one item, indicated by the caller, that they are currently holding. It cannot be picked up in the same hand for 3 seconds.

If this call is made using a physical attack, then the caller must strike either the arm that is using the item indicated, shield or melee weapon for the call to take effect.

Do not attempt to hit non melee-safe items (such as guns). If you wish the disarm call to affect a non-melee safe item, you must strike the arm that is using the item.

\subsection{Execute}

Bypasses global and armour hits.

This call may only be made against unconscious, unresisting (asleep, paralysed, etc, and therefore unable to defend themselves) or grappled targets.

When this call is made with a gun it must be at close range (within 3 feet). Given the close range, a physical projectile must not be fired: instead simulate a shot and make the call.

If this call is made with a melee weapon you must strike the head or the torso location. Given that the target is immobile, please make this a very light blow. You must spend at least 3 seconds roleplaying preparing the blow or lining up the shot before calling ``Execute'' and striking your blow or making your (simulated) shot. Once the call is made, the location hit (head or torso) is reduced to 0 body hits, the target is rendered unconscious and their death count is reduced to 15 seconds remaining before death. If the target's death count is already below 15 seconds, then it is reduced to 0 and they have died.

\subsection{Fatal}

All hits on all locations are reduced to 0 and the target's death count is reduced to 60 seconds when

incapacitated by this call (this does not carry over to subsequent instances of entering the dying state: death count reverts to the normal 120 seconds). This reduction to 60 seconds takes place regardless of any items that the target may be wearing that give them an increased death count.

This call cannot be blocked and will take effect even if parried.

\subsection{Fear}

The target must run directly away from the source of the call for 30 seconds. If the target is unable to run away (because they are trapped/restrained IC or unable to for OC reasons), they must cower and may only defend themselves for 30 seconds.

This call can be resisted with the Courage call (see \textit{pp. 27}).

\subsection{Flaming}

Bypasses global hits.

The target must roleplay as if they had been set on fire.

The target takes 1 hit per location after 15 seconds, and every further 10 seconds until the call is ended. To end the call the target must roleplay attempting to put the fire out for 10 seconds (for example rolling around on the floor). If assisted by another character then it only takes 5 seconds with this roleplay to end the call. If the call is ended by either of the above two actions before 15 seconds has passed, the target takes no damage.

\subsection{Grapple}

A character may make this call while placing their hands on the target in order to restrain them and/or move them. For safety reasons please roleplay this by just placing hands on the person, rather than physically moving them or restraining them; in the case of moving a target using this call they would then stand up and walk in the direction they are being guided; in the case of restraining a target they would roleplay struggling.

You can only use this call on 1 target at a time (you cannot restrain or carry two bodies at once).

\begin{itemize}
\item If the target is unresisting then this call requires 1 hand per grappling character.

\item If the target is resisting then this call requires 2 hands per grappling character.

\end{itemize}
A minimum of 3 people calling Grapple on a target is required for it to take effect. This number can be modified by the Bulging Biceps skill, reducing or increasing the number of people required (see \textit{pp. 68-69}).

\begin{itemize}
\item A character with Tier 2 Bulging Biceps counts as 2 people for the purposes of this call and also requires an additional 1 person to be grappled if they are resisting.

\item A character with Tier 3 Bulging Biceps counts as 3 people for the purposes of this call and also requires an additional 2 people to be grappled if they are resisting.

\end{itemize}
A character held with the Grapple may only be moved at a slow walking pace.

\subsection{Hallucinate}

A referee is required for this call.

This call allows the caller to cause a target to experience a hallucination of the caller's choice. The caller can tell the referee exactly what they wish the target to hallucinate, and they will relay it to the target.

The effectiveness of this call depends on the situation and what the caller wishes the target to hallucinate.

\subsection{Heal}

The target's body hits on the chosen location(s) are restored by the amount indicated by the caller.

There are three main ways that this can be done: the \textbf{Medicae skill} (see \textit{pp. 74}), a \textbf{pharmaceutical} (see \textit{pp. 116-117}) or the \textbf{Omega Body Attunement skill} (see \textit{pp. 75}). Each restores a different number of body hits to different locations.

Pharmaceuticals generally cause hits to be restored instantly, however some do not. The character who gives you the drug will tell you the exact effects. The other two methods take some time to be applied, and while they are being applied your death count continues (you may still die). See \textit{pp. 74} for the times required to use this call.

This call can also be used to bring back to consciousness, or restore limb usage of, those affected by the Subdue call (see \textit{pp. 31-32}). The location and amount of hits restored do not matter: one use of the call will restore all usage to all locations/consciousness affected by the Subdue call.

This call can be made in IC interaction, in particular when using the Medicae skill.

\textbf{Making the Heal call in IC interaction}: Captain Dana O'Malley uses her Medicae skill to heal the bullet wounds in the chest of Private Jenssen. As she roleplays pulling the bullets from the wounds and disinfecting them, then applying pressure to the area to stem the flow of blood, she talks to her patient, reassuring him as he grits his teeth and cries out in pain. ``Okay, I've removed the bullet, you should be stable now, stay with me Jenssen{\dots}''. Jenssen has now been affected by the Stabilise call. Dana then pulls out her medic's tools and gets to work stitching Jenssen up. After a while she says, ``Jenssen,

are you still with me? We're nearly there buddy. I've almost got you{\dots}'' The last few stitches are put in place. ``Okay, that's it. Healed! Try and move your arms for me buddy.'' Jenssen has now been affected by one use of the Heal call.

 

\subsection{Knockdown}

The target is knocked off of their feet and must fall to the ground for at least 3 seconds. Please do this carefully and check behind you before you fall. If you are unable to fall safely, then you must crouch and remain stationary for 3 seconds.

If you block or parry this call, you will not fall down. However, you should roleplay being knocked off- balance for a second as you absorb the force of the blow. This call can be negated in the target has Strength (see \textit{pp. 31}).

Location to zero

The target's global hits are reduced to 0. The target's armour and body hits on the indicated/struck location are reduced to 0.

\subsection{Marksman}

It is recommended that you get a referee when using this call.

The target's global hits are reduced to 0. The target's armour and body hits on their head location are reduced to 0.

\subsection{Mass}

This call may be added to other calls in order to apply them to an area, rather than just 1 target.

Any call preceded by ``Mass'' affects every person in a 15-foot radius from the source of the call, or in a 15- foot cone if indicated. For example, ``Mass Knockdown'' would affect everyone within 15 feet of the source of the call with the Knockdown call effect.

\subsection{Negate}

This call cancels the effect of an Omega power. See the \textit{Omega Protection Attunement skill, pp. 77}.

No effect

The blow or call striking the target appears to not do any damage to the target.

\subsection{Omega}

Bypasses global and armour hits.

Hits made using this call damage creatures that are normally immune to physical damage (such as spirits or greater daemons).

\subsection{Pain}

The target takes no damage but their body is wracked by intense pain. They must fall to the ground and roleplay being in intense pain for 20 seconds. They are unable to take any other actions for the duration.

\subsection{Paralyse}

The target is left unable to move, speak or take any action for 30 seconds.

The target must remain in the exact pose they were in when paralysed, even if they take damage that leaves their limbs incapacitated or their character dying. The death count continues as normal throughout the call effect. After the effect wears off they must roleplay any damage they have taken as normal.

\subsection{Possession}

The target's body is under the control of the caller. The call lasts 60 seconds.

The caller may issue simple instructions (eg, ``follow me'', ``kill your friend'', ``tell everyone to let me go'') to the target. No commands can be given that cause the target to inflict damage upon themselves (eg, ``shoot yourself'').

The caller also takes all hits suffered by the target while under possession, bypassing global hits and armour hits.

When using this call the caller must remain in sight of the target (if the method of calling has a range limit it is only required for the casting). If line of sight is broken for more than 3 seconds, the call ends.

A referee is required for this call. It is best to seek one out and tell them exactly what you want the target to do. The referee will then relay the instructions to the target.

\subsection{Reflect}

This call causes an Omega power that has been used to hit the character who called it, rather than the intended target. The caller can only use Reflect on Omega powers targeted at themselves: they cannot reflect a power aimed at another person.

Omega power calls with the prefix ``Mass'' (eg, ``Omega 15 Mass Knockdown'') may only be negated (see

\textit{pp. 29}) and may not be reflected.

\subsection{Repair}

\textbf{Repair}: Billy ``Breaker'' Lowry is an engineer. As such he is able to repair armour at a reasonably rapid rate. To aid him with this he carries around a few tools with him, a (blunt) needle, strong twine for sewing lighter armours together, and a small LARP-safe hammer for bashing heavier armours back into shape. His fellow trooper's heavy armour is damaged in a skirmish, and Breaker assists by roleplaying restoring the shape of the plates with his hammer and sewing straps with his needle and twine. After 90 seconds of this roleplay he lets his squad mate know his armour has been repaired for 1 armour hit, but he will have to wait another 180 seconds if he wants it fully repaired.

The target's armour hits on the chosen location(s) are restored by the amount indicated by the caller. Any character may repair armour at a rate of 1 armour hit per location per 180 seconds, but characters with the Engineering skill may repair at a fastr rate. Only

characters with the Engineering skill can take part in repair led by a character with the Chief Engineer class attribute. This call can also be used to repair damaged items. It requires a referee for any non-standard items. See \textit{pp. 94} for the times requiered to repair armour.

This call can be made in IC interaction.

\subsection{Repel}

This call causes the target to be pushed back 10 feet from the source of the call.

\subsection{Shatter}

If used via a ranged call the item indicated is damaged and cannot be used again until it has been repaired as per repairing 1 armour hit (see \textit{pp. 82-83}). If delivered via a physical attack the armour, melee weapon or shield that is struck is then damaged and cannot be used again until it has been repaired as per repairing 1 armour hit.

Do not attempt to hit non melee-safe items (such as guns). If you wish the Shatter call to affect a non-melee safe item, you must strike the arm that is using the item.

This call cannot be blocked and will take effect even if parried.

\subsection{Shield}

A target under the effect of this call is immune to all types of non-Omega-based physical damage for 30 seconds. If they attempt any hostile actions in this time, the call is ended immediately. When being hit while under the effects of the Shield call, the target must say ``Shield'' to inform others that they are not being affected.

Note that although physical \textit{damage} does not affect the target while they are using the Shield call, physical

\textit{effects} do: calls such as Knockdown and Repel will still affect the target.

\subsection{Sleep}

This call causes the target to fall asleep for 300 seconds. The target can only be woken in the first 60 seconds of this call by inflicting any type of damage on them. If they are not woken in the first 60 seconds, the target will continue to sleep for a further 240 seconds, but will wake up if exposed to any significant external stimuli, such as a loud noise or being shaken.

Spirit sense

This call grants one of four different abilities chosen by the caller:

\begin{itemize}
\item \textit{Spirit sight}: Grants the ability to see any invisible or ethereal creatures (such as those under the effect of the Blink call, see \textit{pp. 27}) that have spirits for 60 seconds.

\item \textit{Spirit diagnosis}: Grants the ability to sense the type of spirit and the health of the spirit of a visible target.

\item \textit{Spirit analysis}: Grants the ability to analyse a spiritual item in order to discover, approximately, its function. This requires 15 minutes of appropriate roleplay.

\item \textit{Psychometry}: Grants the ability to gain a sense of the spiritual history of an object or place. This requires 15 minutes of appropriate roleplay.

\textbf{Spirit sense: Psychometry:} Rhain turned the letter over in his hands, and started to tune out the surrounding camp. He focused his mind's eye on the item and directed his spirit into it. Around him the world stopped turning, and the mysteries of the letter began to open themselves to him. Breathing in its scent, he let it carry his spirit to an unfamiliar place, and as the vision settled in his mind he began to chant. The vision grew clearer, accompanied by sounds and smells, and as the person - human? - came into view the feelings washed over him. Jealousy{\dots} desperation{\dots} then a need for blood. Finally, as he breathed out the last words of the chant, the place became clear: tents, camo netting, a med bay. He opened his eyes and spoke to the expectant Colonel. ``I've found him. Let's roll.''

\end{itemize}

\subsection{Stabilise}

Stabilisation may be performed more than once on a character during the same instance of dying, and each time it is used it resets the death count to zero and extends it to 240 seconds. Note that if stabilisation is used on a character whose death count is already extended beyond 240 seconds (for instance, by another item), the stabilisation sets the death count to 240 seconds regardless.

This effect lasts until the character is no longer in a dying state.

There are three main ways to be stabilised: the Medicae skill (see \textit{pp. 74}), a dermograft patch (see \textit{pp. 87}) or the Omega Body Attunement skill (see \textit{pp. 75}). When a dermograft patch is used stabilisation is instant. The other two methods take some time to be applied, and while they are being applied your death count continues (you may still die).

Your death count does not reset and change until the healer has completed using the skill and made the Stabilise call. See \textit{Chapter 7: Dying, Healing and Stabilisation} for further details.

\subsection{Strength}

The target is pushed backwards at least 3 feet, knocked down and must fall to the ground for at least 3 seconds. If you are affected by this call, please do this carefully and check behind you before you fall. If you are unable to fall safely, then you must crouch and remain stationary for 3 seconds.

This call cannot be blocked or parried, but it may be ignored: if you are hit by this call and are also under the effects of something that grants you the ability to use the Strength call, you do not take the effect.

\begin{itemize}
\item For the purposes of an item/power that grants you this ability passively (ie, without needing to expend limited uses to use the Strength call) you may negate all calls of Strength used upon you. If you do this, you must say ``Strength'' as you take the blow.

\item For the purposes of an item granting you limited uses of the Strength call (eg, an axe that gives you three calls of Strength per day), you must expend one of the uses if you wish to resist this call.

\end{itemize}
If you use either of these means to ignore the Strength call, please roleplay being knocked off-balance for a second as you absorb the force of the blow.

\subsection{Stun}

The target is stunned/dazed and unable to move or attack for 5 seconds. The target is still able to defend themselves, although not very well.

\subsection{Subdue}

Anyone can use this call and it can only be made with melee weapons.

This call does not cause any lethal damage to the target's hits, and allows the roleplaying of knocking somebody out or incapacitating a hit location. Each blow struck that is intended as non-lethal damage must have ``Subdue'' called with it. If ``Subdue'' is not called, then the blow is considered lethal damage as per normal damage rules.

In order to affect a target with this call they must be struck as many times as their current combined hits on that location. These strikes do not have to come from the same source. However, if the target does not

suffer a Subdue strike within 30 seconds of the last, the strike count resets.

\begin{itemize}
\item If used on a limb location (left arm, right arm, left leg or right leg), then that limb cannot be used for 120 seconds.

\item If used on the head or torso location, then the target will be rendered unconscious for 120 seconds. Limb usage or consciousness can only be restored before the end of the 120-second count by an action that would restore at least 1 hit, resulting in the Heal call. The location of the hit restored does not matter (please see \textit{Chapter 7: Dying, Healing and Stabilisation}).

\textbf{Subdue}: Andrew is a scout wearing medium armour on all locations and has the Dodge skill at Tier 1. He is scouting an enemy encampment when he is ambushed by a patrol. The patrol wish to capture him alive so they can ransom him back to his regiment. They attack Andrew with various melee weapons while calling ``Subdue''. The first strike lands on Andrew's right leg and is absorbed by the global hit granted to him by the skill dodge. The enemy land four more Subdue strikes on the same leg, reducing his armour and body hits on this location to 0. As a result Andrew cannot use that leg for 120 seconds. Since he is unable to move, the patrol closes in around Andrew and strike him with Subdue in the torso four times, reducing his armour and body hits on this location to 0. As a result Andrew falls unconscious.

\end{itemize}

\subsection{Terror}

The target must run directly away from the source of the call for 30 seconds. If the target is unable to run away (because they are trapped/restrained IC or unable to for OC reasons), they must cower and may only defend themselves.

This call usually cannot be resisted with the Courage call, however the highest tier of the skill does allow limited resistance per day (see \textit{pp. 69}).

\subsection{Through}

Hits made using this call bypass global and armour hits, and damage body hits directly.
