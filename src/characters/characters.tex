\part{Character Creation}

\chapter{Step-by-Step Guide to Character Creation}

\subsubsection{Step 1: Pick your regiment.}

All players in \textit{Green Cloaks} are members of one of four regiments. The regiment you choose - unless you are non-human or an Adept - grants you a free regimental specialist skill at Tier 1. This skill counts as a primary skill for the purposes of spending skill points. Please see \textit{Chapter 10: The Regiments} for details on each of the regiments.

Please note that you must join one of these regiments: you cannot have a character who is not a member of a regiment (see \textit{pp. 36} for further information on semi-independent individuals and organisations attached to factions).

\subsubsection{Step 2: Pick your race and class.}

There are five playable races in \textit{Green Cloaks}, most of which have multiple classes available to them. There are many other races in the \textit{Green Cloaks} universe, however only the ones listed in this rulebook are permitted in the Terran Sovereignty Army and therefore playable as a character.

The class that you choose gives you a free starting skill at Tier 1, as well as various abilities unique to that particular class. This skill counts as a primary skill for the purposes of spending skill points. Please see \textit{Chapter 11: Playable Races of the Known World} for details of races and \textit{Chapter 12: Classes} for a details of the classes.

\subsubsection{Step 3: Customise your character's skills by spending your starting skill points.}

Each character starts with a certain number of skill points that are listed in the \textit{Chapter 12: Classes}. Starting characters must spend all of their skill points, and they may not be saved for later use. If a character survives a main event, they earn 2 skill points. Skill points earned by survival of an event may be saved for later use. You must have already purchased the previous tier of a skill in order to purchase the next tier of that skill.

Each class has two sets of skills: \textbf{primary} and \textbf{secondary}. These are listed for each class in \textit{Chapter 12:} \textit{Classes}. Primary skills are a class's main specialisation and it is cheapest to purchase and advance skills in this set.

The skill point costs for \textbf{primary} skills are as follows:

\begin{table}[H]
\begin{tabular}{|l|l|l|} \hline 
\multicolumn{3}{|l|}{Primary skill costs} \\
 \hline Tier 1 & Tier 2 & Tier 3 \\
 \hline 1 & 2 & 3 \\
 \hline \end{tabular}

\end{table}

Secondary skills are a class's sub-specialisation and are more expensive than primary skills to purchase and advance.

The skill point costs for \textbf{secondary} skills are as follows:

\begin{table}[H]
\begin{tabular}{|l|l|l|} \hline 
\multicolumn{3}{|l|}{Secondary skill costs} \\
 \hline Tier 1 & Tier 2 & Tier 3 \\
 \hline 2 & 3 & 4 \\
 \hline \end{tabular}

\end{table}

You may purchase or advance skills from outside your class's primary and secondary skill sets. Doing so is known as taking a \textbf{veteran pick}.

The skill point costs for veteran picks are as follows:

\begin{table}[H]
\begin{tabular}{|l|l|l|l|} \hline 
 & Tier 1 & Tier 2 & Tier 3 \\
 \hline 1st veteran pick & 2 & -- & -- \\
 \hline 2nd veteran pick & 3 & 3 & -- \\
 \hline 3rd veteran pick & 4 & 4 & 4 \\
 \hline 4th veteran pick & 5 & 5 & 5 \\
 \hline 5th veteran pick & 6 & 6 & 6 \\
 \hline 6th veteran pick & 7 & 7 & 7 \\
 \hline 7th veteran pick & 8 & 8 & 8 \\
 \hline 8th veteran pick & 9 & 9 & 9 \\
 \hline 9th veteran pick & 10 & 10 & 10 \\
 \hline 10th veteran pick & 11 & 11 & 11 \\
 \hline \end{tabular}

\end{table}

To use this table, simply locate in the left hand column the number of veteran picks this new skill purchase would give you, and then move along to the column with the tier of skill you wish to purchase.

Bob is playing a trooper and has previously purchased Tier 1 Engineering and Tier 1 Pharmacology. He now wishes to purchase Tier 2 Pharmacology. Since Engineering and Pharmacology are not in his primary or secondary skill set, he has so far purchased two veteran picks and this will be is third. This would cost him 4 skill points.

\subsubsection{Step 4: Character background}

Consider who your character is and where they come from. Some possible questions to consider when getting ideas for your character's background could be (but are not limited to):

\begin{itemize}
\item What motivates them?

\item How have they ended up in their regiment?

\item Do they have family back at home?

\item How old are they?

\item How do they speak? Do they have an accent or a dialect?

\item What are they afraid of?

\item What do they excel at?

\item What is most important to them?

\end{itemize}
If your character is a non-human race their perspective of the universe may be different to that of humans, and they will have grown up in a very different culture. Take this into consideration when thinking about how your character acts and interacts with other characters. If you wish to have an advanced backstory, or a backstory that connects with the regiment you have chosen, please discuss this with your regiment referee(s). Please see \textit{Chapter 11: Playable Races of the Known World} for details on the non-human races and human planetary cultures.

\subsubsection{Step 5: Your costume and equipment}

Your character may start with armour of up to the maximum type their class can wear, and any weapons or shield they have the skill to use. Please physrep these appropriately. The rest of your costume - including physreps for your skills (eg, medical items, engineering equipment) - expresses your character's personality, skill set and role on the battlefield, and may also reflect your character's race, culture and home planet. See \textit{Chapter 11: Playable Races of the Known World} for inspiration.

\subsubsection{Step 6: Name your character}

A name says a lot about a person, so choose wisely! Your name is likely to reflect your race, culture and home planet. See \textit{Chapter 11: Playable Races of the Known World} for examples of names used by each race. Please note that names should not be offensive, and should make sense within the \textit{Green Cloaks} universe - Captain James Kirk of the 23rd Heavy Infantry, or Pvt. First Class Ben Cumberbatch of 98th Kingskeep may seem hilarious at first, but can break immersion for many people.

\chapter{The Regiments}

With the exception of adepts, humans receive their classes through training in their regiments. During this training they are given roles to play within the military structure, and are given extra training in their

regimental specialist skill. This skill represents the role the regiment usually plays in the Terran Sovereignty Army. Non-humans do not receive training in their regiment's specialist skill, as they do not receive military training in the Terran Sovereignty Army, but instead receive a skill through their upbringing or culture.

Please note you \textbf{must} join one of these regiments. You cannot have a character who is not a member of a regiment.

The regimental specialist skills are as follows:

\begin{table}[H]
\begin{tabular}{|l|l|} \hline 
Regiment & Regimental Specialisation \\
 \hline 109th Light Infantry & Melee \\
 \hline 23rd Heavy Infantry & Heavy Weapons \\
 \hline Delmont 205th & Self-sufficient \\
 \hline Kingskeep 98th & Stealth \\
 \hline \end{tabular}

\end{table}

The regimental specialist skill is granted at character creation to all non-Adept human characters at Tier 1 for free, and thereafter counts as a primary skill for purposes of spending skill points (see \textit{Chapter 13:} \textit{Skills}).

\subsection{Ranks}

The starting, and lowest, rank in the Green Cloaks regiments is recruit. Rank in \textit{Green Cloaks} is purely a roleplay mechanic. It does not grant any out-of-character benefits (you do not gain skills, money or equipment for increases in rank). Rank in one regiment does not give you authority over lower ranks in another regiment. Each regimental command structure is independent of the others (in other words it is ``legal'' in character - though not always advisable - to ignore another regiment's commander).

Regardless of your backstory all new recruits to the regiments present at the events start at the rank of recruit. It is possible (after discussion with your command team) to have a higher rank in your background. However, due to the chaos and changes around the One Bakkar war, organisation has been set back to basics and your rank will be reset when you reach the front lines. No authority of any kind that may have existed in your backstory is carried over to in-game play.

There are also a number of semi-independent organisations in the \textit{Green Cloaks} universe: these range from mercenary gangs and military police to private military contractors. All independent contractors (such as civilians and mercenaries) that work with the regiments of the Green Cloaks must be attached to a specific regiment and are subject to the command structure of that regiment, as a normal recruit would be. This means that they are subject to the same military codes and laws that apply to recruits. These groups may also have internal rank, however it does not carry over to military rank.

\subsubsection{Ranks in-game}

The two command team ranks are:

\begin{itemize}
\item Colonel, the head of the regiment

\item Captain, the second-in-command of the regiment.

\end{itemize}
Due to the chaos of the One Bakkar war, military organisation at a lower level has become somewhat less formal. Stretched supplies and communication that is patchy at best between high command and troops has necessitated the regiments being given much more authority over their internal structure.

A regimental rank structure below the rank of Captain is decided upon by the regiment's command team. Each regiment has a different style of command and its structure reflects this. For more information on a certain regiment's rank structure, please contact its command team. However, it is recommended that you

discover this IC, as you are unlikely to know the workings of the regiment as a new recruit. Please note the two command team members are decided upon by the game team and former regiment command. This does not mean you are unable to rise to this rank, however it does carry a large OC responsibility and time commitment.

\subsection{109th Light Infantry}

\textit{``Everybody comes back alive.''}

The 109th regiment hails mainly from the planet Durgan, although it accepts members from throughout the Terran Sovereignty and its allies. Its primary role in the field is as a melee-based light infantry force. There are several specialised units which make up the 109th which, when combined, create a stalwart but fluid force to be reckoned with.

Durgan is the home to the Temple of the Sword. This order's temples teach swordsmanship, and those who study the arts of the blade learn quickly that in the universe of guns and killing machines the sword

never runs out of ammunition. The Temple also embodies the Virtues - charity, faith, honour, justice, loyalty, mercy, truth and valour - that define the spirit of the 109th. Most people choose one particular Virtue to embody, but a rare few study them all. Each of the units have their mottos to live by, but throughout the 109th can be heard the shouts of ``Everybody comes back alive!'', a bold statement of the commitment of the members of this regiment to one other, and their joint belief in the Virtues.

Although they are fearsome on the battlefield, the 109th know how to relax in style during times of rest. They generally have the largest and best-stocked camp. It is welcoming to all who wish to be their friends and is often filled with people from all of the regiments, and others from strange and mysterious places.

Regimental specialist skill: Melee

\subsection{23rd Heavy Infantry}

\textit{``The components of victory are discipline and honour.''}

Founded on Marazion V, within the majestic underground cities carved into the bowels of mountains, among caverns filled with ores and minerals waiting to be mined and processed in their gun factories, the 23rd Heavy Infantry serve the noble families of the planet and protect its people. This mountain-

dwelling regiment is drilled to tactically hold any defensible position, and to watch the backs of their fellow soldiers for any unknown dangers. They can often be seen trudging up to the front lines with crate-loads of ammunition to feed their beastly heavy weapons.

Amongst the Green Cloaks, the 23rd Heavy Infantry are sometimes seen to be like the tortoise edging across the battlefield, slowly and steadily advancing, ready to take up defensive positions. But the proud men and women of the 23rd imagine themselves to be like the terrifying Gigamesh beast that once plagued their home world, with their armoured hides and their heavy weapons roaring a hail of fire. Stalwart in defence and unyielding in attack, the 23rd are true gunnery experts who will bring down the thunder on their enemies. From their Engineers crafting bigger and better guns, to the bullet-spitting Heavy Weapons Specialist, the 23rd Heavy Infantry embodies the indomitable fire power of the Green Cloaks army.

The 23rd Heavy Infantry support one another as a group, covering each other's backs in the field, never leaving a man behind and combining their skills to get the best results. Teamwork is a matter of honour between these soldiers, which will bring them true victory. But there are no heroes in the 23rd, for they believe that life is not worth throwing away to be a hero. The names of those lost on the battlefield are never forgotten, and the 23rd are often seen standing around their blazing fires in the dark of night, telling tales of the brave, fallen warriors. In the 23rd Heavy Infantry you shall never forget, and you shall never be forgotten.

\textbf{Regimental specialist skill:} Heavy Weapons

\subsection{Delmont 205th}

\textit{``Live long and die well.''}

Delmont 205th are a diverse lot, most hailing from the cultural pit that is the planet-city of Delmont. Specialising in battlefield support, logistics and trade, Delmont pride themselves on their craft and handiwork. Whether they are working with metals or herbs, interesting and sometimes morally questionable items have been created using their skills.

Made up of a ragtag crew of humans and aliens from all walks of life, how this unit manages to function together as well as they do is a mystery. They have gathered from the many stacked levels of Delmont's rambling mega city. Some wanted a better life, others needed to escape, and some came from taking the King's Mark. While they may look like a rough bunch to be with you can always guarantee you'll be welcomed to witness for yourself just how warm Delmont can be. Anything you need they can more than likely get: for a fair price of course.

Despite their eclectic nature, they do have one thing in common: they are all headstrong and use their skills

in a fight to make sure things go their way, even more so when there is money on the table.

Delmont are more than happy to be contracted by other regiments to assist with a variety of roles. Whether it is creating pharmaceuticals to get all the troops up to full strength, or standing before the enemy and forming a human sword wall, Delmont are always ready to do what is best for the Terran Sovereignty and best for business.

\textbf{Regimental specialist skill:} Self-sufficient

\subsection{Kingskeep 98th}

\textit{``Don't fight like heroes: fight like bastards.''}

Kingskeep 98th are a rapid response, special reconnaissance regiment comprised of small units of highly trained military personnel, including special forces, psy-warfare and military intelligence. They generally operate behind enemy lines, utilising guerilla tactics and unconventional warfare.

Kingskeep 98th was originally named for a citadel located deep in the dark forests of the planet Ardheim, which proved to be one of the most challenging and dangerous places to train. Accepting recruits from all races and planets of the Terran Sovereignty, it produced the best Scouts and Snipers ever seen. Most recruits are required to attend basic training at the Keep within the forest (30\% pass rate), however some promising recruits are sent straight to the front lines for training in live combat.

Due to the close relationship and well-established trade routes between Ardheim and its neighbouring colder, harsher planet Rossi, Kingskeep's regiments are usually an even mix of Ardheimians and Rossii. The practicality and loyalty of Ardheim has blended with the tenacity and implacability of Rossi to create a regiment operating with highly specialist tactics.

In alliance with the other regiments, Kingskeep 98th is called upon for reconnaissance missions, scouting behind enemy lines, infiltration and exfiltration, ambushes and fire support; it is known for its hit-and-run tactics, stealth, precise marksmen and speed.

There are four simple rules in Kingskeep 98th:

\begin{enumerate}[I]

\item Always carry your dagger.

\item One shot: one kill.

\item Maintain constant vigilance.

\item Don't fight like heroes: fight like bastards.

\end{enumerate}

Regimental specialist skill: Stealth

\chapter{Playable Races of the Known World}

The race that you choose to play dictates the classes that are available to you. Humans have the most diverse range of classes available, and these classes are gained through regimental training in the Terran Sovereignty Army. Classes of the non-human races are tied to the race's culture, and represent roles

that have chosen to work with the Terran Sovereignty Army. There are many different races in the \textit{Green Cloaks} system, however only the ones listed in this rulebook are permitted in the Terran Sovereignty Army and therefore playable as a character. For the non-human races, certain physical representations arerequired to distinguish them from humans. Suggestions for these are given in the relevant race description.

\subsection{Humans}

Humans are a widely varied species that has spread out across the galaxy and developed distinctive cultures on different planets, as well as sub-cultures and communities. As a species they are the quintessential jack-of-all-trades, and they can be found in all sorts of professions. In their discoveries of the planets and space beyond their homeworld of Terra, they have presented an imposing and engaging force throughout the galaxy, eager to make their mark, explore new territory, share scientific advancements and technology, and adopt new languages and cultures.

Wherever they populate, humans create thriving and rich cultures, and they are often very nationalistic. Due to the varied nature of this species, human attitudes and actions vary widely, accents may differ, and clothing will not only be worn for practicality and comfort but also for fashion and affiliation. Since

forming alliances with non-human races, humans have adopted some of the fashions of those races, often depending on the proximity of their home planet to that of the non-human race.

Humans are also versatile, and often adopt the approach and attitude of the species they interact with the most: a human who spends a lot of time working with mascens is likely to become more brusque, blunt, and bold, and perhaps add tribal adornments to their clothing; a human in close diplomatic relations with myr'na may speak more gently than their human friends, and cultivate a calmer, more peaceful attitude to life.

A typical human lifespan is around 120-150 years, however this is significantly affected by the planet they live on. Humans commonly die at a younger age than this, as their species discovers new diseases that they have not built a tolerance for, and as they are perhaps the most risk-taking, inquisitive and explorative species, engaged in the bloodiest warfare against the One Bakkar.

Humans have colonised many planets over time, and those that have withstood the challenges of colonisation, and the threats of the One Bakkar expansion, have developed a unique culture, as well as training and fielding their own military force.

\subsection{Human Home Planets}

\subsubsection{Ardheim}

\textit{Pronunciation: AARD-hime}

\textit{Name of the people: Ardheimian (sing.) / Ardheimians (pl.)}

A planet of two very different sides, Ardheim boasts some of the most dense and richly forested areas of the galaxy around its equator, and uninhabitable frozen tundra at the poles. There is no middle ground and the one gives way to the other in stark contrast. It was colonised in 4865, after years of testing and sample- taking to ensure habitability, by the scientific team and pioneers led by Professor Matthias Blake, who landed in dropships filled with all they would need to not only survive but also build the first structures deep in the forests. As time went on they found that one genus of tree native to Ardheim had particularly hardy wood that would absorb the strikes of axes or saws and blunt their blades quickly, and once they were

able to dropship more capable felling machinery to the planet they began using these trees to build their structures. They named this tree J\'{a}rnr\'{o}t, ``\textit{Ironroot}''.

Today, Ardheim is famed for its cities built within the forest, and it has strong alliances and trade agreements with its neighbouring smaller, colder planet Rossi. Its pioneering history has created a culture

of people who place a focus on survival, practicality and loyalty. Due to its back-to-basics beginnings, the earliest Ardheimians could not rely on firepower, so instead trained with ``old-fashioned'' military tactics: shields and melee weapons. Although many of them excel at ranged combat, producing some of the finest snipers of the Terran Sovereignty Army, a significant number of Ardheimians excel at shieldwall tactics, skirmishing in heavily wooded terrain, and guerilla tactics. The most famous city of Ardheim, and one of its oldest, is Kingskeep, which gives its name to several regiments that train there. Situated on the border of untamed and uncultivated deep forest, this city is home to the eponymous Keep, which serves as a military training ground for specialist units. The nearby forests will either make a person or break them, and the specialist regiments do not accept the broken.

Ardheim is a storytelling nation, steeped in a rich folklore that has developed around the discovery and habitation of the planet. Certain aspects of the planet, such as the fact that it is one of nine in that system (of which only Ardheim and Rossi are habitable), caused the pioneers to give names to things, projects and places inspired by old Norse, Icelandic and Scandinavian myth. Over time, this folklore developed into spiritual ideas and concepts of the universe, and the Ardheimian people today often speak of the great tree Yggdrasil that forms the axis mundi of their planet (though nobody has yet found the tree), though their ideas of Gods seem to reflect ideals to aspire to rather than belief in existence of them. Nevertheless, it is not unusual to find scientists who call upon Odin for his wisdom, military men and women who shout the name of Thor into the face of their enemy, or a man imagining the power of Skadi flowing through him as he takes aim at what he hopes will be his family's next meal. Today, the founders of the planet are seen as

embodying the features of these gods - such as Matthias Blake as the wise Odin - so when an Ardheimian calls on these mythic figures they are as much calling upon the virtues of the historic people who founded their nation as they are divine forces.

Perhaps unsurprisingly for a pioneer nation, Ardheimians are a warm, welcoming, and life-affirming people. They are proud that they have cultivated this overwhelming planet, and they have many stories to tell and reasons to celebrate. In times of relaxation they often get together with others and share food and mead, entertaining each other with stories, song, competitions and toasts. With those they consider their friends they are extremely loyal, and they prize this loyalty amongst those who fight together in units and regiments also. To fight alongside a fellow shield-bearer makes you brother or sister to them, in a bond that is sealed in the blood spilled in battle.

Arheimians in the military utilise the same armour and weaponry as others, issued to them as standardequipment. However, they are proud of their homeworld, and often add furs and wrapped leather to their armour or clothing as a mark of where they come from. They favour camoflage gear over kit that stands out, but many of the heavier fighters prefer chainmail over other heavy armour; to reduce the metallic shine from this they sometimes blacken it. Those who fight with shields prefer roundshields painted with geometric or animal designs. Nearly all Ardheimians, whether civilian or military, wear practical clothes,

eschewing flowing garments or gaudy fashions. Ardheimians also prefer to mend or make something out of old items, rather than buying something new, so sometimes their gear may look a little patchy, or might not match. But if it does the job, who really cares?

\textbf{Inspiration for characterisation:} Vikings, Iceland, Scandinavia, oral tradition, \textit{Beowulf}, \textit{D\&D} Rangers, bushcraft, Viet Cong, Saxons, survivalists.

\subsubsection{Cantiacorum}

\textit{Pronunciation: can-tee-uh-COOR-um}

\textit{Name of the people: Cantiarchii (sing. and pl.) (can-tee-AAR-kee)}

Cantiacorum is the capital of the Segovax cluster and the political hub for the entire region. Discovered and colonised by Lord Segovax in 4805, very little work in terms of terraforming was required to make it suitable for human life. It is from Cantiacorum that the rest of the cluster was colonised and where those who can afford to travel further into the cluster past Delmont usually end up. It is similar to Terra in terms of culture, but about 40 or so years behind given the distance between the two. The two planets share a very similar orbital cycle and the length or their days is almost the same. This allows the two planets to operate effectively together.

41

An elected parliament runs the day-to-day affairs of the planet with the prime minister in overall charge and answerable only to the king of Terra. It also holds the headquarters of the regional military and is the epicentre of the Adept Program. As such, Adepts in the cluster are usually trained there before being sent to serve with their respective regiments.

Being the main hub for pretty much everything in the Segovax cluster, Cantiacorum has the best of all that can be offered. Its soaring architecture is similar to that of Durgan, but made of marble and silver from Marazion V. It has the cultural diversity of Delmont, but without the rampant crime. The homes and offices of the wealthy are decorated with carvings and furniture made from the unparalleled Ironroot trees of Ardheim.

When Cantiacorum was colonised Delmont had already been largely stripped bare of resources and had begun its slow descent into toxic wasteland. To avoid the same fate for Cantiacorum, many areas have been left untouched. This means the surface of Cantiacorum is broken up among large sweeping cities and verdant areas of forest and meadow.

Humans who live on this planet often dress in fitted clothing with long sleeves. Intricate embroidered materials are common, as well as fine jewellery and elaborate hairstyles. Since most Cantiarchii work in administrative, political, academic or diplomatic roles, they do not feel the need for their clothing to be practical, instead viewing it as a way to demonstrate status or showcase personal taste. Cantiarchii are often proud people who value interpersonal relationships highly, and often use friendships and marriages as a way to cement diplomatic alliances. Those who seek employment on other colonies are often sent as delegates, diplomats, teachers and record-keepers; many find themselves in the offices of high-ranking military personnel, drafting despatches, advising on political matters, and keeping records of engagements.

So far, Cantiacorum has been untouched by the One Bakkar. This is partly due to the sheer amount of military strength at its disposal and partly due to the determination of the current front line able to keep the enemy at bay.

\textbf{Inspiration for characterisation:} High culture, Vatican city, Rockefeller Center, United Nations, Coruscant (\textit{Star Wars}).

\subsubsection{Delmont}

\textit{Pronunciation: DEL-mont}

\textit{Name of the people: Delmonte (DEL-mon-tee) (sing.) / Delmontes (pl.)}

When the Segovax cluster was first discovered in 4800, and in the first stages of terraforming on the outer fringes of Terran space, the planet of Delmont suffered as pioneers flooded from other planets to see what futures they could make for themselves there. The planet was used as a staging area to explore further into the cluster, but before any true structure or order could be implemented, people were colonising the juvenile planet and stripping it of its rich natural resources. Before long there was nothing left to be scraped from the planet's bare surface and the beginnings of a rambling city started to form as people searching for their new lives found that they could not afford to travel beyond Delmont into the rest of the newly discovered planets, causing population to soar from ill-managed immigration.

Now Delmont is infamous for its harsh acidic landscape, but more so for the sprawling stacks of city sectors piled on top of each other, creating a hazardous metropolis of hidden slums, chaotic trading points and urban mazes. The gigantic city of Delmont has spread upwards as well as outwards, with the very first settlers' homes lost away in the deep, dank undercaverns, polluted by radiation and uncharted on any map.

Anyone of any importance who visits the planet is unlikely to see beyond the first few layers - they are the most recently built, rich, and well-organised of the planet, modern and utopian. On the top of the artificial surface is a beautiful, versatile landscape, well-terraformed and actively farmed to produce food for the wealthy few Delmontes who live near the surface. Immediately below are the official trade points, homes of the rich, and offices of the highest city officials who pretend to care. The Council of Delmont is presided

over by The Castor, though history has shown that this role is usually filled by the self-serving and indulgent

who, keen to line their own pockets, ignore the needs of the poorest and most vulnerable.

42

Beneath this facade the population of Delmont is left to rot in the dark, in crumbling wards and sectors creaking under the weight of one another. The further down it goes, the poorer people get: ascending through the levels of the city is an insanely difficult task for anyone who starts off low down. People are kept out of sight, buried away, working long hours in disastrously dangerous factories and filthy living conditions. With little regulation or law enforcement, the ceilinged streets are rife with crime and gang warfare, and deep down there is word of slavery and worse. No-one truly knows how deep the sectors go, with some abandoned due to radiation, cave-ins, or similar. Many sectors remain uncharted, and there are tales of sectors being cut off from the rest of the city, left to fend for themselves.

Despite this, there is a strong sense of camaraderie among Delmontes: to find someone you can truly rely on and trust can make the difference between life or death, and finding your true ``family'', whatever the circumstances, can be vital for survival. Many regiments in the military from Delmont are known for forming a sense of belonging that goes beyond their work. They are also known to be very resourceful, knowing the most efficient ways to make do in tough times, leading to an affinity for Engineers and Medics in logistical positions - perfect for battlefield surgery and quick fixes. A Delmonte believes everything has a use whether intended or not!

Beyond the standard equipment they are issued, Delmontes are able to make the best of their surroundings. Running by the philosophy ``if it ain't broke don't fix it'', many of their possessions are worn down but still functional, or patched together - but if it serves its function, then it suits them just fine. Many members of the Delmont military are not suitably prepared for their time on the front lines, with sometimes incredibly basic training, but they are adept at solving their problems with unique solutions.

Many members of the Delmont military never saw the sky, sun, or stars before joining up and being shipped off-planet. For many the change is a huge shock: where is the ceiling? Who made the trees? What is rain for, and how is it leaking without any pipes? The idea of an environment where things exist without purpose is a strange one, and collides with their urban nature: for many of the Delmontes on lower levels, the sun

is nothing more than a myth, an idealised concept to strive towards. Millions live and die without seeing the sun. When everything in your world has a purpose, even the concept of the self becomes a thing to trade, a commodity, and your own time and actions become as valuable as the result. Many treat joining the military as a serious action - becoming the cargo you are profiting from. There is a mantra that Delmontes know and often repeat:

``Recruitment is investment. Deployment is transaction.Death is payment.

Survival is advertisement.''

\textbf{Inspiration for characterisation:} Mos Eisley (\textit{Star Wars}), \textit{Shadowrun} RPG, \textit{Necromunda}, \textit{Brave New World} by Aldous Huxley, gang culture, criminal underworld, black market trade.

\subsubsection{Durgan}

\textit{Pronunciation: DUR-gun}

\textit{Name of the people: Durganite (sing.) / Durganites (pl.)}

Durgan is a planet of small continents and archipelagos with a temperate climate and warm oceans.

It is here that men and women take up the ancient art of the sword in its various forms and are often acclaimed as masters of melee combat. Along with their martial prowess they are a studious people rich in philosophy, and tend to study one of the eight Virtues at the respective temple (loyalty, honour, valour, truth, justice, faith, charity and mercy) and thereafter strive to apply its ethos to daily life. Each type of sword is associated with one of the Virtues - the rapier for honour, the double-handed sword for valour, etc. This is not to say that in order to study a particular style of swordsmanship one must study the associated virtue, though it's seldom that one will go without the other.

Before colonisation scientists studied the planet from afar to ensure viability as a long-term habitat. As soon as scans revealed a habitable climate, atmosphere and the presence of water, it was a simple case of landing and exploring the surface, with the first landfall taking place in 4889. The initial settlement was

43

Swanpool, but this was quickly superseded by the city of Harpthorne, which would become the planet's capital. The tall, lean buildings and lighthouses that guard Harpthorne's harbour paint a picture of peace, despite the hustle and bustle of the city below.

Heading the colonisation of Durgan was John Ivanhoe, a man known for his studies of various ancient martial sword techniques. Naturally, many that followed him held similar interests, and once the colony was fully established he set up the Temple of the Sword that, in time, would become the Temple of Truth in Harpthone - now one of eight separate Temples of the Sword, each dedicated to a Virtue. He would not live to see how influential and widespread his Way of Virtue would become. Those wishing to study the

Virtues enrol in the Temples, and each year's group of graduates are the focus of the annual Festival of the Forging, a ceremony for those leaving a Temple after completing their studies. This festival lasts for a week, and marks the point at which the student becomes a master of their chosen art, forging their own sword and proving it in combat with one of their teachers. This proving is more ceremonial than a true test of mettle, and is often just an exchange of a few blows before the student is considered to be proven. This tradition was instituted by Ivanhoe, and is sometimes referred to as the Founder's Festival in his memory.

The many seas of Durgan are littered with ocean-going vessels of all types and the oceans yield a bountiful supply of all manner of fish. Many Durganites find employment as shipwrights, fishermen, farmers, and merchants who cross the oceans on trade runs. It is not only the merchants and fishermen who spend so much time on the oceans. Those undertaking a traditional Durganite pastime of sailing and rowing exercise their rights to practice on open waters for future events. These sports that train the body and endorse teamwork and unity are held in high regard and seen as important factors by the populace, and by the Temples that sponsor them. As a result, boat racing on the coasts and in the seas are celebrated events that have become frequent since the days of settlement.

The military of Durgan are easily identifiable as they often carry a sword as part of their basic kit. To allow for effective use of melee weapons in the field most regiments from Durgan take the form of light infantry units, designed for quick movement and flexibility in battle, able to switch between ranged and melee combat with ease.

\textbf{Inspiration for characterisation:} Cornwall, Yorkshire, Scottish islands, abbeys and ancient places of learning, knights, chivalry, the age of Romance, sea-faring nations.

\subsubsection{Marazion V}

\textit{Pronunciation: ma-RA-zee-un five}

\textit{Name of the people: Marazionite (sing.) / Marazionites (pl.)}

With its vast mountain ranges and its single ocean, it was only natural that civilisation on Marazion V expanded underground at an early stage in the planet's colonisation. Mining, carving and hollowing out mountains has allowed majestic tower cities such as Sangomont to rise up to touch the highest cavern ceilings, protected within the mountains from the harsh wild winds that skirt around the mountain's outer rock layer. The planet is rich in mineral resources and metal ores, which are mined, refined or smelted, and which form the basis of the planet's economy and trading exports. Some of these resources are common to other planets - iron, copper, kernite - but Marazion V has one crystal unique to its planet, which gives off a strange and calming iridescent glow. These crystals are commonly used as light sources around the cities and towns of the planet. There are a few guarded tunnels leading to the surface of the planet, which

many a Marazionite has travelled up on a calm day to take in the sweet, awe-inspiring sounds of the ocean, known as The Sound of Marazion, or watch the unparallelled twinkling light of the surrounding asteroid belt in the night sky.

Marazion V is governed by a group known as the Families, the living family clans of those courageous pioneers who helped found Marazion V in 4859. The Families number eight clans: Castildon; Drason; Grant; Morey; Rodrick; Silhorn; Scotts and Von Vortex, between them owning almost everything on Marazion V, including the factories and the underground agricultural centres. To live on Marazion V, one must belong to or work for these rich families, and the most well-known is the Von Vortex clan. To serve under one these families comes with great rewards, as some will use their wealth to improve the wellbeing of their workers, paying them a steady wage, offering healthcare and even building more homes closer to

44

the factories and mines. Clans such as the Silhorns will select the brightest and the most talented minds to enter the engineering schemes and enrol in the engineering academies.

Out of all the planets known in the Segovax Cluster, Marazion V is considered the most advanced in industry. Under the surface of this harsh planet there are thousands of smelters, factories and workshops, fuelled by an almost constant stream of raw materials from the planet itself and most of the major worlds in the cluster. Without this industrial powerhouse supplying them, it is likely the other colonies in the cluster would have perished long ago.

Given this key status, Marazion V is also the one of the most well-protected planets in the cluster. Surrounding its upper atmosphere is an asteroid belt formed from the remains of a moon lost a millennia ago, and a series of stations, satellites and magnetic minefields are in place to help defend the planet. This is why Marazion V is sometimes nicknamed ``the planetary castle''.

Beneath its surface, engineers are continuously researching and building new equipment to improve mining and farming production, redesigning factories to mass-produce weapons, jewellery and everyday objects, and redeveloping buildings to stand the test of time. To a Marazionite, engineering is not only a way to improve the quality of work, but to also defend the quality of life itself.

\textbf{Inspiration for characterisation:} Dwarf cities, Scottish clans, steel foundries, industry, strong family loyalty.

\subsubsection{Rossi}

\textit{Pronunciation: ROSS-ee}

\textit{Name of people: Rossii (sing. and pl.)}

Rossi is the smaller, colder, harsher neighbour to Ardheim, being colonised much later than the forest planet. Its population is far smaller, but even more tenacious: the environment on Rossi is bitterly cold in the winter, and cold in the summer (though Rossii just call it ``fresh''). Its economy, however, thrives, as it has become an industrialised planet swiftly, with factories working around the clock. Most of these factories produce weaponry and munitions which supply not only the Rossi Defence Force but also the Defence Forces of other planets, but in particular Ardheim, with whom Rossi has strong trade routes. Ardheim mostly exports its famed timber to Rossi, and Rossi trades it for the products from its factories, and the vodka that it is famous for.

Rossi is a largely socialist planet, with the means of production owned communally, and an absence of social or economic class. The great machines and factories of the planet are intended to ease the work of all and not to enable a few to grow rich at the expense of millions of people. This has bred into the people of Rossi a hardworking culture, where everybody pitches in and understands that they're all in it together. To a Rossii, laziness or the shirking of a duty is perhaps one of the greatest sins, though not as great as that of ignoring or being rude to one's grandmother. Rossii elders are given a lot of respect, but particularly grandmothers, because they are seen to embody the virtues of Grandmother Rossi herself - the affectionate way that Rossii people viewing and speak of their motherland. Grandmother Rossi may be a harsh and old mother, challenging her children and grandchildren, her body - like the environment - largely barren, but she will protect them with everything she has to the bitter end, and give hell to those who would harm them. Despite this, however, many Rossii choose to emigrate to Ardheim when they come of age,

as the neighbouring planet is alluring with its promise of more warmth, fewer bears, jobs besides factory work and farms, and the sites that even Rossii tell stories of. As such, the regiments trained in Kingskeep on Ardheim are often made up of an even mix of Ardheimians and Rossii, and the two get on well together. Both have a strong tradition of storytelling and drinking (though arguments about mead or vodka often arise), and both come from a background of need for survival and practicality.

Rossii clothing is largely unadorned and unassuming, and a Rossii will usually not choose to wear an item of clothing or accessory that doesn't have practical use. This is a direct result of the socialist ideology of the planet: objects are valued by their use-value, as opposed to the accumulation of capital, wealth or status, and beauty is a luxury that many Rossii do not see as having much use-value. Rossii, due to the nature

of their planet, often dress in very warm clothing, ensuring that their heads are covered. Even when on a

warmer planet, a Rossii may choose to wear a lighter hat for the security of feeling it provides. Rossii are also rarely seen without some sort of drink to take the edge off things: even a Rossii that chooses not to partake of alcohol will keep water in a hipflask to keep up appearances, and some may even put on a show of being slightly drunk at all times. Many military units that have Rossii recruits find it useful to ensure that they have an accessible, yet limited, store of vodka available, as Rossii seem to function more efficiently with it. Of course, this means that Rossi has the opportunity to export vast amounts of vodka to the military stationed all across the galaxy{\dots}

Rossii people, perhaps because their environment is so harsh and bleak, have developed an industrious nature, but also one streaked through with dark humour. To others, it may seem like they are resigned to their cold, bitter world, but this is not the case: even though they may joke about the cruelty of Rossi or the ferocity of the bears, the bone-breaking cold or the back-breaking factories, Rossii will never fall out of love with any of those things. They made them who they are: resilient, tenacious, stubborn, and able to take anything while smiling darkly in its face.

\textbf{Inspiration for characterisation:} Russia, communism, socialism, Cossacks, Siberia.

\subsubsection{Terra}

\textit{Pronunciation: TEH-ruh}

\textit{Name of the people: Terran (sing.) / Terrans (pl.)}

Terra, called Earth long ago by humans before they became a star-faring race, is the centre of human civilisation and the the birthplace of humanity. It is the focal point of a great number of things that define humanity: politics, trade, commerce, military strength, science, culture and much more. Since taking to the stars humanity hasn't walked an easy road, but through as much luck as skill Terra has remained untouched by war. It is the headquarters of the Terran Sovereignty Army, the Terran Sovereignty Navy, Terran Special Operations, and has a branch of the enigmatic Adept Program.

Life on Terra for civilians is, for the most part, comfortable. As the wealthiest planet in human-occupied space many people have more than enough to get by. Many citizens of Terra will dress in the latest fashions, whatever that may be at the time, and all types of consumer goods and luxuries are widely available. Constant development of Terra has led to overpopulation issues, which in turn has caused the construction of millions of square miles of geocentrically orbiting habitats (habs) connected by space elevators. The quality of these habs is directly proportional to the cost of living there, meaning many of the poorer citizens of Terra are forced to live in the more cramped and less pleasant habs.

There are few native Terrans in the Segovax cluster, as the long distance between Terra and any of the Segovax cluster colonies requires travellers to enter cryo-sleep for 80 years in order to reach their

destination. Yet a few desperate souls are willing to forsake everything they knew for a chance at a new adventurous life on the fringes of human civilisation. People from Terra are usually proud and patriotic. They have not only seen but been a part of the best that humanity has to offer. This means, however, that they tend to have an unrealistic view of the rest of the Sovereignty. Some of the more conspirational individuals doubt the war exists at all, and believe it to be a propaganda tool of the Sovereignty. Few, however, agree with this view.

The reach of Terra in the Sovereignty is long, but out here on the very edge of human endeavor it is somewhat diluted. Most know that they will never see their leaders or even Terra itself, for to take the journey is to lose everyone you have known. However all acknowledge that it is the heart of humanity in the cosmos and must be protected at all costs.

\textbf{Inspiration for characterisation}: city-planets, Rome, high-culture, Capital (\textit{Hunger Games}), Xandar (Guardians of the Galaxy).

\subsubsection{Tetrarch}

\textit{Pronunciation: TET-raark}

\textit{Name of the people: Tetrarchii (sing. and pl.) (tet-RAARK-ee)}

A desert planet between Marazion V and Zennor, this planet is sparsely populated and mainly serves as a refuelling point between the two planets, as well as supplying high-quality silicon for use in electronics. The living on Tetrarch is hard, dry and hot, and so the Tetrarchii have learned to make the most of what they have: scrap, and lots of it. When it was first discovered in 4868 Tetrarch's natural resource - silica-rich sand

was largely ignored; instead it was used as a junkyard, with surrounding colonies dumping what they didn't need or want on the planet and in its orbit. A couple of colonies made quite a tidy living out of scrap- running to and from Tetrarch, and gradually people began to inhabit it and sell the scrap on to those who would buy it. It took barely a decade for investors to see the golden opportunity in the sand beneath their feet, as the neighbouring colonies - particularly Marazion V - needed a nearby source of silicon, and the mines and factories were quickly built. However, the colony's infrastructure is still in development, and as such beyond the work in mines and factories inhabitants have few resources and no luxuries, and they eke out a life among the dust, sand and exposure of the planet's deserts.

Although much of the landscape has been cleared of the scrap dumped in the early days of the colonies, in some of the more arid deserts the skeletons of old ships, vehicles and machines can still be seen littering the barren places. Not a people to let things go to waste, the Tetrarchii use these to build what they need, from shelter and appliances to vehicles and tools. They see use in something that others would consider beyond use. They have built a few sprawling towns across the landscape, which exist almost solely to slake thirst and entertain. Lack of water on the planet means that fluids are usually imported, and it is easier for a traveler to obtain a cup of Ardheimian mead on Tetrarch than a cup of water. They are also likely to find in these towns a good number of gambling dens, pawn shops, bars, and places of questionable virtue.

The Tetrarchii favour practical clothing that leaves none of the body exposed. The planet can be exceedingly hot by day, and freezing at night, and when a sandstorm picks up it can choke. Full head coverings such as tagelmusts are common, often accompanied by protective goggles to keep the sand out of the eyes. The lower parts of the legs and arms are usually wrapped to keep sand out of trousers, shirts and boots, and hard-wearing materials such as leather are preferred - this, of course, is recycled from the scrapyards. To inhabitants of neighbouring colonies, the Tetrarchii look unkempt and dirty: many

of them prefer to shave their hair completely or grow it into dreadlocks to avoid having to maintain it in their environment, and their clothing is scrappy, recycled, and worn.

As survival is uncertain on this planet, the Tetrarchii follow the rule of safety in numbers: being alone in the desert is an almost certain road to death. As such, they often live in small yet close-knit communes, which may or may not share blood-ties, where children are raised by all, everyday chores and work are shared by those who can undertake it, and skills are taught freely. Despite the hardships of life on this inhospitable planet, the Tetrarchii are generally a happy people, thankful for each day they are still alive,

openly expressing joy and celebrating every juncture of life. After all, they've got people they can rely on, an open sky above them, and plenty of land to make their own. Rites of passage, such as namings, coming of age, and wisenings, are marked with a Gathering, and fires burn on those nights that can be seen for miles around.

The naturally hot environment has led to the immigration of tae'go to the colony, and it is almost certain that Tetrarch has the largest population of this species in the cluster. The tae'go live on friendly terms with the human Tetrarchii, and some communities have the two species co-existing, sharing life in the communes.

The immigration of the tae'go has improved life for the humans there, as they have shared with them skills

that have enabled them to find food and water even in the harshest desert climate.

Few Tetrarchii join the military, though with recent events the numbers are increasing. Some see it as a way to explore a wider world beyond the scrapyard, others send their wages back to their commune so they can improve their lives. As its population is so small, Tetrarch does not have its own military, so recruits join up on a neighbouring colony and are sent where needed.

\textbf{Inspiration for characterisation:} \textit{Mad Max}, Tatooine (\textit{Star Wars}), \textit{Tank Girl}, the American Wild West, Burning Man, Nevada desert, \textit{Dune} by Frank Herbert, \textit{Firefly}.

\subsubsection{Zennor}

\textit{Pronunciation: ZEH-noor}

\textit{Name of people: Zennorite (sing.) / Zennorites (pl.)}

Located on the western side of the Segovax cluster, Zennor has the highest water-to-land ratio of all colonised planets in the cluster. From orbit the vast oceans can be clearly seen, as well as three main continents and a number of clusters and archipelagos. Due to the fact that it is on the borders of Terran Sovereignty space, Zennor has taken the brunt of the recent One Bakkar hostilities, and much of its culture and civilisation has been destroyed or driven to neighbouring systems. The One Bakkar first invaded the eastern continent of Al'Azif, once the technological centre for Zennor and home to the university of Al'Azure, famous for its innovative new forms of energy production. It is hoped that Al' Azif will one day be regained and the technological sector re-established. Unfortunately all the facilities were either destroyed or fell into the hands of the One Bakkar.

Melrose, both continent and capital city, was home to the main branches of the now defunct political system, replaced by martial law as a result of the One Bakkar invasion. It was here that the first colonists landed, using lumber from the forests to the west of the continent and stone from northern Lupeth to expand. Southern Melrose eventually became the agricultural centre for the planet. This continent is home to the Melrose Resistance, founded in 6011 by Lady von Beck, which aims to hamper any One Bakkar advancements by undertaking espionage and sabotage missions. The Resistance often take out supply lines of the enemy, using guerilla manoeuvres and partisan actions.

To the west of Zennor is the continent of Lupeth, comprised of two land masses separated by a partially submerged mountain range called the Spine of Lupeth. It is named after the wolf-like creatures that inhabit these mountains. The main city of this continent, Tebron, is currently guarded by the reserve armies of the Terran Sovereignty Army, and therefore relatively safe for inhabitants. This continent has been steeped in folklore for as long as there have been people on Zennor: the winds can be heard blowing between the peaks in the Spine of Lupeth, creating a howling sound that can be heard by those that live in the shadows of the Spine. Tebron functions as the initial landing zone for Sovereignty forces arriving on Zennor. It also functions as a respite where forces take leave when not on the front lines.

Humans were not the first humanoid race on Zennor, as the first colonists soon discovered that they shared the planet with an intelligent race called the Nesh'an, birdlike creatures with markings staining their skin of a colour that denotes their tribe of origin. These tribes do not mix or work together, and have very different approaches and motives. Humans live in relative peace with the Nesh'an, adopting friendly relations with the tribe denoted by the blue stainings, and leaving alone those denoted by red stainings.

Since the invasion of the One Bakkar, the people of Zennor have seen their priorities change considerably.

Many who were studying at Al'Azure, or involved in energy production, emigrated to Marazion V. With the collapse of the political system and implementation of martial law, most people have turned to agriculture and manual labour, while many have signed up for the military. Due to the limited amount of trade into the planet from neighbouring colonies, Zennorites lead simple lives with little time or resources for luxuries, fashion, study or the arts. Their clothing does not bear anything distinct, though it is usually

simple, unassuming, drab in colour and easily mended. If any jewellery is worn it is small, carved wooden pendants.

\textbf{Inspiration for characterisation}: current Earth, European folklore, partisan groups, simple living.

\subsubsection{Other planets}

The above listed planets are the largest and most populated planets in the cluster, however there do exist number of very barely habitable small planets and colonised moons. We would prefer that you pick a character that comes from one of the above listed planets, however we do allow a little variation if

requested. Please bear in mind that if you do wish to create another planet for your character to come from you must first contact a referee and liaise with them to make sure it fits into the \textit{Green Cloaks} canon. Note that any planets created this way will be extremely minor in terms of population and influence: the biggest most developed planets in the area are already listed.

\subsection{Mascen}

The recent history of this species has been riddled with exploitation and slavery since the war swept over their homeworld. After the One Bakkar ravaged their planet with technology and firepower far more advanced than anything the mascen possessed, they were enslaved in the wake of the ruin by a private organisation called the Terran Battlefield Reclamation, who fronted the atrocity by claiming they were salvaging anything left from the battle.

The slavery went unnoticed by the Terran Sovereignty until long after the mascen homeworld had disappeared behind the One Bakkar front line. By the time they had been freed and protected by Sovereignty law, there was nothing for them to return to. Many of the mascen adapted, forming tribes scattered throughout the Segovax cluster on the edges of Terran space. Many joined the military to aid the fight and reclaim their homeworld.

The mascen are unique in that they are comprised of two separate sub-species. They have a simple names for things, and this is also true of the names they have given to these sub-species: ``big'uns'' and ``little'uns''. Mascen big'uns are typically primitive, violent, and bloodthirsty. They are tougher than the average humanoid and able to withstand far more punishment than a human before being incapacitated. A common saying among soldiers fighting alongside them is ``when bullets start flying, hide behind a big'un''. Their skin is usually green or grey and they have small horns protruding from their foreheads. Their ears also taper to a point.

The little'uns are responsible for building and invention, and given more time for their culture to develop they would certainly be a very innovative species. The concept of explosions is very appealing to these little'uns. As it stands, however, they have been thrust into a universe they were not ready for, and their tinkering is far more of an eclectic experiment that accidentally yields results than a careful and precise practice. Stranger still are the few little'uns that hold a shamanic position in the mascen society, and display an innate connection to the Omega.

Hierarchy in mascen society is not based on size, however. Big'uns do not view the little'uns as inferior in any way; no more than a 6-foot human thinks any less about a 4-foot human. Even though they are physically stronger they acknowledge the intelligence and organisation of the little'uns.

\subsubsection{Physrep requirements}

Mascen big'uns must appear to be quite bulky, and have green to grey skin with a pair of horns on the forehead. Their ears should be pointed. Their clothes are simple and tribal adornments are common. Their equipment, however, is supplied by the Terrans and thus can be the same as everybody else's. The little'uns must appear scrawny, with similar colour skin to the big'uns but no horns. Tribal adornments are encouraged for little'uns also.

\subsubsection{Mascen names}

Due to the guttural nature of mascen language, this species prefer one- or two-syllable names in simple alias style, often primarily consisting of nouns and adjectives. Hard consonants and phonetics are also common, such as `k', `sc', `t'. There tends to be little difference in female and male gendered names.

\textbf{Examples}: Grin, Skunk, Krot, Krunk, Brack, Tooth, Rocko

\subsection{Myr'na}

The mysterious myr'na are the most recent species that the Terran Sovereignty has made contact with. They are a highly spiritual and peaceful people who are close in nature to the plane of Omega. They are blessed with the ability to clear their mind and tap into this power, healing the wounds of themselves and others. They have a relatively long lifespan, usually around 200-250 years. The myr'na's skin is a shade of blue or grey, with light-coloured hair. They dress in flowing, elaborate, brightly coloured clothing, adorned with bells and sashes. Their faces are painted with ornate geometric clan markings.

The myr'na are a race fundamentally connected to the Omega, and are able to channel its power naturally from birth. However the Omega is a dangerous place, filled with unimaginable horrors. If a myr'na becomes too lost in his or her own abilities, such as in the heat of battle, they are at risk of being consumed and drawn into the Omega. To combat this they practice pacifism and restraint in all things violent, but will do their best to protect their friends without harming others. Possibly as a result of this they are very conscious of their personal space. The act of touching another when not healing or without permission is considered extremely offensive.

If a myr'na decides they wish to fight they must give up all but their most innate connection to the Omega for the rest of their lives, lest they become an uncontrollable danger to those around them. When a myr'na takes this oath, they sever their connection to the Omega with a complex ritual. They then paint their clan markings in red, indicating they will both shed both their own blood and the blood of others. Over time their skin fades to a paler shade as the influence of Omega leaves them and they lose their inhibitions about their own personal space and touching.

The myr'na culture is composed of a system of clan families, ruled over by a council of elders. The five

major clans are:

\textbf{Akiyama} - Based in the foothills of the mountains of the myr'na homeworld, this particular clan have wealth based on mining.

\textbf{Kaneko} - Responsible for a large proportion of the trade routes, this family is one of the most powerful clans in the myr'na world. They have a large amount of political influence, being able to push votes and use their influence to keep the peace.

\textbf{Shizuka} - The main exports of the Shizuka clan are incense and fabric, and their dress tends to the more sumptuous and beautiful.

\textbf{Tanaka} - This clan is made up largely of agricultural families, who between them cultivate the most land amongst the myr'na.

\textbf{Yukimura} - Based in the polar areas of the myr'na homeworld, their main trade is fishing and hunting.

Their dress tends to be a little warmer than that of the clans in the central areas.

Many smaller clans are scattered across the land, and they tend to be more self-sufficient, but concentrated in smaller areas. Each clan works with the others for the harmony of the myr'na, and by sharing all that is produced, conflict is avoided.

The myr'na claim that they have observed the Terran Sovereignty for some time, but have only recently decided to venture outside their home system (whose exact location is not known by the Terrans, but is rumoured to be some distance from their space). It seems that a cultural rift has formed between the older myr'na who are set in their ways and the younger generations who wish to explore the galaxy.

\subsubsection{Physrep requirements}

The myr'na dress in flowing, brightly coloured clothes such as robes and kimonos. Many myr'na also complement their appearance with bells and sashes and non-metallic jewellery, smooth glass being a favourite. Skin must be a deep shade of blue or grey, if a healer, or a pale shade of blue or grey if a warrior. Hair must be white, grey, or silver or otherwise a light shade. Clan markings must be geometric (circles, squares, solid blocks etc, not tribal warpaint style). Healers can have clan markings of any colour besides

red; warriors must have red clan markings.

\subsubsection{Myr'na names}

Valuing peace and diplomacy, the language of this species is usually gentle and flowing, frequently using dipthongs (ae, ai, ou, ae, ia) to elongate words. Often a name is given to a child based on its meaning, and these meanings frequently refer to the natural world. A myr'na deeply values his/her family name, and in formal situations will introduce themselves with their full given name. Some myr'na who join the Terran Sovereignty Army or move in Terran societies choose to adopt an informal name that is more easily pronounceable to their allies.

\textbf{Examples}:

Female names - Aiko; Haruna; Rei; Kaeni; Maeyukai; Taemokou; Imaeru Male names - Chiang; Shiu; Baeru; Saubouru; Shichirou; Katoichi

\subsection{Tae'go}

Up until recently the reptilian tae'go were an elusive nuisance for the Terran Sovereignty Army. The only contact with them was their lightning-fast raids on food and weapons supplies. Their stealth and agility are far superior to anything encountered by the Terran Sovereignty, and the targets of the tae'go are rarely aware of their attackers before it's too late.

Having long ago fled the destruction of their desert homeworld, Tazrak, at the hands of the One Bakkar, the tae'go have been moving towards Terran space, making use of what little shelter they could find on

inhabitable planets. Rapidly running out of room, they hid on the fringes of the Terran Empire, dotted across the front line and living in small camps and communes. After a few years of enmity, negotiations were opened with the tae'go and an agreement was eventually reached. Able-bodied tae'go would be folded into the Terran military and be deployed on the front lines along with Terran forces. None will speak openly of their home{\dots} to return to tae'go space, they believe, is to invite death.

Tae'go are a highly nomadic people with a strong emphasis on family ties and kinship. Their traditional clothing is designed to be light yet suited to a desert climate. When not on covert missions they usually wear head wraps and long, simple robes designed to protect them from the sun. Many adorn themselves with gold (or at least gold-looking) jewellery, and this is generally seen as a status symbol. When the need for stealth arises they clothe themselves in dark, neutral colours to suit the terrain, making sure the garb is light and easy to move in. They consider massed charges, and other direct action, foolish and risky, and believe that even the largest enemy can be taken down with enough patience and carefully applied force.

Due to their highly patient nature, probably stemming from their cold-bloodedness, an even-handed tae'go can make an excellent diplomat, however they apply this patience in many aspects of their lives as they wait for opportunities to arise. Because of this tae'go also make the finest assassins and thieves, and since their merging with the Terran Sovereignty you will find tae'go in many professions, from scouts on the front line to hitmen in the employ of criminals.

\subsubsection{Physrep requirements}

The tae'go are a reptilian species but are as varied as humans. Some have snouts, some have tails, some have frills, some have all three. Being from a very hot climate, the tae'go must wear thick, heavy-looking clothing to keep warm so their metabolisms do not slow down. If a tae'go gets too cold they become sluggish and slow, which can be dangerous when running from an aggressor. Therefore, a player must physrep wrapping up warm unless the IC climate is particularly hot. Cloaks and shawls, for example, will achieve this effect (please note that if the OC climate is too hot to do this, referees will take note and you will not have to wrap up warm!).

\subsubsection{Tae'go names}

Due to the reptilian bodies of the tae'go, their language is sibilant, often putting emphases on an ``s'' , ``z'' or ``tz''. It is not unknown for them to hiss slightly, in particular when threatened. Tae'go names reflect this language.

\textbf{Examples}:

Female names - Krizzt; Sarassa; Larask; Drackt; Tzilisk Male names - Draszz; Sarass; Isklan; Zassk; Siszlak

\subsection{Vrede}

The vrede appear to despise war in all forms, seeing it as a barbaric last resort; however when they need to fight they are devastating, boasting high-tech weaponry and defensive equipment, and the minds of their tacticians are unparalleled. The actions of the vrede when the One Bakkar had almost pushed as far as Terra undoubtedly saved the human race from annihilation.

A new generation of vrede has recently been given permission from the vrede collective to join the Terrans and learn more about galaxy outside their area of space. From these vrede humans and other species have learned more about this elusive race. They have discovered that the vrede are a highly organised and efficient race, with a strict hierarchy built on the generation of their birth, with the earliest generations at the top and each generation below increasing in size. Each generation has authority over the younger generations, and these younger ones will follow the older almost without question. Reproduction - and information about it - in the vrede collective is strictly regulated and the vrede found among Terrans have no idea how it occurs.

The vrede lifespan is unknown, however it would seem to be many times that of a normal human. This also makes it hard to track the age of the vrede collective by generations alone, as it seems the birth of a new generation is decided upon by the upper generations and is not after a set time. The latest generation refer to themselves as the 25th and claim that they were born with the purpose of travelling outside the collective to learn from and assess the other races in the galaxy. They are part of an expeditionary force sent by their government to act as information gatherers, emissaries and their representatives among the allied peoples as well as additional support for the Terrans. The vrede, ever secretive and never wishing to give up an advantage, opted not to inform this generation of the inner working and details of their collective so that in the event of capture the information would not be at risk. As such the 25th know little beyond what is stated here and claim never to have met a vrede above the 20th generation; none know the whereabouts of the vrede home planet.

Vrede are demi-shapeshifters. They can slightly change their physical appearance but not their skin colour, which tends to be one colour with slight variations of shade. Some are even patterned with stripes or spots. This includes any clothes they appear to be wearing as they are simply extensions of the vrede's body. On rare occasions vrede will wear limited human-style clothes, such as a tie or waistcoat, but this is more out of curiosity about human culture and their obsession with fashions. They will seldom fully cover themselves, as this limits their shape shifting ability.

Each vrede carries advanced technology for their protection, know as vrede-tech. They are fiercely protective of it to the point where most vrede-tech is programmed to deactivate permanently upon the death of its user or if tampered with incorrectly. Vrede will go to great lengths to recover vrede-tech not in the possession of a vrede.

A vrede will never refer to someone by rank unless they are in their direct chain of command, and are highly secretive individuals as befits their race. They consider it an insult to the intelligence of whoever they are talking to to be completely frank, and it is good practice to allow another to find the hidden catch in a deal, or the half truth in their words. This does not mean they are untrustworthy, however it means making deals with the vrede becomes an interesting endeavour. One thing is certain however: if a vrede is being totally frank and honest with you about a deal, battle strategy, or in general conversation, they are insulting you.

\subsubsection{Physrep requirements}

Whatever form vrede take they are always a unified colour, though a rare few show spots, stripes or similar. Any clothes worn are actually an extension of the vrede's skin. However, this does not extend to kit such as ammo pouches or boots etc. All skin and clothes worn must be of the same colour, allowing slight variations of shade. This goes for spots and other patterns as well. All weapons and equipment should look of a high quality/high technology level.

\subsubsection{Vrede names}

Vrede are given designations of generation and number as a name by the collective. However, the generation of vrede that has been sent out into the wider world, and joined the Terran Sovereignty Army, have found it prudent and conducive to the development of connections with members of other races to

take on names from other races. They tend towards human names as these are the majority race in the Terran Sovereignty Army. Given that vrede are shapeshifters, gender is an alien concept to them. As such, a vrede may have any name they choose.

\chapter{Classes}

Every character in \textit{Green Cloaks} must be one of the following classes. Your class decides your race and specialisation, granting bonuses to your area of expertise as well as guiding your character progression.

At the start of every event, all classes receive as standard issue 1 dermograft patch and their wage.

\subsection{Human Classes}

With the exception of Adepts, humans receive their classes through training in their regiments. During this training they are given roles to play within the military structure, and are given extra training in their regiment's specialist skill. This specialist skill represents the role the regiment usually plays in the Terran Sovereignty Army. Non-humans do not receive training in their regiment's specialist skill, as their class is not received through military training in the Terran Sovereignty Army, but is instead gained through their upbringing or culture.

The regimental specialist skills are as follows:

\begin{table}[H]
\begin{tabular}{|l|l|} \hline 
Regiment & Regimental Specialisation \\
 \hline 109th Light Infantry & Melee \\
 \hline 23rd Heavy Infantry & Heavy Weapons \\
 \hline Delmont 205th & Self-sufficient \\
 \hline Kingskeep 98th & Stealth \\
 \hline \end{tabular}

\end{table}

The regimental specialist skill is granted at character creation to all non-Adept human characters at Tier 1 for free, and thereafter counts as a primary skill for purposes of spending skill points. See \textit{Chapter 13: Skills}.

\subsubsection{Adept}

\begin{table}[H]
\begin{tabular}{|l|l|l|} \hline 
Free starting skill(s) & Primary skills & Secondary skills \\
 \hline Omega Attunement 1* & Melee Weapons Omega Attunement 1* Omega Attunement 2* Pistol & Awareness Courage\par Omega Attunement 3* \\
 \hline \multicolumn{3}{|l|}{Skill points to spend: 3} \\
 \hline \multicolumn{3}{|l|}{Maximum armour type: light} \\
 \hline \end{tabular}

\end{table}

* indicates that you may choose any of the Omega Attunement skills; Omega Attunement 1 and 2, once chosen, will always be a primary skill for your character; what you choose for Omega Attunement 3 will always be a secondary skill for your character.

Adepts are an odd mix of science and the supernatural. They are the results of extensive scientific testing based upon strange technology previously encountered in the field. They are able to tap into the Omega plane (the ``other'' plane where ``other'' things live) by tearing holes in the fabric between it and the Alpha (the material) plane.This allows them to wield powers that more primitive cultures would call magic. Some other species have their own versions of Adepts but these tend to have a

more innate connection to the Omega than humans, who rely on implants and scientific modifications for their connection. Due to the extensive amount of testing performed on them their methods of creation are many and varied. They do, however, all possess some form of cybernetic implant which gives them a connection to the Omega, and this should be physrepped by the player.

For more information about the history and creation of human Adepts, see \textit{Chapter 18: Ritual and the} \textit{Omega}.

\subsubsection{Class attributes:}

\begin{itemize}
\item \textbf{Conditioning}: Adepts receive 15 focus points per day (see \textit{pp. 103}).

\item \textbf{Omega Disciple}: Adepts can undertake Thaumaturgy research (see \textit{Chapter 17: Thaumaturgy}).

\item \textbf{Arcane Rites}: Adepts can perform rituals (see \textit{pp. 103-104}).

\end{itemize}
\textit{Engineer}

\begin{table}[H]
\begin{tabular}{|l|l|l|} \hline 
Free starting skill(s) & Primary skills & Secondary skills \\
 \hline Engineering & Engineering & Awareness \\
 \hline  & Explosives & Extra Hits \\
 \hline Regimental specialisation & Pistol & Rifle \\
 \hline  & Self-sufficient &  \\
 \hline  & Regimental specialisation &  \\
 \hline \multicolumn{3}{|l|}{Skill points to spend: 2} \\
 \hline \multicolumn{3}{|l|}{Maximum armour type: medium} \\
 \hline \end{tabular}

\end{table}

\subsubsection{Class attributes:}

\begin{itemize}
\item Engineers receive Tier 1 in their regiment's specialist skill, and this skill counts as a primary skill for the purposes of spending skill points.

\item \textbf{Inventor}: Engineers can undertake Engineering research (see \textit{pp. 91-92}).

\item \textbf{Multispec}: At character creation Engineers are issued with a multispectrum systems analyser (or ``multispec'' for short). This takes the form of a small portable computer that assists with a variety of engineering tasks. Players should physrep it appropriately.

\item \textbf{Chief Engineer}: Engineers can lead up to two other characters who also have the Engineering skill while analysing, crafting, repairing or hacking. Each assistant grants a 20-second reduction on the time it takes to repair armour or items when using the Engineering skill. The Engineering tier of the leading Engineer is used as the base. The Engineering tier of the assistants does not matter when using this ability. See \textit{pp. 90-94} for more details and the times required for repairing.

\item \textbf{Teaching}: Engineers can teach how to craft an item of tech that they know how to make to those with the Engineering skill. Only one character may be taught at a time, and only items that have been fully learned or researched by the Engineer may be taught; any items on which research is incomplete cannot be taught. See \textit{pp. 92} for more details.

\end{itemize}
\textit{Heavy Weapons Specialist}

\begin{table}[H]
\begin{tabular}{|l|l|l|} \hline 
Free starting skill(s) & Primary skills & Secondary skills \\
 \hline Heavy Weapons & Bulging Biceps & Courage \\
 \hline \textbf{\textit{or}} & Extra Hits & Explosives \\
 \hline Melee Weapons & Heavy Weapons & Pistol \\
 \hline \textbf{\textit{or}} & Melee Weapons &  \\
 \hline Extra Hits &  &  \\
 \hline  & Regimental specialisation &  \\
 \hline Regimental specialisation &  &  \\
 \hline \multicolumn{3}{|l|}{Skill points to spend: 2} \\
 \hline \multicolumn{3}{|l|}{Maximum armour type: heavy} \\
 \hline \end{tabular}

\end{table}

\subsubsection{Class attributes:}

\begin{itemize}
\item Heavy Weapons Specialists receive Tier 1 in their regiment's specialist skill, and this skill counts as a primary skill for the purposes of spending skill points.

\item \textbf{Proper Training}: Heavy Weapons Specialists can move faster when using heavy weapons than other classes. They are granted the following bonus at the relevant tiers of the Heavy Weapons skill:

\begin{itemize}
\item Tier 1: may move at a slow walking pace while firing heavy weapons.

\item Tier 2: may move at a walking pace while firing heavy weapons.

\item Tier 3: may move at a jogging pace while firing heavy weapons.

\end{itemize}
\item \textbf{Bullet-proof}: Heavy Weapons Specialists receive 1 extra armour hit per location in addition to any armour they wearing.

\item \textbf{Canny}: Once per encounter a Heavy Weapons Specialist may use 1 additional call from the Melee Weapons skill (see \textit{pp. 75}), provided they are already able to use that call.

\end{itemize}
\textit{Medic}

\begin{table}[H]
\begin{tabular}{|l|l|l|} \hline 
Free starting skill(s) & Primary skills & Secondary skills \\
 \hline Medicae & Courage & Bulging Biceps \\
 \hline  & Medicae & Extra Hits \\
 \hline Regimental specialisation & Pharmacology & Rifle \\
 \hline  & Pistol &  \\
 \hline  & Regimental specialisation &  \\
 \hline \multicolumn{3}{|l|}{Skill points to spend: 2} \\
 \hline \multicolumn{3}{|l|}{Maximum armour type: medium} \\
 \hline \end{tabular}

\end{table}

\subsubsection{Class attributes:}

\begin{itemize}
\item Medics receive Tier 1 in their regiment's specialist skill, and this skill counts as a primary skill for the purposes of spending skill points.

\item \textbf{Medical Supplies}: Medics are issued with an additional two dermograft patches per event for free (giving them a total of three).

\item \textbf{Medical Researcher}: Medics can undertake Pharmacology research if they have the Pharmacology skill (see \textit{pp. 97}).

\item \textbf{Deathwatch}: Medics can spend 5 seconds of appropriate roleplay to diagnose a dying character, after which the dying character can tell them their current death count.

\item \textbf{Triage}: A Medic can lead up to two other characters who also have the Medicae skill while working. Each assistant grants a 5-second reduction on the time it takes to use the Stabilise call while using the Medicae skill. The minimum time this call can be reduced to is 10 seconds. Each assistant grants a 20-second reduction on the time it takes to use the Heal call while using the Medicae skill, except where it would bring the time below 30 seconds (see chart below). The minimum time this call can be reduced to is 30 seconds. The Medicae tier of the leading Medic is used as the base for the time it takes to Stabilise or Heal. The Medicae tier of the assistants does not matter when using this ability.

\end{itemize}
\textit{Stabilisation and healing times with the Medicae skill:}

\begin{table}[H]
\begin{tabular}{|l|l|l|l|} \hline 
\multirow{1}{*}{}& Tier 1 & Tier 2 & Tier 3 \\
\cline{2-2}\cline{3-3}\cline{4-4} & \multicolumn{3}{|l|}{Stabilisation times using Medicae skill (in seconds)} \\
 \hline Alone & 30 & 25 & 20 \\
 \hline 1 assistant & 25 & 20 & 15 \\
 \hline 2 assistants & 20 & 15 & 10 \\
 \hline  & \multicolumn{3}{|l|}{Heal times using Medicae skill (in seconds)} \\
 \hline Alone & 100 & 80 & 60 \\
 \hline 1 assistant & 80 & 60 & 40 \\
 \hline 2 assistants & 60 & 40 & 30 \\
 \hline \end{tabular}

\end{table}

\textit{Scout}

\begin{table}[H]
\begin{tabular}{|l|l|l|} \hline 
Free starting skill(s) & Primary skills & Secondary skills \\
 \hline Backstab & Awareness & Dodge \\
 \hline \textbf{\textit{or}} & Backstab & Pistol \\
 \hline Stealth & Self-sufficient & Rifle \\
 \hline  & Stealth &  \\
 \hline Regimental specialisation &  &  \\
 \hline  & Regimental specialisation &  \\
 \hline \multicolumn{3}{|l|}{Skill points to spend: 2} \\
 \hline \multicolumn{3}{|l|}{Maximum armour type: medium} \\
 \hline \end{tabular}

\end{table}

\textbf{\textit{Class attributes:}}

\begin{itemize}
\item Scouts receive Tier 1 in their regiment's specialist skill, and this skill counts as a primary skill for the purposes of spending skill points.

\textbf{Ranger}: A pair of 109th Scouts have been sent out to track a large and aggressive mascen warband, headed northeast roughly two hours ago. They know they were last seen at the edge of the forest, so they begin their search there. Pvt. Sorrel carefully inspects the ground, noting hardness or softness of the dirt, searching for tracks. She searches for a few minutes, and it's almost at the same time that she discovers the tracks entering the forest as Sgt. Grey notices the crushed undergrowth and snapped twigs. They spend the next couple of minutes inspecting these signs further, discussing quietly with each other what they have found: how many could this force be? Were they running or walking? Finally, the referee approaches them and offers the information: they estimate that the warband was about 15-strong, moving fast{\dots} and they are no longer heading northeast. Sorrel and Grey stand up and look around, and see the fires in the distance to the west, back at basecamp.

\item \textbf{Ranger}: Scouts are able to spend time checking for, or removing, tracks in an area with appropriate roleplay. The amount of time taken

and the information granted as a result of it are at the discretion of the referee present.

\item \textbf{Fox Step}: Scouts receive an additional 2 seconds to the time allowed to move in cover while using the Stealth skill (see \textit{pp. 80-81}). Note this allows them to move in cover using Tier 1 Stealth.

\item \textbf{Take Cover!}: Scouts may take one other person into stealth with them. When this is done neither the person covered nor the Scout may move or stealth will be broken, regardless of the tier of Stealth used. Please use appropriate roleplay in attempting to cover the person you are hiding.

\textbf{Take Cover!}: Pvt. Gundarsson hears the distinctive sound of the enemy's steel-capped boots approaching, and knows they outnumber the small force of Kingskeep's reconnaisance unit that has been sent out. Around him he sees the rest of the unit disappear into the high ferns, but ahead of him he can see Vasily, their Adept, out in the open. He makes a snap decision and grabs Vasily by the shoulders. ``We've gotta get into cover. Let them pass. We can't fight this time.'' Vasily nods, and follows Gundarsson as he shows the way into the deeper cover. Vasily lies on the grass face down, peering through a small gap in the undergrowth, and Gundarsson carefully moves the ferns so they obscure both him and his comrade. He gets them both covered just in time, and as the enemy march past Vasily holds his breath.

\end{itemize}

\textit{Sniper}

\begin{table}[H]
\begin{tabular}{|l|l|l|} \hline 
Free starting skill(s) & Primary skills & Secondary skills \\
 \hline Rifle & Awareness & Courage \\
 \hline \textbf{\textit{and}} & Marksman & Dodge \\
 \hline Marksman & Rifle & Pistols \\
 \hline  & Stealth &  \\
 \hline Regimental specialisation &  &  \\
 \hline  & Regimental specialisation &  \\
 \hline \multicolumn{3}{|l|}{Skill points to spend: 1} \\
 \hline \multicolumn{3}{|l|}{Maximum armour type: medium} \\
 \hline \end{tabular}

\end{table}

\subsubsection{Class attributes:}

\begin{itemize}
\item Snipers receive Tier 1 in their regiment's specialist skill, and this skill counts as a primary skill for the purposes of spending skill points.

\item \textbf{Tools for the Job}: Snipers are issued 5 marksman rounds per event as standard.

\item \textbf{Incapacitate the Target}: Snipers can use the Cripple call as per the Marksman skill (see \textit{pp. 73}), without expending any marksman rounds, once per encounter.

\item \textbf{Keen Eye}: Snipers receive a 5-second reduction on the aiming time when using the marksman skill.

\end{itemize}

\begin{table}[H]
\begin{tabular}{|l|l|l|l|} \hline 
\multirow{1}{*}{}& \multicolumn{3}{|l|}{Aiming times for Cripple and Marksman call using the Marksman skill (in seconds)} \\
\cline{2-2}\cline{3-3}\cline{4-4} & Tier 1 & Tier 2 & Tier 3 \\
 \hline Cripple call & 40 & 35 & 30 \\
 \hline Cripple call + Sniper class & 35 & 30 & 25 \\
 \hline Marksman call & 45 & 40 & 35 \\
 \hline Marksman call + Sniper class & 40 & 35 & 30 \\
 \hline \end{tabular}

\end{table}

\textit{Trooper}

\begin{table}[H]
\begin{tabular}{|l|l|l|} \hline 
Free starting skill(s) & Primary skills & Secondary skills \\
 \hline Dodge & Courage & Bulging Biceps \\
 \hline \textbf{\textit{or}} & Dodge & Pistol \\
 \hline Melee Weapons & Melee Weapons & Self-sufficient \\
 \hline \textbf{\textit{or}} & Rifle &  \\
 \hline Rifle &  &  \\
 \hline  & Regimental specialisation &  \\
 \hline Regimental specialisation &  &  \\
 \hline \multicolumn{3}{|l|}{Skill points to spend: 2} \\
 \hline \multicolumn{3}{|l|}{Maximum armour type: heavy} \\
 \hline \end{tabular}

\end{table}

\subsubsection{Class attributes:}

\begin{itemize}
\item Troopers receive Tier 1 in their regiment's specialist skill, and this skill counts as a primary skill for purposes of spending skill points.

\item \textbf{Tenacious}: Troopers receive 2 global hits as per the the Dodge skill (see \textit{pp. 70}) without the need for the skill, and these can stack with the Dodge skill (a total of 2 without the Dodge skill; 3 at Tier 1; 4 at tier 2; and 5 at Tier 3).

\item \textbf{Dodged the Bullet!}: Once a Trooper reaches Tier 3 of the Dodge skill they only need to expend 2 global hits granted by the skill to avoid a call instead of all of them. This means they can potentially use the Tier 3 Dodge skill ability twice per encounter.

\item \textbf{Canny}: Once per encounter a Trooper may use 1 additional call from either the Melee Weapons skill (see \textit{pp. 75}) or the Rifle skill (see \textit{pp. 79}), provided they are already able to use that call.

\end{itemize}
\subsection{Mascen Classes}

\textit{Big'un}

\begin{table}[H]
\begin{tabular}{|l|l|l|} \hline 
Free starting skill(s) & Primary skills & Secondary skills \\
 \hline Bulging Biceps & Bulging Biceps & Explosives \\
 \hline \textbf{\textit{or}} & Courage & Self-sufficient \\
 \hline Melee Weapons & Extra Hits & Rifle \\
 \hline  & Melee Weapons &  \\
 \hline \multicolumn{3}{|l|}{Skill points to spend: 3} \\
 \hline \multicolumn{3}{|l|}{Maximum armour type: heavy} \\
 \hline \end{tabular}

\end{table}

\subsubsection{Class attributes:}

\begin{itemize}
\item \textbf{Tough Hide}: Big'uns receive 1 extra body hit to all locations (a total of 3 without skills or armour).

\item \textbf{Unwavering}: Big'uns receive 1 Fear resist per encounter as per the Courage skill (see \textit{pp. 69}) without the need for the skill, and these can stack with the Courage skill (a total of 1 at without the Courage skill; 2 at tTer 1; 3 at Tier 2; and 4 at Tier 3). Once a Big'un reaches Tier 3 of the Courage skill they also gain 1 additional use per day of the ``Courage: Resist Terror'' ability, allowing them to resist up to 2 calls of Terror a day.

\item \textbf{Unstoppable}: Twice per day Big'uns may enter a state of pure rage for 10 seconds. When in this state all damage taken is delayed until the rage ends, at which point they suffer all damage taken at once. While using this ability you must make the call ``No effect'' when hit. This ability does not

stop calls that would knock you down or otherwise impede your movement. You must roleplay being extremely aggressive and angry towards friend and foe alike, you have no thought for your safety and will not back down from a fight. This ability may not be used twice in the same encounter.

\end{itemize}
\textit{Little'un Shaman}

\begin{description}
\item[]
    \begin{table}[H]
\begin{tabular}{|l|l|l|} \hline 
Free starting skill(s) & Primary skills & Secondary skills \\
 \hline Omega Spirit Attunement & Dodge\par Omega Spirit Attunement Omega Attunement 2* Self-sufficient & Awareness\par Omega Attunement 3* Pistol \\
 \hline \multicolumn{3}{|l|}{Skill points to spend: 3} \\
 \hline \multicolumn{3}{|l|}{Maximum armour type: medium} \\
 \hline \end{tabular}

\end{table}

\item[	*]indicates that you may choose any of the Omega Attunement skills; Omega Attunement 2, once chosen, will always be a primary skill for your character; what you choose for Omega Attunement 3 will always be a secondary skill for your character.

\end{description}

\subsubsection{Class attributes:}

\begin{itemize}
\item \textbf{Easily Squished}: Little'un Shamans receive 1 less body hit to all locations (a total of 1 without skills or armour).

\item \textbf{Resourceful}: Little'un Shamans have the ability to re-draw their first Self-sufficient result once per day.

\item \textbf{At one with the Omega}: Little'un Shamans receive 20 focus points per day (see \textit{pp. 103}).

\item \textbf{Arcane Rites}: Little'un Shamans can perform rituals (see \textit{pp. 103-104}).

\item \textbf{Omega Disciple}: Little'un Shamans can undertake Thaumaturgy research (see \textit{Chapter 17:} \textit{Thaumaturgy}).

\end{itemize}
\textit{Little'un Tinkerer}

\begin{table}[H]
\begin{tabular}{|l|l|l|} \hline 
Free starting skill(s) & Primary skills & Secondary skills \\
 \hline Self-sufficient & Dodge & Awareness \\
 \hline \textbf{\textit{or}} & Engineering & Explosives \\
 \hline Engineering & Pistol & Rifle \\
 \hline  & Self-sufficient &  \\
 \hline \multicolumn{3}{|l|}{Skill points to spend: 3} \\
 \hline \multicolumn{3}{|l|}{Maximum armour type: medium} \\
 \hline \end{tabular}

\end{table}

\subsubsection{Class attributes:}

\begin{itemize}
\item \textbf{Easily Squished}: Little'un Tinkerers receive 1 less body hit to all locations (a total of 1 without skills or armour).

\item \textbf{Resourceful}: Little'un Tinkerers have the ability to re-draw their first Self-sufficient result once per day.

\item \textbf{Mechnically Minded}: If they have the Engineering skill, Little'un Tinkerers can undertake Engineering research (see \textit{pp. 91-92}).

\item \textbf{Chief Engineer}: Little'un Tinkerers can lead up to two other characters who also have the Engineering skill while analysing, crafting, repairing or hacking. Each assistant grants a 20-second reduction to the time it takes to repair armour or items when using the Engineering skill. The Engineering tier of the leading Engineer is used as the base. The Engineering tier of the assistants does not matter when using this ability. For the times required for repair using the Engineering skill, see \textit{pp. 94}.

\item \textbf{Teaching}: Little'un Tinkerers can teach how to craft an item of tech that they know how to make to those with the Engineering skill. Only one character may be taught at a time, and only items that have been fully learned or researched by the Engineer may be taught; any items on which research is incomplete cannot be taught. See \textit{pp. 92} for more details.

\end{itemize}
Myr'na Classes

\textit{Healer}

\begin{description}
\item[]
    \begin{table}[H]
\begin{tabular}{|l|l|l|} \hline 
Free starting skill(s) & Primary skills & Secondary skills \\
 \hline Omega Body Attunement & Extra Hits Medicae\par Omega Body Attunement Omega Attunement 2* & Awareness Courage\par Omega Attunement 3* \\
 \hline \multicolumn{3}{|l|}{Skill points to spend: 3} \\
 \hline \multicolumn{3}{|l|}{Maximum armour type: light} \\
 \hline \end{tabular}

\end{table}

\item[	*]indicates that you may choose any of the Omega Attunement skills; Omega Attunement 2, once chosen, will always be a primary skill for your character; what you choose for Omega Attunement 3 will always be a secondary skill for your character.

\end{description}

\subsubsection{Class attributes:}

\begin{itemize}
\item \textbf{Divine}: Myr'na Healers receive 15 focus points per day (see \textit{pp. 103}).

\item \textbf{Arcane Rites}: Myr'na Healers can perform rituals (see \textit{pp. 103-104}).

\item \textbf{Omega Disciple}: Myr'na Healers can undertake Thaumaturgy research (see \textit{Chapter 17:} \textit{Thaumaturgy})

\item \textbf{Martyr}: Myr'na Healers have the ability to absorb the wounds of others. After 5 seconds of roleplay a myr'na Healer may use the Heal call (see \textit{pp. 28}) to restore 1 body hit to 1 location on a target on touch. However, hits healed this way cause the same amount of damage (bypassing global and armour hits) to the myr'na Healer performing the healing. They can choose the location(s) that are damaged. You cannot assign damage to a location that has no body hits remaining.

\item \textbf{Do No Harm}: Myr'na Healers are pacifists. They will resist the urge to knowingly deal physical damage of any kind to living creatures. Creatures of the Omega (demons, spirits, etc) are not counted as living by the myr'na and can be attacked. If the myr'na Healer finds themselves in a situation where they deal physical damage to a living creature, they will be wracked by intense physical pain for 120 seconds and have a headache for an hour. If a myr'na Healer ever inflicts harm that directly results in the death of a living creature they will enter a coma-like trance. In this trance they can be led about and follow very basic instructions. The player must inform a referee as soon as possible if this occurs and they will advise what happens next.

\item \textbf{Arcane Essence}: Myr'na Healers may negate up to 4 focus points of Omega Attunement powers used against them per day. This ability works as per the Omega Protection Attunement power, Negate (see \textit{pp. 77}), however no verbal activation of the skill is required: they simply call ``Negate'' and the number of focus points negated.

\end{itemize}
\textit{Warrior}

\begin{table}[H]
\begin{tabular}{|l|l|l|} \hline 
Free starting skill(s) & Primary skills & Secondary skills \\
 \hline Melee & Courage Extra hits Medicae\par Melee Weapons & Bulging Biceps Heavy Weapons Rifle \\
 \hline \multicolumn{3}{|l|}{Skill points to spend: 3} \\
 \hline \multicolumn{3}{|l|}{Maximum armour type: medium} \\
 \hline \end{tabular}

\end{table}

\subsubsection{Class attributes:}

\begin{itemize}
\item \textbf{Unwavering}: Myr'na Warriors receive 1 Fear resist per encounter as per the Courage skill (see \textit{pp.}

\textit{69}) without the need for the skill, and these can stack with the Courage skill (a total 1 at without the Courage skill; 2 at Tier 1; 3 at Tier 2; and 4 at Tier 3). Once a myr'na Warrior reaches Tier 3 of the Courage skill they also gain 1 additional use per day of the Courage: Resist Terror ability, allowing them to resist up to 2 calls of Terror a day.

\item \textbf{Arcane Essence}: Myr'na Warriors may negate up to 4 focus points of Omega Attunement powers used against them per day. This ability works as per the Omega Protection Attunement ability, Negate (see \textit{pp. 77}) however no verbal activation of the skill is required: they simply call ``Negate'' and the number of focus points negated.

\item \textbf{Canny}: Once per encounter a myr'na Warrior may use 1 additional call from the Melee Weapons skill, provided they are already able to use the call.

\end{itemize}
Tae'go

\begin{table}[H]
\begin{tabular}{|l|l|l|} \hline 
Free starting skill(s) & Primary skills & Secondary skills \\
 \hline Pharmacology & Backstab & Dodge \\
 \hline \textbf{\textit{or}} & Pharmacology & Medicae \\
 \hline Stealth & Self-sufficient & Rifle \\
 \hline  & Stealth &  \\
 \hline \multicolumn{3}{|l|}{Skill points to spend: 3} \\
 \hline \multicolumn{3}{|l|}{Maximum armour type: medium} \\
 \hline \end{tabular}

\end{table}

\subsubsection{Class attributes:}

\begin{itemize}
\item \textbf{Chemically Minded}: If they have the Pharmacology skill, tae'go can undertake Pharmacology research (see \textit{pp. 97}).

\item \textbf{Bilious}: Twice per day tae'go may roleplay spitting/regurgitating on a melee weapon for 5 seconds. After this is done the next strike from the melee weapon has either the Sleep call (see \textit{pp. 30}) or the Paralyse call (see \textit{pp. 29}).

\item \textbf{Ranger}: Tae'go are able to spend time checking or removing tracks in an area with appropriate roleplay. The amount of time taken and the information granted as a result are at the discretion of the referee present.

\item \textbf{Fox Step}: Tae'go receive an additional 2 seconds to the time allowed to move in cover while stealthed (see \textit{pp. 80-81}). Note this allows them to move in cover using Stealth Tier 1.

\end{itemize}
\subsection{Vrede}

\begin{table}[H]
\begin{tabular}{|l|l|l|} \hline 
Free starting skill(s) & Primary skills & Secondary skills \\
 \hline Engineering & Courage & Melee Weapons \\
 \hline \textbf{\textit{or}} & Dodge & Pistol \\
 \hline Rifle & Engineering & Self-sufficient \\
 \hline  & Rifle &  \\
 \hline \multicolumn{3}{|l|}{Skill points to spend: 3} \\
 \hline \multicolumn{3}{|l|}{Maximum armour type: may not wear any armour} \\
 \hline \end{tabular}

\end{table}

\subsubsection{Class attributes:}

\begin{itemize}
\item \textbf{Can't Take it With You}: Due to their shapeshifting ability, vrede cannot wear armour.

\item \textbf{Tech-minded}: Vrede can undertake Engineering research if they have the Engineering skill (see \textit{pp. 91-92}), and are the only race capable of using and, with the Engineering skill, manufacturing vrede- tech items.

\item \textbf{Chief Engineer}: Vrede can lead up to two other characters who also have the Engineering skill while analysing, crafting, repairing or hacking. Each assistant grants a 20-second reduction to the time it takes to repair armour or items when using the Engineering skill. The Engineering tier of the leading Engineer is used as the base. The Engineering tier of the assistants does not matter when using this ability. See \textit{pp. 90-94} for more details and the times required for repairing.

\item \textbf{Teaching}: Vrede can teach to those with the Engineering skill how to craft an item of tech that the teacher knows how to make. Only one character may be taught at a time, and only items that have been fully learned or researched by the Engineer may be taught; any items on which research is incomplete cannot be taught. See \textit{pp. 92} for more details. The exception to this is vrede-tech, which may only be taught to other vrede with the Engineering skill.

\item \textbf{Vrede-tech}: At character creation vrede characters are issued with a vrede-tech light energy shield and a vrede-tech rifle.

\begin{itemize}
\item The \textbf{vrede-tech light energy shield} grants 4 global hits and regenerates all hits after each encounter (normal energy shields only regenerate half). The item is genetically keyed to the owner and will deactivate permanently upon their death.

\item The \textbf{vrede-tech rifle} can be used once per encounter to make one of the following calls: Through (see \textit{pp. 32}), Stun (see \textit{pp. 31}), Knockdown (see \textit{pp. 29}) or Disarm (see \textit{pp. 27}). The item is genetically keyed so that it can only be used by its owner, and will deactivate permanently on their death.

\end{itemize}

\end{itemize}
\chapter{Skills}

Skills are the different abilities a character may possess that reflect in-character knowledge, training or specialisations. Each skill has three tiers. You must have gained the previous tier in order to advance to the next tier. If you advance to the next tier your lower tier is replaced.

Each character starts with a certain number of skill points that are listed in \textit{Chapter 12: Classes}. Starting characters must spend all of their skill points, and they may not be saved for later use. If a character survives a main event, then they earn 2 skill points. Skill points earned by survival of an event may be saved for later use.

If you have saved skill points and your character dies, you may use up to 50\% (rounded up) of the saved skill points on your new character, but only after they have survived an event (they may not be spent at character creation).

Each class has two sets of skills: \textbf{primary} and \textbf{secondary}. These are listed in \textit{Chapter 12: Classes}. Primary skills are a class's main specialisation and it is cheapest to purchase and advance skills in this set.

The skill point costs for \textbf{primary} skills are as follows:

\begin{table}[H]
\begin{tabular}{|l|l|l|} \hline 
\multicolumn{3}{|l|}{Primary skill costs} \\
 \hline Tier 1 & Tier 2 & Tier 3 \\
 \hline 1 & 2 & 3 \\
 \hline \end{tabular}

\end{table}

Secondary skills are a class's sub-specialisation and are more expensive than primary skills to purchase and advance.

The skill point costs for \textbf{secondary} skills are as follows:

\begin{table}[H]
\begin{tabular}{|l|l|l|} \hline 
\multicolumn{3}{|l|}{Secondary skill costs} \\
 \hline Tier 1 & Tier 2 & Tier 3 \\
 \hline 2 & 3 & 4 \\
 \hline \end{tabular}

\end{table}

You may purchase or advance skills from outside your class's primary and secondary skill sets. Doing so is known as taking a \textbf{veteran pick}.

The skill point costs for veteran picks are as follows:

\begin{table}[H]
\begin{tabular}{|l|l|l|l|} \hline 
 & Tier 1 & Tier 2 & Tier 3 \\
 \hline 1st veteran pick & 2 & -- & -- \\
 \hline 2nd veteran pick & 3 & 3 & -- \\
 \hline 3rd veteran pick & 4 & 4 & 4 \\
 \hline 4th veteran pick & 5 & 5 & 5 \\
 \hline 5th veteran pick & 6 & 6 & 6 \\
 \hline 6th veteran pick & 7 & 7 & 7 \\
 \hline 7th veteran pick & 8 & 8 & 8 \\
 \hline 8th veteran pick & 9 & 9 & 9 \\
 \hline 9th veteran pick & 10 & 10 & 10 \\
 \hline 10th veteran pick & 11 & 11 & 11 \\
 \hline \end{tabular}

\end{table}

To use this table, simply locate in the left hand column the number of veteran picks this new skill purchase would give you, and then move along to the column with the tier of skill you wish to purchase.

\subsection{Awareness}

\textit{Some people are simply more observant than others. A keen eye and an understanding of situations that can present danger, such as ambushes, hidden enemies, and traps, can be the skill that saves a regiment from sudden and unexpected annihilation.}

A referee is generally required for most uses of this skill and the character must have a valid reason to use the skill. A referee may also give extra information to a character using Awareness in a relevant situation.

\textit{Tier 1:}

\begin{itemize}
\item A character may declare to a referee the use of this skill to search a location for hidden objects or people. The search takes \textbf{60 seconds} of appropriate roleplay. After this time the character declares ``Awareness 1''. This call will reveal to the caller the location of any person or object within 30 feet under the effects of Stealth 1 (see \textit{pp. 80-81}).

\item If this call is made within 30 feet of you and is of a high enough level to locate you in stealth you must clearly state your Stealth tier so the caller can hear you. If you are hidden, but not stealthed, you must say ``0''.

\item Once revealed to them, the caller can point out and reveal stealthed characters or objects to other characters who will then be able to see them also.

\end{itemize}
\textit{Tier 2:}

\begin{itemize}
\item A character may declare to a referee the use of this skill to search a location for hidden objects or people. The search takes \textbf{50 seconds} of appropriate roleplay. After this time the character declares ``Awareness 2''. This call will reveal to the caller the location of any person or object within 30 feet under the effects of Stealth 1 or 2.

\item If this call is made within 30 feet of you and is of a high enough level to locate you in stealth you must clearly state your Stealth tier so the caller can hear you. If you are hidden, but not stealthed, you must say ``0''.

\item Once revealed to them, the caller can point out and reveal stealthed characters or objects to other characters who will then be able to see them also.

\item \textbf{Once per day} the character may think on and/or discuss a problem with others for at least 5 minutes. This must be done in the presence of a referee. Once finished the referee may elect to give the player extra insight into the problem. The effectiveness of this ability depends on the situation and discretion of the referee.

\end{itemize}
\textit{Tier 3:}

\begin{itemize}
\item A character may declare to a referee the use of this skill to search a location for hidden objects or people. The search takes \textbf{40 seconds} of appropriate roleplay. After this time the character declares ``Awareness 3''. This call will reveal to the caller the location of any person or object within 30 feet under the effects of Stealth 1, 2 or 3.

\item If this call is made within 30 feet of you and is of a high enough level to locate you in stealth you must clearly state your Stealth tier so the caller can hear you. If you are hidden, but not stealthed, you must say ``0''.

\item Once revealed to them, the caller can point out and reveal stealthed characters or objects to other characters who will then be able to see them also.

\item \textbf{Twice per day} the character may think on and/or discuss a problem with others for at least 5 minutes. This must be done in the presence of a referee. Once finished the referee may elect to give the player extra insight into the problem. The effectiveness of this ability depends on the situation and discretion of the referee.

\textbf{Awareness}: Peter is playing a character with the Awareness skill, and while scouting out an area was ambushed in a clearing by people who were hidden using Stealth. As a result of this experience, he decides that he does not want to be ambushed again, and in the next clearing he informs a referee that he is using his Awareness skill to check the area. He roleplays 50 seconds of searching the area, and then declares ``Awareness 2'', causing anyone within 30 feet hidden under Stealth 1 or 2 to be revealed to his character.

\end{itemize}

\subsection{Backstab}

\textbf{\textit{Dagger}}

A melee weapon between 6 and 18 inches in length. Anyone can use this weapon, and no skill is required to do so.

Daggers can be held in one or two hands while being used.

You may use one dagger in each hand at the same time.

\textit{Know exactly when and where to strike: from behind, when the enemy's not looking.}

A single call granted by Backstab can be used on the first attack made with a \textbf{dagger} on a target from behind. Only \textbf{one call} can be made this way on any \textbf{one target per encounter}. The target may be aware of attack, however they must be bodily facing away from the attacker. You cannot reach around behind someone to strike.

\textit{Tier 1:}

Ability to use the call Through (see \textit{pp. 32}) when using Backstab.

\textit{Tier 2:}

Ability to use the calls Through or Cripple (see \textit{pp. 27}) when using Backstab.

\textit{Tier 3:}

Ability to use the calls Through, Cripple or Stun (see \textit{pp. 31}) when using Backstab.

\subsection{Bulging Biceps}

\textit{Life can be tough, and life on the front lines even more so. Years of heavy lifting, training, physical combat and survival in harsh climates have paid off, giving the ability to use your brute strength to protect, aid, and destroy.}

Bulging Biceps represents a character having above-average strength. As well as granting the listed abilities, it can also be applied to relevant roleplay situations under guidance from a referee, and allows a character to use a shield.

This skill modifies the use of the Grapple call (see \textit{pp. 28}).

\textbf{\textit{Using a Shield}}

A shield blocks all incoming physical attacks from ranged and melee weapons, unless otherwise stated in the call (if one accompanies the attack. See \textit{Chapter 8: Calls}.). The maximum size of a shield is 60 inches in height by 36 inches width. For the shield to grant defence, it must be held in one or two hands by the player; if it is slung or carried on the back it does not grant any protection, and hits will be taken as normal to the covered location.

If you come under significant fire from ranged weapons while using a shield to block the darts, you must roleplay \textbf{bracing against the force of the fire} until it has decreased to a less significant level. You are still able to move while bracing, but only at a slow walking pace.

If you use your shield to block a physical attack that is accompanied by a call that can be blocked (see \textit{Chapter 8: Calls}), you must roleplay as if you had just blocked a particularly strong shot or challenging blow.

A shield may be rendered unusable by the calls ``Crush'' and ``Shatter''. In order to be repaired and used again, somebody with the Engineering skill must repair it for 1 armour hit (see \textit{pp. 94})

\textit{Tier 1:}

Ability to use a physical shield.

\textit{Tier 2:}

\begin{itemize}
\item Ability to use a physical shield.

\item For the purposes of using the ``Grapple'' call, you count as 2 people and you also require an additional 1 person to be grappled, if you are resisting. When using this skill in these ways please declare ``Bulging Biceps 2'' as you perform the action.

\end{itemize}
\textit{Tier 3:}

\begin{itemize}
\item Ability to use a physical shield.

\item For the purposes of using the ``Grapple'' call, you count as 3 people and you also require an additional 2 people to be grappled, if you are resisting. When using this skill in these ways please declare ``Bulging Biceps 3'' as you perform the action.

\item You may roleplay picking up and throwing an unresisting person with some effort with both hands a short distance (10 feet max).

\item Allows the use of the Knockdown call (see \textit{pp.}

\end{itemize}
\textit{29}) with any melee weapon \textbf{once per encounter}.

Allows you to use a heavy weapon one-handed for 10 seconds \textbf{once per encounter}. This does allow dual wielding of heavy weapons in this period. This ability does not allow for a character to grapple more than one target.

\textbf{Using Knockdown and Bulging Biceps 3}: Katie plays a melee-based Trooper with Bulging Biceps at Tier 3. As her character, Iliana, she wields a one-handed sword. She is fighting in the front lines against the tech-dead when she sees her commander taken down behind enemy lines. She rushes forward, mustering her comrades to join her, and she heads the charge, calling ``Knockdown!'' as she strikes an enemy. It is knocked down, away from the body of her commander, and as her comrades fight the rest she moves to drag away the unconscious commander. ``Bulging Biceps 3!'' Katie calls, as Iliana drags her commander away from danger towards the waiting Medics.

\subsection{Courage}

\textit{It takes something a bit more than nerve to stand and face the terrifying situations of the battlefield, but somebody has to be the one to stand whilst others flee.}

This skill allows the character to resist the effects of Fear (see \textit{pp. 28}) a limited number of times per encounter so that it has no effect. It also reduces the effect of fear if they choose not to resist it. The character must resist with appropriate roleplay, such as roaring at the enemy, taunting them, and making it clear they are not frightened.

\textit{Tier 1:}

\begin{itemize}
\item Ability to resist Fear \textbf{once per encounter} with the Courage call.

\item Only required to flee for \textbf{20 seconds} when affected by Fear.

\end{itemize}
\textit{Tier 2:}

\begin{itemize}
\item Ability to resist Fear \textbf{twice per encounter} with the Courage call.

\item Only required to flee for \textbf{15 seconds} when affected by Fear.

\end{itemize}
\textit{Tier 3:}

\begin{itemize}
\item Ability to resist Fear \textbf{three times per encounter} with the Courage call.

\item Only required to flee for \textbf{10 seconds} when affected by Fear.

\item Ability to resist Terror (see \textit{pp. 32}) \textbf{once per day} with the call ``Courage: Resist Terror''. If you choose

not to resist the Terror you must still flee for the full \textbf{30 seconds}.

\textbf{Courage}: Alex is playing Gregorovitch, a Rossii Heavy with the Courage skill at Tier 1. An enemy adept calls ``Fear'' at Gregorovitch, who shouts ``Ha! It'll take more than you to make me run! Courage!'' His next retort is a rain of heavy gunfire.

\end{itemize}

\textbf{Courage}: Krakkt, a fearsome mascen Big'un in chainmail with the Courage skill at Tier 2, faces down a creature that is innately scary. When the player, Rebecca, hears it declare ``Fear'' she breathes in deeply, stares the creature directly in the eyes and roars, charging it with her shield in front. ``Courage!'' she cries, as she closes in on the beast.

\subsection{Dodge}

\textit{Rule 1 of Green Cloaks basic training: don't get shot.}

This skill represents above-average agility and the ability to avoid being shot or hit. This skill takes the form of slowly regenerating global hit points (see \textit{pp. 23}).

You may not take the Extra Hits skill (see \textit{pp. 72}) if you have this skill. Global hits granted by this skill have the following \textbf{additional} attributes:

\begin{itemize}
\item The hits cannot be used for any attack from behind.

\item The hits cannot be used for any attack from a hidden ranged attacker.

\item The hits may only be granted if you can move your body of your own accord (not when you are grappled, unconscious, paralysed or controlled).

\end{itemize}
\textit{Tier 1:}

The character has \textbf{1 global hit per encounter}.

\textit{Tier 2:}

The character has \textbf{2 global hits per encounter}.

\textit{Tier 3:}

\textbf{Using Tier 3 Dodge to negate all damage and calls}: Zim is a Scout with Tier 3 Dodge. He walks into his camp and is accosted by an unknown Trooper with an extremely large gun. The unknown Trooper shoots Zim in the torso and makes the call ```Through, Crush!''. Zim, however, uses his Tier 3 Dodge ability and negates the damage and the calls. He calls ``Dodge!'' and roleplays ducking down to indicate he has done this. Zim then kills the unknown Trooper while he is reloading his gun.

\begin{itemize}
\item The character has \textbf{3 global hits per encounter}.

\item Ability to negate all damage and calls from any one physical attack from a weapon such as a gun or melee weapon by expending all global hits granted by the Dodge skill at once, and saying ``Dodge'' to the attacker. The attack must have been made with a melee weapon or dart, not a blast from a grenade or an Omega ability. The Marksman call (see \textit{pp. 29}) \textbf{can} be dodged as it is meant to represent a physical attack, but the character must have clearly seen the person who fired at them when the attack was made. Please use suitably dramaticrole play when you use this ability.

\end{itemize}
\subsection{Engineering}

\textit{An invaluable skill on the battlefield: the ability to make and maintain weapons, armour and equipment for the troops. And when you're not fixing holes in someone's energy shield, there's always time to invent another gadget that will take the enemy by surprise{\dots} and hopefully not blow your friends up.}

The Engineering skill gives the ability to analyse, craft, repair, and hack different levels of technology (see \textit{pp. 90-94}). Only the Engineer, mascen Little'un Tinkerers and vrede classes can undertake Engineering research (see \textit{pp. 91-92}).

If you wish to attempt analysis of an item of tech, please alert a referee before starting. Results and time required will depend on roleplay, your Engineering skill tier and other IC circumstances.

Unless you are of the Engineer class (in which case you will be issued this as standard), in order to undertake any hacking you must craft a multispectrum systems analyser (also called a ``multispec''). This takes the form of a small portable computer which assists with a variety of engineering tasks. See \textit{pp. 111} in \textit{Appendix E. Crafting and Components Lists: Engineering} for the components and EWP cost to craft a multispec.

All actions (analysing, crafting, repairing, and hacking) taken using this skill require appropriate roleplay and physical representations. Please note that although physical representations of tools are not required to be LARP-safe, you must refrain from using any sharp or pointed objects for safety reasons.

Anybody with the Engineering skill can be led by a character of the Engineer class in a team of up to three people when crafting, repairing or hacking. Each assistant to the Engineer grants a 20-second reduction to the time it takes to repair armour or items when using the Engineering skill. The Engineering tier of the leading Engineer is used as the base. The Engineering tier of the assistants does not matter when using this ability. The minimum time it takes to use the Repair call with the Engineering skill is 30 seconds, taking into account all modifiers. If they are being led by a character of the Engineer class, anybody

with the Engineering skill can undertake engineering research (even though they are unable to do so by themselves).

More details on analysing, crafting and hacking can be found in the \textit{Chapter 15: Engineering} and \textit{Appendix}

\textit{E. Crafting and Components Lists: Engineering}.

More details on repairing armour can be found on \textit{pp. 94}.

\textit{Tier 1:}

\begin{itemize}
\item Ability to analyse, craft, repair, and hack up to \textbf{standard-tier} tech items.

\item Ability to repair armour faster. Armour is repaired at the rate of 1 hit to all locations per \textbf{100 seconds} of appropriate roleplay. See below for the times required to repair armour using the Engineering skill.

\end{itemize}
\textit{Tier 2:}

\begin{itemize}
\item Ability to analyse, craft, repair, and hack up to \textbf{advanced-tier} tech items.

\item Ability to repair armour faster. Armour is repaired at the rate of 1 hit to all locations per \textbf{80 seconds} of appropriate roleplay.

\end{itemize}
\textit{Tier 3:}

\begin{itemize}
\item Ability to analyse, craft, repair, and hack up to \textbf{expert-tier} tech items.

\item Ability to repair armour faster. Armour is repaired at the rate of 1 hit to all locations per \textbf{60 seconds} of appropriate roleplay.

\end{itemize}

\begin{table}[H]
\begin{tabular}{|l|l|l|l|} \hline 
\multirow{1}{*}{}& \multicolumn{3}{|l|}{Armour repair times using the Engineering skill (in seconds)} \\
\cline{2-2}\cline{3-3}\cline{4-4} & Tier 1 & Tier 2 & Tier 3 \\
 \hline Alone & 100 & 80 & 60 \\
 \hline 1 assistant & 80 & 60 & 40 \\
 \hline 2 assistants & 60 & 40 & 30 \\
 \hline \end{tabular}

\end{table}

\subsection{Explosives}

\textit{You've been through all the training, you've used these things before, so what's the worst that could happen? But to be on the safe side, you might want to get everybody else a safe distance away before attempting to cut the right wire{\dots}}

The Explosives skill allows the use of the many types of explosive devices found in game. More complex and dangerous explosives require a higher level of experience and training. It also allows the setting and disarming of explosive traps. This skill usually requires the presence of a referee.

More details on explosives can be found on \textit{pp. 86}. Some of the explosives available are listed in \textit{Appendix}

\textit{Crafting and Components Lists: Engineering}.

\textit{Tier 1:}

Ability to use and disarm \textbf{standard-tier} explosives.

\textit{Tier 2:}

Ability to use and disarm \textbf{advanced-tier} explosives.

\textit{Tier 3:}

Ability to use and disarm \textbf{expert-tier} explosives.

\subsection{Extra Hits}

\textit{Maybe it's all the Rossii knife-fights you've been in, or maybe you've taken so many bullets that your skin is now just one big scar, or perhaps you're just naturally more resilient; whatever it is, you can withstand more damage than most people before dropping{\dots} and in the coming war you're going to need it.}

The character is physically tougher than most, and can take more punishment before succumbing to their wounds.

You may not take the Dodge skill (see \textit{pp. 70}) if you have taken this skill.

\textit{Tier 1:}

The character has \textbf{1 extra body hit} per location.

\textit{Tier 2:}

The character has \textbf{2 extra body hits} per location.

\textit{Tier 3:}

The character has \textbf{3 extra body hits} per location.

\subsection{Focus}

\textit{There is always room for improvement, and power is no exception.}

This skill can only be taken by the Adept, myr'na Healer and mascen Little'un Shaman classes. This skill grants the character additional focus points (see \textit{pp. 103}) per day.

\textit{Tier 1:}

Grants an additional \textbf{3 focus points} per day.

\textit{Tier 2:}

Grants an additional \textbf{6 focus points} per day.

\textit{Tier 3:}

Grants an additional \textbf{12 focus points} per day.

\subsection{Heavy Weapons}

\textbf{\textit{Heavy weapon}}

A foam dart gun that can fire darts without any manual cocking between shots.

No limit on magazine size.

The Heavy Weapons skill is required to use this weapon.

Heavy weapons must be held in both hands whilst being fired (this property can be modified briefly with use of the Bulging Biceps skill at Tier 3. See \textit{pp. 68-69}).

While firing a heavy weapon the character may not move from their position (this property can be modified through use of the Heavy Weapons skill and/or the Heavy Weapons Specialist class attribute, ``Proper Training''. See \textit{pp. 56}).

\textit{Sometimes there's nothing quite so effective - or satisfying - as unloading a full drum of bullets into your enemy's chest.}

The character is trained in the use of heavy weapons, wielding huge amounts of destructive power. Improved training allows the character to utilise these weapons to their full effect.

\textit{Tier 1:}

\begin{itemize}
\item Ability to use heavy weapons.

\item The character \textbf{may not move} while firing a heavy weapon.

\end{itemize}
\textit{Tier 2:}

\begin{itemize}
\item Ability to use heavy weapons.

\item The character may move at a \textbf{slow walking pace} while firing a heavy weapon.

\end{itemize}
\textit{Tier 3:}

\begin{itemize}
\item Ability to use heavy weapons.

\item The character may move at a \textbf{normal walking pace} while firing a heavy weapon.

\end{itemize}
\subsection{Marksman}

\textit{Rule 2 of Green Cloaks basic training: shoot them in the head. Preferably from a vast distance away.}

The rifle skill is required to use this skill. A referee should be present during use of this skill if at all possible. The referee will decide success based on quality of roleplay and other factors, such as hazards that directly threaten the character.

Using specialised ammunition, a marksman may eliminate threats from a distance or use well-placed shots to cripple limbs. One marksman round (see \textit{pp. 87}) must be expended to use any call granted by this ability. This may be modified for uses of the Cripple call by the Sniper class attribute ``Incapacitate the Target'' (see \textit{pp. 59}). A physical round does not need to be fired or hit the target in order for the calls granted by this ability to be used.

All calls used with the Marksman skill have a limited range of 300 feet. Aiming must be uninterrupted for use of this skill: if you break line of sight with the target for longer than 3 seconds then you must start your skill count again.

Lining up a shot for Marksman or Cripple will not break stealth (see \textit{pp. 80-81}), however making the call itself will.

The time it takes to use this skill are reduced for a character of the Sniper class (see \textit{pp. 59}). However, the minimum time it takes to use the Cripple call is 25 seconds, and the minimum time to use the Marksman call is 30 seconds, taking into account all modifiers.

\textit{Tier 1:}

\begin{itemize}
\item Ability to use the Marksman call (see \textit{pp. 29}) on a target after uninterrupted aiming for \textbf{45 seconds}.

\item Ability to use the Cripple call (see \textit{pp. 27}) on a target after uninterrupted aiming for \textbf{40 seconds}.

\end{itemize}
\textit{Tier 2:}

\begin{itemize}
\item Ability to use the Marksman call on a target after uninterrupted aiming for \textbf{40 seconds}.

\item Ability to use the Cripple call on a target after uninterrupted aiming for \textbf{35 seconds}.

\end{itemize}
\textit{Tier 3:}

\begin{itemize}
\item Ability to use the Marksman call on a target after uninterrupted aiming for \textbf{35 seconds}.

\item Ability to use the Cripple call on a target after uninterrupted aiming for \textbf{30 seconds}.

\end{itemize}

\begin{table}[H]
\begin{tabular}{|l|l|l|l|} \hline 
\multirow{1}{*}{}& \multicolumn{3}{|l|}{Aiming times for Cripple and Marksman call using the Marksman skill (in seconds)} \\
\cline{2-2}\cline{3-3}\cline{4-4} & Tier 1 & Tier 2 & Tier 3 \\
 \hline Cripple call & 40 & 35 & 30 \\
 \hline Cripple call + Sniper class & 35 & 30 & 25 \\
 \hline Marksman call & 45 & 40 & 35 \\
 \hline Marksman call + Sniper class & 40 & 35 & 30 \\
 \hline \end{tabular}

\end{table}

\subsection{Medicae}

\textit{When the bullets start flying someone has to be there to patch up the holes. It's not a glamorous job, but somebody's gotta do it!}

The Medicae skill allows a character to tend to the wounds of another character. This takes the form of either stabilising the character to prevent bleeding out, or restoring body hits to an already stable, but injured, character (see \textit{Chapter 7: Dying, Healing and Stabilisation}).

All actions taken by this skill require appropriate roleplay and physical representations. Please note that although physical representations of tools are not required to be LARP-safe, please refrain from using any sharp or pointed objects for safety reasons.

Using this skill to make the Heal call requires both the target and healer to remain stationary for the duration: if either of them moves from their position the skill count is reset for the hit that is currently being restored. However, the target and healer may move during the use of the Stabilise call.

Means exist to reduce the time it takes to complete uses of this skill. However, the minimum time it takes to use the Stabilise call is 10 seconds, and the to use the Heal call 30 seconds, taking into account all modifiers.

Those with the Medicae skill can be led by those of the Medic class in a team. Each assistant to a Medic grants a 5-second reduction to the time it takes to use the Stabilise call, and a 20-second reduction to the time it takes to use the Heal call. The Medicae tier of the leading Medic is used as the base for the time it takes to stabilise or heal. The Medicae tier of the assistants does not matter when using this ability.

\begin{table}[H]
\begin{tabular}{|l|l|l|l|} \hline 
\multirow{1}{*}{}& Tier 1 & Tier 2 & Tier 3 \\
\cline{2-2}\cline{3-3}\cline{4-4} & \multicolumn{3}{|l|}{Stabilisation times using Medicae skill (in seconds)} \\
 \hline Alone & 30 & 25 & 20 \\
 \hline 1 assistant & 25 & 20 & 15 \\
 \hline 2 assistants & 20 & 15 & 10 \\
 \hline  & \multicolumn{3}{|l|}{Heal times using Medicae skill (in seconds)} \\
 \hline Alone & 100 & 80 & 60 \\
 \hline 1 assistant & 80 & 60 & 40 \\
 \hline 2 assistants & 60 & 40 & 30 \\
 \hline \end{tabular}

\end{table}

\textit{Tier 1:}

\begin{itemize}
\item Ability to use the Stabilise call (see \textit{pp. 31}) on a dying target after \textbf{30 seconds} of appropriate Medicae roleplay.

\item Ability to use the Heal call (see \textit{pp. 28-29}) to restore 1 body hit to every location on a target after \textbf{100 seconds} of appropriate Medicae roleplay.

\end{itemize}
\textit{Tier 2:}

\begin{itemize}
\item Ability to use the Stabilise call on a dying target after \textbf{25 seconds} of appropriate Medicae roleplay.

\item Ability to use the Heal call to restore 1 body hit to every location on a target after \textbf{80 seconds} of appropriate Medicae roleplay.

\end{itemize}
\textit{Tier 3:}

\begin{itemize}
\item Ability to use the Stabilise call on a dying target after \textbf{20 seconds} of appropriate Medicae roleplay.

\item Ability to use the Heal call to restore 1 body hit to every location on a target after \textbf{60 seconds} of appropriate Medicae roleplay.

\end{itemize}
\subsection{Melee Weapons}

\textit{Whether a spear, sword or axe, some people still feel that getting up close and personal is the way to go.}

The character is trained in the use of melee weapons. Improved training allows them to use these weapons to their full effect.

You cannot use any call granted by this skill more than once per encounter, even if you are wielding more than one one-handed or two-handed melee weapon. The only exception to this is if you are of the Trooper or myr'na Warrior class, and using the Canny class attribute (see \textit{pp. 59} and \textit{pp. 63}).

\textit{Tier 1:}

\textbf{\textit{One-handed Weapon}}

A melee weapon over 18 inches and no longer than 42 inches in length.

The Melee Weapons skill is required to use one-handed weapons.

This weapon can be held in one or two hands while being used.

You may use one one-handed weapon in each hand at the same time.

\textbf{\textit{Two-Handed Weapon}}

A melee weapon over 42 inches and no longer than 84 inches in length.

The Melee Weapons skill is required to use two-handed weapons.

Two-handed weapons must be held in both hands in order to parry or attack.

Ability to use one-handed and two-handed melee weapons.

\textit{Tier 2:}

\begin{itemize}
\item Ability to use one-handed and two-handed melee weapons.

\item Ability to use \textbf{\textit{either}} the Through call (see \textit{pp. 32}) with a one- handed melee weapon, \textbf{\textit{or}} the Knockdown call (see \textit{pp. 29}) with a two-handed melee weapon, \textbf{once per encounter}.

\end{itemize}
\textit{Tier 3:}

\begin{itemize}
\item Ability to use of one and two handed melee weapons.

\item Ability to use \textbf{\textit{either}} the Through or Disarm (see \textit{pp. 27}) calls with a one-handed melee weapon, or the Knockdown or Cripple (see \textit{pp. 27}) calls with a two-handed melee weapon, \textbf{twice per encounter}.

\end{itemize}
\subsection{Omega Body Attunement}

\textit{Why do the Green Cloaks still employ butchers when the Omega can offer such subtle ways of healing?}

This skill can only be taken by the Adept, myr'na Healer and mascen Little'un Shaman classes.

Abilities granted by this skill cost focus points to activate (see \textit{pp. 103}), and their costs are listed in the description of each ability, below.

To use an ability in this skill set you must spend at least \textbf{3 seconds} roleplaying and verbally activating the ability before indicating a target and announcing the effect. You must mention the power of the Omega and the Attunement, body, in your activation; for example, ``By the power of the Omega I cleanse these wounds and heal your body''. After activating an ability granted by this skill you must state ``Omega'', followed by the focus point cost and then the call effect eg, ``Omega 1 stabilise''.

\textit{Tier 1:}

\begin{itemize}
\item Ability to use the Stabilise call (see \textit{pp. 31}) on a dying target instantly on touch, for a cost of \textbf{1 focus point}.

\item Ability to use the Heal call (see \textit{pp. 28-29}) to restore 1 body hit to 1 location on a target on touch after

\textbf{10 seconds} of chanting, for a cost of \textbf{1 focus point}.

\end{itemize}
\textit{Tier 2:}

\begin{itemize}
\item Ability to use the Stabilise call on a dying target instantly on touch, for a cost of \textbf{1 focus point}.

\item bility to use the Heal call to restore 1 body hit to 1 location on a target on touch after 10 seconds of chanting for a cost of \textbf{1 focus point}.

\item Ability to use the Heal call to restore 4 body hits to damaged locations of the healer's choice on a target on touch after \textbf{10 seconds} of chanting, for a cost of \textbf{3 focus points}.

\end{itemize}
\textit{Tier 3:}

Ability to use the Stabilise call on a dying target instantly on touch, for a cost of \textbf{1 focus point}.

\begin{itemize}
\item bility to use the Heal call to restore 1 body hit to 1 location on a target on touch after 10 seconds of chanting for a cost of \textbf{1 focus point}.

\item Ability to use the Heal call to restore 4 body hits to damaged locations of the healer's choice on a target on touch after \textbf{10 seconds} of chanting, for a cost of \textbf{3 focus points}.

\item Ability to affect a target with the Paralyse call (see \textit{pp. 29}) on touch, for a cost of \textbf{3 focus points}.

\end{itemize}
\subsection{Omega Energy Attunement}

\textit{Everything burns: from the oldest stars to the newest human soul, and that energy is yours to harness and direct.}

This skill can only be taken by the Adept, myr'na Healer and mascen Little'un Shaman classes.

Abilities granted by this skill cost focus points to activate (see \textit{pp. 103}), and their costs are listed in the description of each ability below.

To use an ability in this skill set you must spend at least \textbf{3 seconds} roleplaying and verbally activating the ability before indicating a target and announcing the effect. You must mention the power of the Omega and the Attunement, energy, in your activation; for example, ``By the power of the Omega I focus energy to strike you.'' After activating an ability granted by this skill you must state ``Omega'', followed by the focus point cost and then the call effect eg, ``Omega 1 Bolt''.

\textit{Tier 1:}

\begin{itemize}
\item Ability to affect a target within a 30-foot range with the Repel call (see \textit{pp. 30}), for a cost of \textbf{1 focus point}.

\item Ability to affect a target within a 30-foot range with the Bolt call (see \textit{pp. 27}), for a cost of \textbf{1 focus point}.

\end{itemize}
\textit{Tier 2:}

\begin{itemize}
\item Ability to affect a target within a 30-foot range with the Repel call, for a cost of \textbf{1 focus point}.

\item Ability to affect a target within a 30-foot range with the Bolt call, for a cost of \textbf{1 focus point}.

\item Ability to affect a target with the Cripple call (see \textit{pp. 27}) on touch, for a cost of \textbf{2 focus points}. It is permissable to activate this ability, then delay delivering it for up to 10 seconds.

\end{itemize}
\textit{Tier 3:}

\begin{itemize}
\item Ability to affect a target within a 30-foot range with the Repel call, for a cost of \textbf{1 focus point}.

\item Ability to affect a target within a 30-foot range with the Bolt call, for a cost of \textbf{1 focus point}.

\item Ability to affect a target with the Cripple call on touch, for a cost of \textbf{2 focus points}. It is permissable to activate this ability, then delay delivering it for up to 10 seconds.

\item Ability to affect a target within a 30-foot range with the Blast call (see \textit{pp. 26}), for a cost of \textbf{3 focus points}.

\end{itemize}
\subsection{Omega Mind Attunement}

\textit{Control somebody's perception, and you have them in your grasp.}

This skill can only be taken by the Adept, myr'na Healer and mascen Little'un Shaman classes.

Abilities granted by this skill cost focus points to activate (see \textit{pp. 103}), and their costs are listed in the description of each ability below.

To use an ability in this skill set you must spend at least \textbf{3 seconds} roleplaying and verbally activating the ability before indicating a target and announcing the effect. You must mention the power of the Omega and the Attunement, mind, in your activation; for example, ``By the power of the Omega I cloud your mind and confuse your senses.'' After activating an ability granted by this skill you must state ``Omega'', followed by the focus point cost and then the call effect eg, ``Omega 1 Stun''.

\textit{Tier 1:}

Ability to affect a target within a 30-foot range with the Stun call (see \textit{pp. 31}), for a cost of \textbf{1 focus point}.

\textit{Tier 2:}

\begin{itemize}
\item Ability to affect a target within a 30-foot range with the Stun call, for a cost of \textbf{1 focus point}.

\item Ability to affect a target with the Befriend call (see \textit{pp. 26}) on touch, for a cost of \textbf{2 focus points}. It is permissable to activate this ability, then delay delivering it for up to 10 seconds.

\end{itemize}
\textit{Tier 3:}

\begin{itemize}
\item Ability to affect a target within a 30-foot range with the Stun call, for a cost of \textbf{1 focus point}.

\item Ability to affect a target with the Befriend call on touch, for a cost of \textbf{2 focus points}. It is permissable to activate this ability, then delay delivering it for up to 10 seconds.

\item Ability to affect a target within a 30-foot range with the Fear call (see \textit{pp. 28}), for a cost of \textbf{2 focus points}.

\item Ability to affect a target within a 30-foot range with the Hallucinate call (see \textit{pp. 28}), for a cost of \textbf{4 focus points}.

\end{itemize}
\subsection{Omega Protection Attunement}

\textit{I understand all pieces of the Omega: when they threaten you I can turn them in on themselves, showing a mirror into the abyss.}

This skill can only be taken by the Adept, myr'na Healer and mascen Little'un Shaman classes.

Abilities granted by this skill cost focus points to activate (see \textit{pp. 103}), and their costs are listed in the description of each ability below.

To use an ability in this skill set you must spend at least \textbf{3 seconds} roleplaying and verbally activating the ability before indicating a target and announcing the effect. You must mention the power of the Omega and the Attunement, protection, in your activation; for example ``By the power of the Omega I protect my allies from your attack.'' After activating an ability granted by this skill you must state ``Omega'', followed by the focus point cost and then the call effect eg, ``Omega 1 Negate''

\textit{Tier 1:}

Ability to use the Negate call (see \textit{pp. 29}) on an effect within 30 feet that has been caused by another Omega Attunement skill, for the \textbf{focus point cost of the power that has been negated}.

\textit{Tier 2:}

\begin{itemize}
\item Ability to use the Negate call on an effect within 30 feet that has been caused by another Omega Attunement skill, for the \textbf{focus point cost of the power that has been negated}.

\item Ability to use the Reflect call (see \textit{pp. 29}) on an effect targeted at yourself that has been caused by another Omega Attunement skill, for the \textbf{focus point cost of the power reflected plus 2}.

\end{itemize}
\textit{Tier 3:}

\begin{itemize}
\item Ability to use the Negate call on an effect within 30 feet that has been caused by another Omega Attunement skill, for the \textbf{focus point cost of the power that has been negated}.

\item Ability to use the Reflect call (see \textit{pp. 30}) on an effect targeted at yourself that has been caused by another Omega Attunement skill, for the \textbf{focus point cost of the power reflected plus 2}.

\item Ability to affect yourself with the Shield call (see \textit{pp. 30}), for a cost of \textbf{3 focus points}.

\end{itemize}
\subsection{Omega Spirit Attunement}

\textit{The spirit is a vicarious and nebulous thing - but speak the right words and it becomes your plaything.}

This skill can only be taken by the Adept, myr'na Healer and mascen Little'un Shaman classes.

Abilities granted by this skill cost focus points to activate (see \textit{pp. 103}), and their costs are listed in the description of each ability below.

To use an ability in this skill set you must spend at least \textbf{3 seconds} roleplaying and verbally activating the ability before indicating a target and announcing the effect. You must mention the power of the Omega and the Attunement, spirit, in your activation; for example, ``By the power of the Omega I strike at your spirit.'' After activating an ability granted by this skill you must state ``Omega'', followed by the focus point cost and then the call effect eg, ``Omega 1 Knockdown''.

\textit{Tier 1:}

Ability to affect a target within a 30-foot range with the Knockdown call (see \textit{pp. 29}), for a cost of

\textbf{1 focus point}.

\textit{Tier 2:}

\begin{itemize}
\item Ability to affect a target within a 30-foot range with the Knockdown call, for a cost of \textbf{1 focus point}.

\item Ability to affect yourself with the Spirit sense call (see \textit{pp. 31}), for a cost of \textbf{2 focus points}.

\end{itemize}
\textit{Tier 3:}

\begin{itemize}
\item Ability to affect a target within a 30-foot range with the Knockdown call, for a cost of \textbf{1 focus point}.

\item Ability to affect yourself with the Spirit sense call, for a cost of \textbf{2 focus points}.

\item Ability to affect a target within a 30-foot range with the Possession call (see \textit{pp. 30}), for a cost of

3 focus points.

\end{itemize}
\textbf{Pharmacology}

\textit{The natural world is a cornucopia of ingredients, and you've got a lab just waiting to turn them into something useful{\dots}}

Pharmacology gives the ability to craft and analyse different herbs and pharmaceuticals found in-game. If you wish to attempt analysis of an item, alert a referee before starting. Results will depend on roleplay, skill tier and other IC circumstances. When you register as a character with the Pharmacology skill, you will be able to choose \textbf{six ingredients} that your character knows the primary properties of (see \textit{pp. 116}).

Only the Medic and tae'go classes can undertake Pharmacology research (if they have the Pharmacology skill). For more details on this, and on crafting and analysing, see \textit{Chapter 16: Pharmacology}\textit{,} and \textit{Appendix}

\textit{Crafting and Ingredients Lists: Pharmacology}.

\textit{Tier 1:}

Ability to craft and analyse up to \textbf{standard-tier} pharmaceuticals.

\textit{Tier 2:}

Ability to craft and analyse up to \textbf{advanced-tier} pharmaceuticals.

\textit{Tier 3:}

Ability to craft and analyse up to \textbf{expert-tier} pharmaceuticals.

\subsection{Pistol}

\textit{From the sniper's sidearm to the gunslinger's decked-out custom job, pistols are ubiquitous on the battlefield.}

The character is trained in the use of pistols, allowing precise and devastating shots. Improved training allows the character to use these weapons to their full effect.

\textbf{\textit{Pistol}}

A small foam dart gun that can fire between 2 and 6 darts before reloading and requires manual cocking between shots.

The Pistol skill is required to use this weapon.

Maximum magazine size of 6 darts. Pistols can be held in one or both hands while being fired.

Pistols can be used in each hand at the same time.

\textit{Tier 1:}

Ability to use pistols.

\textit{Tier 2:}

\begin{itemize}
\item Ability to use pistols.

\item Ability to use the Through call (see \textit{pp. 32}) \textbf{once per encounter}.

\end{itemize}
\textit{Tier 3:}

\begin{itemize}
\item Ability to use pistols.

\item Ability to use the Through and/or Knockdown call (see \textit{pp. 29})

\textbf{twice per encounter}.

\end{itemize}
\textbf{Rifle}

\textbf{\textit{Rifle}}

A large foam dart gun that requires manual cocking between shots and can use a removable magazine. Magazines are restricted to a maximum of 18 darts.

\textbf{\textit{Or}}

A large foam dart gun that requires manual cocking between shoots and cannot use a removable magazine. It is allowed to hold more than 18 darts. The Rifle skill is required to use this weapon.

Rifles must be held in both hands whilst being fired.

\textit{Rifles are the staple weapon of most armed forces, but only with the proper training is their full potential unlocked. For less modern (or less orthodox) groups, the weapon of choice is the bow.}

The character is trained in the use of rifles and bows. Improved training allows the character to use these weapons to their full effect.

\textit{Tier 1:}

Ability to use rifles and bows. All shots fired with bows call Knockdown (see \textit{pp. 29}).

\textit{Tier 2:}

\begin{itemize}
\item Ability to use rifles and bows. All shots fired with bows call Knockdown.

\item Ability to \textbf{\textit{either}} use the Through (see \textit{pp. 32}) or Stun (see \textit{pp.}

\textbf{\textit{Bow}}

A bow of 30 lbs pull or less.

Must be able to fire LARP-safe arrows.

The Rifle skill is required to use this weapon.

All hits from a bow cause the ``Knockdown'' call.

All bows must be poundage tested and tagged by a referee before use. The user must also attend a safety brief with a referee before use.

\textit{31}) call with a rifle, \textbf{\textit{or}} the Through call with a bow, \textbf{once per encounter}. Note that a bow would still cause the Knockdown effect in addition to this call.

\end{itemize}
\textit{Tier 3:}

\begin{itemize}
\item Ability to use of rifles and bows. In addition, all shots fired with bows call ``Knockdown''.

\item Ability to \textbf{\textit{either}} use the Through, Stun, Knockdown or Disarm (see \textit{pp. 27}) call with a rifle, \textbf{\textit{or}} the Through or Cripple (see \textit{pp.}

\textit{27}) call with a bow, \textbf{twice per encounter}. Note that a bow would still cause the Knockdown effect in addition to these calls.

\end{itemize}
Self-sufficient

\textit{Foraging, scavenging, ``liberating'' or however you want to put it: the ability to find useful things in one's}

\textit{environment is extremely handy.}

This skill allows the character to forage and scavenge for useful items. Higher tiers of this skill increase the number of items you are likely to find, and their rarity. To use this skill please visit GOD at the designated times during time in.

When you go to GOD you will be given the option to go to a low, medium or high risk area. The higher the

risk the more chance you have of finding better items.

\textit{Tier 1:}

20 minutes of roleplay required to forage/scavenge.

\textit{Tier 2:}

\begin{itemize}
\item 20 minutes of roleplay required to forage/scavenge.

\item Increased chance of more items and of items of higher rarity than Tier 1.

\end{itemize}
\textit{Tier 3:}

\begin{itemize}
\item 20 minutes of roleplay required to forage/scavenge.

\item Increased chance of more items and of items of higher rarity than Tier 2.

\end{itemize}
\subsection{Stealth}

\textit{The ability to remain unseen can grant a huge advantage over your enemies. All it takes is one ambush to even the odds.}

This skill allows a character to effectively hide themselves in cover and avoid detection. While in stealth you cannot be seen.

Cover is defined as significant stationary terrain or objects, which cover at least 50\% of your size from at least one side when you are concealed around or in them. When you are inside a structure (building, tent, etc) the inside walls of the structure do not count as cover. Large bushes while you are crouched or small bushes while you are prone, trees, a pile of logs, outside walls and the outside of tents are all valid cover. People, bodies and shields are not valid cover. You must remain in a crouched stance or lower at all times while in stealth, and standing will break your stealth. You cannot move from the place where you entered stealth, and doing so will break your stealth. You can, however, move your upper body, and the higher tiers of the Stealth skill do allow limited movement. See \textit{Appendix I. Stealth Cover Diagrams} for an illustrated guide to what cover allows stealth.

Using any skill or performing any overt action (such as firing your gun) will break your stealth. Note that lining up a shot for the Marksman call is not a skill use, however making the call itself is. Noise caused as a result of movement in stealth cannot be heard. Speaking softly does not break your stealth, however speaking at a normal volume or louder will.

Once stealth has been broken you cannot enter it again for 15 seconds. This time is reduced at higher tiers of the Stealth skill.

Stealth is denoted by holding one finger in the air. For each extra tier of the Stealth skill you have you hold an additional finger in the air.

If the call ``Awareness'' (see \textit{pp. 26}) is made within 30 feet of you and is of a high enough level to locate you in stealth you must clearly state your Stealth tier so the caller can hear you. If you are hidden, but not stealthed, you must say ``0''.

\textit{Tier 1:}

\begin{itemize}
\item Ability to enter Stealth 1. You must hold \textbf{one finger} in the air when using this ability.

\item After leaving stealth you must wait at least \textbf{15 seconds} before entering it again.

\item You \textbf{cannot move} from the location you entered stealth.

\end{itemize}
\textit{Tier 2:}

\begin{itemize}
\item Ability to enter Stealth 2. You must hold \textbf{two fingers} in the air when using this ability.

\item After leaving stealth you must wait at least \textbf{10 seconds} before entering it again.

\item You can move at a \textbf{slow walking pace} (whilst remaining crouched) in Stealth 2 for \textbf{3 seconds}, after which you must remain stationary for \textbf{5 seconds} before you can move again. If you leave cover at any time your stealth will be broken.

\end{itemize}
\textit{Tier 3:}

\begin{itemize}
\item Ability to enter Stealth 3. You must hold \textbf{three fingers} in the air when using this ability.

\item After leaving stealth you must wait at least \textbf{5 seconds} before entering it again.

\item You can move at a \textbf{slow walking pace} (whilst remaining crouched) in Stealth 3 for \textbf{5 seconds}, after which you must remain stationary for \textbf{5 seconds} before you can move again. If you leave cover at any time your stealth will be broken.

\end{itemize}
\chapter{Equipment}

\subsection{Armour}

All armour, apart from non-standard armour, is standard issue, and does not cost anything to purchase in- game. You do not need a skill to be able to wear armour as costume, however to gain its protective benefit either your class or race must enable you to wear it. All armour must be physrepped appropriately, and the physrep must be checked for sharp edges, protruding elements or other dangerous features. A referee may ask you to modify or completely remove any armour which may cause damage to the wearer, other players, or weapons striking its surface. Suggestions for physrepping different types of armour are given in each armour type's description below. For more on hits granted by armour see \textit{pp. 22-23}.

\textit{Heavy armour}

The heaviest commonly available armour, normally bulky and restrictive but providing the best level of protection for the front line.

Grants \textbf{3 armour hits} to the covered location(s).

Usable by: Heavy Weapons Specialist, mascen Big'un, Trooper.

Heavy armour is defined as any metal armour and any armour that consists of rigid plates (these may also be foam or plastic).

Acceptable physreps of this armour include: riot armour, medieval plate, chainmaille, plated motocross armour, real flak armour (or flak covers containing rigid foam plates) and foam armour (made to look like real armour).

\textit{Medium armour}

A mid-range armour for support specialists, those who need to spend some time in the front line to complete a task before extracting.

Grants \textbf{2 armour hits} to the covered location(s).

Usable by: Heavy Weapons Specialist, mascen Big'un, Trooper, Medic, Engineer, mascen Little'un Shaman, mascen Little'un Tinkerer, tae'go, Sniper, Scout and myr'na Warrior.

Acceptable physreps of this armour include: leather armour (but \textbf{not} leather jackets/clothes), flak jacket covers and tactical vests (full vests, not just webbing).

\textit{Light armour}

The lightest armour, generally composed of heavy cloth woven with ballistic weave fibres which can repel limited small-arms fire. Used by those for whom speed and stealth are a primary concern.

Grants \textbf{1 armour hit} to the covered location(s).

Usable by: Heavy Weapons Specialist, mascen Big'un, Trooper, Medic, Engineer, mascen Little'un Shaman, mascen Little'un Tinkerer, tae'go, Sniper, Scout, myr'na Warrior, Adept and myr'na Healer. Acceptable physreps of this armour include: heavy coats, leather jackets and cloaks (green, ideally!).

\textit{Non-Standard Armour}

It is possible to find or craft armour that is different to the three standard types. This can be done in various ways for different effects. The most common form of advanced armour is ``patterned'' armour, a rare armour that grants 1 additional armour hit on top of the hits granted by its standard equivalent. However, it does require maintenance over time, as well as significant time and resources to craft. See \textit{Appendix E. Crafting} \textit{and Components Lists: Engineering} for some of the non-standard armours that can be crafted.

\textbf{Repairing armour hits}

The Repair call restores armour hits to a character. The main method of using this call is the Engineering skill. However, any character may repair armour at a rate of 1 armour hit per location per 180 seconds. Only characters with the Engineering skill can take part in repair led by a character the with the ``Chief Engineer'' class attribute. You cannot repair armour you are currently wearing, however you can repair armour that other people are wearing. It takes some time to repair armour using the Engineering skill, and the armour hits are not restored until the repairer has completed their skill count and made the Repair call (see \textit{pp. 94}).

Physical representation

In order to gain the benefits of armour you must be wearing a physical representation of it. Examples of these can be found in the descriptions above. However, we would prefer people to look the part rather than be restricted by the rules in some cases. In this case you may wear physical representations of armour that appears heavier than that which your class can wear. You will gain no rules benefit from this, and it will only count as the maximum armour your character's class can wear. Note that in the case of lammied armour, wearing a class higher than the maximum your character can wear only grants the hits of the maximum armour your character can wear, without the bonus granted by its special crafting.

\textbf{Wearing a physrep of armour heavier than your class can wear}: Sanjay plays an Adept character, therefore his maximum armour is light. However, he has spent the time to buy and modify a rigid helmet to fit his character. The helmet is technically heavy armour, but Sanjay may use the helmet and only count it as light.

What Armour type is my physical representation?

If you are in doubt about the armour classification of your physical representation please email a picture of

it to a referee before an event. If you are at an event please speak to your regiment referee.

\subsection{Weapons}

All weapons must be weapons checked at the start of every event and before every large battle. This process involves a referee or designated weapons checker checking the weapon visually and through handling to ensure that it is safe for use. All blasters must be weapons checked by \textit{Blastersmiths UK}, who will tag them if they pass the check. Even if a weapon is brand new or purchased on-site it must undergo a weapons check. Weapons checking at the start of an event and before battles will be announced, and there will also be experienced weapons checkers in GOD, and among the referees, if you have any concerns at any time regarding your weapon's safety. Please note that a weapons checker may perform random checks if they notice a weapon that they have concerns about, or if somebody brings a particular item to their attention. All weapons checkers undertake training, which is recorded by the system, to ensure the highest level of weapon safety possible.

The below weapons are standard issue, and do not cost anything to purchase in-game. If you have the prerequisite skill for using them, you can do so with an appropriate physrep (see \textit{Appendix G. Dart Gun} \textit{Categories}).

The weapon categories below are designed to accommodate as much of the available range of dart guns as possible while maintaining balance in the game. This may mean that you are unable to use a certain blaster in a way that you wish. Please remember in these cases that the rules team has gone to a lot of effort to make sure as many blasters as possible can be used in this system. But it must be remembered that these are just props designed to serve a purpose, and some may be better for the purposes of the game than others.

\textit{BOOMco} blasters are currently banned due to their incompatibility with standard dart ammunition and high projectile speed.

Given the large scope of the modifications that can be made to these blasters that can be made, we recommend you consult us before you make any changes that you feel might put a gun outside these categories. If you are unsure what category your gun falls into, please see \textit{Appendix G. Dart Gun} \textit{Categories} for a full list. Updates to this list when new guns are released can be found on the \textit{Green Cloaks} website or in GOD at events.

Comedy weapons (LARP-safe items that cannot realistically be used as a weapon given their physical representation) are restricted to regimental camps and do not cause damage.

\subsection{Melee Weapons}

\textit{Dagger}

A melee weapon between 6 and 18 inches in length.

Anyone can use this weapon, and no skill is required to do so. Daggers can be held in one or two hands while being used.

You may use one dagger in each hand at the same time.

Calls granted from the Melee Weapons skill cannot be used with a dagger.

\textit{One-handed Weapon}

A melee weapon over 18 inches and no longer than 42 inches in length. The Melee Weapons skill is required to use one-handed weapons.

This weapon can be held in one or two hands while being used.

You may use one one-handed weapon in each hand at the same time.

\textit{Two-Handed Weapon}

A melee weapon over 42 inches and no longer than 84 inches in length. The Melee Weapons skill is required to use two-handed weapons.

Two-handed weapons must be held in both hands in order to parry or attack.

\textit{Stab-safe weapons}

Stab-safe weapons come in various sizes and their size and application dictates how they may be used (see below).

The Melee Weapons skill is required to use stab-safe weapons.

All stab-safe weapons must resemble stabbing weapons, eg short spears.

If you wish to use a stab-safe weapon you must pass a competency test: you will need to register at GOD when you sign in for an event, where you will be given details on this.

You cannot dual-wield stab-safe weapons regardless of their size.

\textbf{Short stab-safe weapons}

Short stab-safe weapons must be 42 inches long.

For the purpose of the Melee Weapons skill, they count as a one-handed weapon. You may only stab with a short stab-safe weapon.

You may not slash or parry with a short stab-safe weapon.

Long stab-safe weapons

Long stab-safe weapons are over 42 inches in length and no longer than 84 inches. You can choose to use a long stab-safe weapon in one or two hands.

When used in one hand they must be held in the center of the weapon.

When used in one hand for the purpose of the Melee Weapons skill, they count as a one-handed weapon.

When used in one hand you may only stab.

When used in one hand you may not parry or slash.

When used in two hands for the purpose of the Melee Weapons skill, they count as a two-handed weapon.

When used in two hands they must be held with one hand in the center and one hand near the base of the weapon.

When used in two hands a firm grip must be maintained when using the weapon: you may not slide

the weapon through one hand like a pool/snooker cue. When used in two hands you may parry, slash and stab.

\subsection{Ranged Weapons}

Weapons that discharge darts from multiple barrels at the same time cannot deal more than 1 point of damage to any single location regardless of how many darts hit that location with one volley. If darts hit separate locations this will cause damage to all locations struck.

\textit{Single-shot pistol}

A small foam dart pistol that can only fire 1 dart before reloading. Anyone can use this weapon, and no skill is required to do so.

Single-shot pistols can be held in one or both hands while being fired. Single-shot pistols can be used in each hand at the same time.

Calls granted from the Pistol skill cannot be used with a single-shot pistol.

\textit{Throwing weapons}

A coreless LARP-safe throwing weapon that is no more than 18 inches in any dimension. Anyone can use this weapon, and no skill is required to do so.

You may not stab with this weapon type, despite it having no core. A maximum of 18 inches in any dimension.

\textit{Pistol}

A small foam dart gun that can fire between 2 and 6 darts before reloading and requires manual cocking between shots.

Maximum magazine size of 6 darts.

The Pistol skill is required to use this weapon.

Pistols can be held in one or both hands while being fired. Pistols can be used in each hand at the same time.

\textit{Rifle}

A large foam dart gun that requires manual cocking between shots and can use a removable magazine. Magazines are restricted to a maximum of 18 darts.

\textit{or}

A large foam dart gun that requires manual cocking between shots and cannot use a removable magazine. It is allowed to hold more than 18 darts.

The Rifle skill is required to use this weapon.

Rifles must be held in both hands whilst being fired.

\textit{Heavy weapon}

A foam dart gun that can fire darts without any manual cocking between shots. No limit on magazine size.

The Heavy Weapons skill is required to use this weapon.

Heavy weapons must be held in both hands whilst being fired (this can be modified through the use of the Bulging Biceps skill at Tier 3. See \textit{pp. 68}.).

While firing a heavy weapon the character may not move from their position (this property can be modified through use of the Heavy Weapons skill and/or the Heavy Weapons specialist class attribute, ``Proper Training''. See \textit{pp. 56}.).

\textit{Bow}

A bow of 30lbs pull or less.

Must be able to fire LARP-safe arrows.

The Rifle skill is required to use this weapon.

All hits from a bow cause the Knockdown call (see \textit{pp. 29})

All bows must be poundage tested and tagged by a referee before use. The user must also attend a safety brief with a referee before use.

\subsection{Shields}

A character must have the Bulging Biceps skill to use a shield. A shield blocks all incoming physical attacks from ranged and melee weapons, unless otherwise stated in the call (if one accompanies the attack. \textit{See} \textit{Chapter 8: Calls}.). The maximum size of a shield is 60 inches in height by 36 inches width. For the shield to grant defence, it must be held in one or two hands by the player; if it is slung or carried on the back it does not grant any protection, and hits will be taken as normal to the location it covers.

If you come under significant fire from ranged weapons while using a shield to block the darts, you must roleplay \textbf{bracing against the force of the fire} until it has decreased to a less significant level. You are still able to move while bracing, but only at a slow walking pace.

If you use your shield to block a physical attack that is accompanied by a call that can be blocked (\textit{see} \textit{Chapter 8: Calls}), you must roleplay as if you had just blocked a particularly strong shot or challenging blow.

A shield may be rendered unusable by the Crush (see \textit{pp. 27}) and Shatter (see \textit{pp. 30}) calls. In order to be repaired and used again, somebody must repair it for 1 armour hit (see \textit{pp. 94}).

\subsection{Ammunition}

Please consult \textit{Appendix J. Blastersmiths UK Blaster Safety Guidelines} for information on the allowed types of ammunition at \textit{Green Cloaks}. Some brands of darts are banned for safety reasons. Any banned ammo types will be removed by the referees when found and placed safely at GOD; it can then be collected by the owner once the event has finished.

All darts must be safety-checked to make sure they are not damaged or of a banned type. Please take all darts you intend to use into the play area at the start of the event. You may not store darts in the OC area and go back for them later as they will need to be safety checked.

Keeping track of your own darts at \textit{Green Cloaks} is an impossible task. It is recommended you do not bother to mark them. At the end of the event you can visit GOD and reclaim a sensible amount, but expectto lose around 25-50\% of any darts taken to an event.

Please note that regardless of dart type, all darts cause standard damage.

Please pick up any darts you see on the ground if you have the time. However, you may not refire or reload them into a weapon: instead please give them to a referee or weapons checker who will check the darts for damage before returning them to you if they are safe.

If ammo supply is limited for IC reasons, you may need to wait for (or seek out) an IC opportunity to resupply.

\subsection{Explosives}

These items can be created by anybody with the Engineering skill in-game (see \textit{Chapter 15: Engineering}). Although they are created with the Engineering skill, a character must have the Explosives skill in order to use them. They can take the form of various items, and the physreps should be appropriate for the item.

Their effects and possible requirements are indicated on the attached lammies or single-use printed paper (which must be ripped when the explosive is used). Many of the uses of explosives require the presence of a referee. For some of the explosives that may be crafted using the Engineering skill see \textit{Appendix E.} \textit{Crafting and Components Lists: Engineering}.

\subsection{Currency}

All players will be given \textbf{an amount of currency for every event they attend}, to reflect the fact that their character would receive a regular wage. The currency used in the Terran Sovereignty is the crown, the mark and the credit, in descending order.

5 credits = 1 mark

5 marks = 1 crown

Currency is physrepped by paper notes supplied by the Game Team. The costs of goods and services

change frequently to reflect supply and demand, and the changing face of the war.

\subsection{Miscellaneous}

\textit{Dermograft patches}

Also called ``dermos'', this is a one-use medical patch that, when applied to the skin, instantly stabilises a dying character. It must be applied to the skin, and the Stabilise call made by the person who applied it (see \textit{pp. 31}). These are phyrsrepped by printed pieces of paper that must be ripped on use.

All characters are issued with \textbf{1 dermograft patch per event}, unless they are a Medic in which case \textbf{3} are issued. They may also be crafted using the Pharmacology skill. See \textit{pp. 116} for the PWP and ingredients required for this.

\textit{Marksman rounds}

These are single-use rifle rounds that have been specially engineered for accurate, long-distance, high- penetration shots. They are required for the calls made using the Marksman skill (see \textit{pp. 73}). These are physrepped by printed pieces of paper that must be ripped on use. Snipers are issued with \textbf{5 marksman rounds per event}. They may also be crafted using the Engineering skill. See \textit{pp. 111} for th EWP and components required for this.

\textit{Pharmaceuticals}

These are single-use items that are manufactured by anybody with the Pharmacology skill (see \textit{Chapter} \textit{16: Pharmacology}). They can take the form of various items, but most of them come in consumable form or injection. The physrep should be appropriate for the item. Please note that any injectable pharmaceuticals must \textbf{not} be physrepped by needles or sharp objects of any kind. Their effects are indicated on an accompanying piece of paper that must be ripped on use.

\textit{Tech items}

These items can be created by anybody with the Engineering skill in-game (see \textit{Chapter 15: Engineering}). They can take the form of various items, and the physreps should be appropriate for the item. Their effects and possible requirements are indicated on the attached lammies.

\textit{Vrede-tech}

Vrede-tech are items manufactured by, and only useable by, the vrede. They can take the form of various items, and the physreps should be appropriate for the item. Their effects and possible requirements are indicated on the attached lammies. Vrede-tech often has unique upgrades, and progression different to that of tech on the crafting lists (see \textit{Appendices E.} and \textit{F.}). If an item is vrede-tech, it will state so on its lammie.
